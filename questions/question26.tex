\section{Миноры. Алгебраические дополнения. Разложение определителя по строке/столбцу}

\begin{definition}
    Пусть $A$ --- произвольная матрица. Всякая матрица, составленная из элементов матрицы $A$, находящихся на пересечении каких-либо выбранных строк и каких-либо выбранных столбцов, называется \textbf{подматрицей} матрицы $A$.
\end{definition}

\begin{definition}
    Определитель квадратной подматрицы порядка $k$ называется \textbf{минором} порядка $k$ матрицы $A$. В частности, если $A$ --- квадратная матрицы порядка $n$, то минор поряка $n - 1$, получаемый вычёркиванием $i$-ой строки и $j$-го столбца, называется \textbf{дополнительным минором} элемента $a_{ij}$ и обозначается через $M_{ij}$.
\end{definition}

\begin{definition}
    Число 
    $$
    A_{ij} = (-1)^{i + j}M_{ij}
    $$
    называется \textbf{алгебраическим дополнением} элемента $a_{ij}$.
\end{definition}

\begin{theorem}[Формула разложения определителя по строке/столбцу]
    $$
    \det A = \sum_{j}a_{ij}A_{ij} = \sum_{i}a_{ij}A_{ij}.
    $$
\end{theorem}

Сначала докажем следующую лемму:

\begin{lemma}
    $$
    \det
    \begin{pmatrix}
        a_{11} & \cdots & a_{1j} & \cdots & a_{1n}\\
        \vdots & \ddots & \vdots & \ddots & \vdots\\
        0      & \cdots & a_{ij} & \cdots & 0     \\
        \vdots & \ddots & \vdots & \ddots & \vdots\\
        a_{n1} & \cdots & a_{nj} & \cdots & a_{nn}
    \end{pmatrix} = a_{ij}A_{ij}.
    $$
\end{lemma}

\begin{proof}
    Поменяем местами $i$-ую строку со всеми предыдущими, а затем $j$-ый столбец со всеми предыдущими. При этом всего мы произведём $i - 1$ перестановку строк и $j - 1$ перестановку столбцов. Поэтому результат домножится на
    $$
    (-1)^{i - 1 + j - 1} = (-1)^{i + j}.
    $$

    В результате получится определитель вида
    $$
    \det
    \begin{pmatrix}
        a_{1j} & 0      & \ldots & 0     \\
        a_{1j} & a_{11} & \cdots & a_{1n}\\
        \vdots & \vdots & \ddots & \vdots\\
        a_{nj} & a_{n1} & \cdots & a_{nn}
    \end{pmatrix} = a_{ij}M_{ij}.
    $$

    Последнее равенство выполняется в силу теоремы об определителе матрицы с углом нулей. С учётом вычисленного нами знака, отсюда и получается доказываемое равенство.
\end{proof}

Теперь несложно доказать теорему 26.1:

\begin{proof}
    Вспомним формулу для вычисления определителя:
    $$
    \det A = \sum_{\sigma \in S_n}\sgn\sigma a_{1\sigma()} \ldots a_{n\sigma(n)}.
    $$

    Каждое слагаемое содержит ровно 1 элемент из $i$-ой строки, а предыдущая лемма означает, что сумма тех членов, которые содержат $a_{ij}$ равна $a_{ij}A_{ij}$. Отсюда вытекает формула разложеиня по строке. Аналогично доказывается формула разложения по столбцу.
\end{proof}

\begin{remark}
    Антон Александрович рассказывал, как быстро понимать знак у алгебраического дополнения $A_{ij}$ (возведение $-1$ в степень --- сложная и трудоёмкая операция). Он предложил следующую визуализацию:
    $$
    \begin{pmatrix}
        $+$ & $-$ & $+$ & \cdots\\
        $-$ & $+$ & $-$ & \cdots\\
        $+$ & $-$ & $+$ & \cdots\\
        \vdots & \vdots & \vdots & \ddots
    \end{pmatrix}
    $$
\end{remark}


