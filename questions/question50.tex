\section{Поле частных целостного кольца. Вложение целостного кольца в своё поле частных Поле рациональных дробей. Формальное и функциональное равенство рациональных дробей}

Таким же образом, как кольцо целых чисел расширеяется до поля рациональных дробей, любое целостное кольцо можно расширить дополя.

Пусть $A$ --- целостное кольцо. Рассмотрим множество пар $(a, b)$, гед $a, b \in A$, $b \ne 0$, и определим в нём отношение эквивалентности 
$$
(a_1, b_1) \sim (a_2, b_2) \Leftrightarrow a_1b_2 = a_2b_1.
$$

Рефлексивность и симметричность этого отношения очевидны; докажем его транзитивность. Если $(a_1, b_1) \sim (a_2, b_2)$ и $(a_2, b_2) \sim (a_3, b_3)$, то
$$
a_1b_2b_3 = a_2b_1b_3 = a_3b_1b_2,
$$
откуда после сокращения на $b_2$ получаем
$$
a_1b_3 = a_3b_1,
$$
т.\,е. $(a_1, b_1) \sim (a_3, b_3)$.

Из данного определения следует, что
$$
(a, b) \sim (ac, bc)\eqno(\ast)
$$
для любого $c \ne 0$. С другой стороны, как показывает следующая ниже цепочка эквивалентностей, любая эквивалентность $(a_1, b_1) \sim (a_2, b_2)$ является следствием эквивалентностей типа $(\ast)$:
$$
(a_1, b_1) \sim (a_1b_2, b_1b_2) = (a_2b_1, b_1b_2) \sim (a_2, b_2).
$$

Определим теперь сложение и умножение пар по правилам
$$
(a_1, b_1) + (a_2, b_2) \vcentcolon = (a_1b_2 + a_2b_1, b_1b_2),\quad (a_1, b_1) \cdot (a_2, b_2) \vcentcolon = (a_1a_2, b_1b_2).
$$

Докажем, что определённое выше отношение эквивалентности согласовано с этими операциями. В силу предыдущего достаточно показать, что при умножении обоих членов одной из пар $(a_1, b_1)$ и $(a_2, b_2)$ на элемент $c \ne 0$ сумма и произведение этих пар заменятся эквивалентными им парами; но очевидно, что при такой операции оба члена суммы и произведения умножатся на тот же элемент $c$.

Класс эквивалентности, содержащий пару $(a, b)$, условимся записывать как дробь $a / b$. Ввиду доказанного выше операции сложения и умножения пар определяют операции сложения и умножения дробей, осуществляемые по обычным правилам:
$$
\frac{a_1}{b_1} + \frac{a_2}{b_2} = \frac{a_1b_2 + a_2b_1}{b_1b_2},\quad \frac{a_1}{b_1} \cdot \frac{a_2}{b_2} = \frac{a_1a_2}{b_1b_2}.
$$

Докажем, что относительно этих операций дроби образуют поле.

Очевидно, что сложение дробей коммутативно и ассоциативно. Дробь $\frac{0}{1}$ служит нулём для операции сложения дробей, а дробь $\displaystyle -\frac{a}{b}$ противоположна дроби $\displaystyle\frac{a}{b}$. Таким образом, дроби образуют абелеву группу относительно сложения.

Коммутативность и ассоциативность умножения очевидны. Следующая цепочка равенств доказывает дистрибутивность умножения дробей относительно сложения (две дроби приведём к общему знаменателю):
$$
\left(\frac{a_1}{b} + \frac{a_2}{b}\right)\frac{a_3}{b_3} = \frac{(a_1 + a_2)a_3}{bb_3} = \frac{a_1a_3 + a_2a_3}{bb_3} = \frac{a_1}{b}\frac{a_3}{b_3} + \frac{a_2}{b}\frac{a_3}{b_3}.
$$

Дробь $1 / 1$ служит единицей для операции умножения дробей, а при $a \ne 0$ дробь $\displaystyle\frac{b}{a}$ обратна дроби $\displaystyle\frac{a}{b}$.

\begin{definition}
    Построенно поле называется \textbf{полем частных целостного кольца} $A$ и обозначается через $\mathrm{Quot}\,A$.
\end{definition}

Сложение и умножения дробей вида $a / 1$ сводятся к соответствующим операциям над их числителями. Кроме того, $a / 1 = b / 1$ только при $a = b$. Следовательно, дроби такого вида образуют подкольцо, изоморфное $A$. Условившись отождествлять дробь вида $a / 1$ с элементом $a$ кольца $A$, мы получим вложение кольца $A$ в поле $\mathrm{Quot}\,A$.

Далее, поскольку 
$$
\frac{a}{b}\frac{b}{1} = \frac{a}{1},
$$
дробь $a / b$ равна отношению элементов $a$ и $b$ кольца $A$ в поле $\mathrm{Quot}\,A$. 

В силу $(a, c) \sim (ac, bc)$ дробь не изменится, если её числитель и знаменатель умножить или разделить (если это возможно) на один и тот же элемент кольца $A$. Если $A$ --- евклидово кольцо, то путём сокращения числителя и знаменателя на их наибольший общий делитель любая дробь приводится к виду $a / b$, где $\gcd(a, b) = 1$. 

\begin{definition}
    Такой вид дроби называется \textbf{несократимым}.
\end{definition}

\begin{statement}
    Любой вид дроби над евклидовым кольцом получается из любого её несократимого вида умножением числителя и знаменателя на один и тот же элемент.
\end{statement}

\begin{proof}
    Пусть $\displaystyle\frac{a}{b} = \frac{a_0}{b_0}$, причём $(a_0, b_0) = 1$. Из равенства $ab_0 = a_0b$ следует, что $b_0 \mid a_0b$ и, значит, $b_0 \mid b$. Пусть $b = cb_0$; ясно, что тогда $a = ca_0$.
\end{proof}

\begin{remark}
    Как следствие, несократимый вид дроби над евклидовым кольцом определён однозначно с точностью до умножения числителя и знаменателя на один и тот же необратимый элемент.
\end{remark}

\begin{definition}
    Поле частных кольца $\mathbb{F}[x]$ многочленов над полем $\mathbb{F}$ называется \textbf{полем рациональных дробей} над полем $\mathbb{F}$ и обозначается $F(x)$.
\end{definition}

Каждая рациональная дробь определяет функция на $\mathbb{F}$ со значениями в $\mathbb{F}$, определённую там, где её знаменатель (в несократмой записи) не обращается в ноль.

\begin{definition}
    Назовём две рациональные функции равными, если на множестве, где оба заменателя не равны нулю, эти функции равны.
\end{definition}

\begin{theorem}
    Если две рациональные дроби формально равны, то они функционально равны. Если же поле $\mathbb{F}$ бесконечно, то верно и обратное.
\end{theorem}

\begin{proof}
    Пусть $\frac{f}{g} = \frac{h}{s}$. Возьмём $a \in \mathbb{F}$ такое, что $g(a) \ne 0$ и $s(a) \ne 0$. Тогда $f(x)s(x) = g(x)h(x)$. Следовательно, $f(a)s(a) = g(a)h(a)$, что влечёт $\frac{f(a)}{g(a)} = \frac{h(a)}{s(a)}$.

    Наоборот, пусть поле $\mathbb{F}$ бесконечно и две функции $\frac{f(x)}{g(x)}$ и $\frac{h(x)}{s(x)}$ совпадают везде, кроме корней $g$ и $s$. Тогда их разность
    $$
    \frac{f(x)}{g(x)} - \frac{h(x)}{s(x)} = \frac{f(x)s(x) - h(x)g(x)}{g(x)s(x)}
    $$
    обращается в ноль везде, кроме корней $g$ и $s$. То есть, многочлен $f(x)s(x) - h(x)g(x)$ имеет бесконечное количество корней. Значит, этот многочлен формально равен нулю. Следовательно, $\frac{f}{s} = \frac{h}{s}$.
\end{proof}


