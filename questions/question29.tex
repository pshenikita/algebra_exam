\section{Теорема о ранге матрицы. Метод окаймляющх миноров}

\begin{theorem}[О ранге матрицы]
    Пусть $A \in \underset{m \times n}{\mathrm{Mat}}$. Ранг матрицы $A$ равен максимальному порядку ненулевого минора этой матрицы.
\end{theorem}

\begin{proof}
    Для того, чтобы доказать заявленное равенство, докажем неравенства в одну и другую сторону. Сперва докажем, что ранг не меньше, чем порядок любого ненулевого минора данной матрицы. Пусть в $A$ есть ненулевое минор порядка $k$, и пусть это $M_{i_1, \ldots, i_k}^{j_1, \ldots, j_k}$. Так как определитель этой $k \times k$ матрицы ненулевой, её строки линейно независимы. Однако строки данной матрицы --- это строки $A_{(i_1)}, \ldots, A_{(i_k)}$, в которых убраны некоторые координаты с одинаковыми номерами. Следовательно, строки $A_{(i_1)}, \ldots, A_{(i_k)}$ линейно независимы, то есть, $\rk A \geqslant k$.

    Пусть теперь $\rk A = k$. Докажем, что в $A$ найдётся ненулевой минор порядка $k$. Выберем базисные строки $A_{(i_1)}, \ldots, A_{(i_k)}$ и рассмотрим подматрицу $P$ размера $k \times n$ в матрицы $A$ состоящую из этих строк. Так как строки $P$ линейно независимы, то $\rk P = k$. Выберем базисные столбцы матрицы $P$, их ровно $k$ штук, они линейно независимы. Это в точности подматрицы порядка $k \times k$ данной матрицы и её ранг равен $k$, а значит, её определитель ненулевой.
\end{proof}

\begin{definition}
    Минор $(k + 1) \times (k + 1)$ называется \textbf{окаймляющим} минором для данного минора $k \times k$, если соответствующая подматрицы получена добавлением одной строки и одного столбца к подматрице минора $k \times k$.
\end{definition}

\begin{theorem}[Метод окаймляющих миноров]
    Если данный минор $k \times k$ матрицы $A$ не равен нулю, а все его окаймляющие миноры равны нулю, то $\rk A = k$.
\end{theorem}

\begin{proof}
    По теореме о ранге матрицы $\rk A \geqslant k$. Допустим, что $\rk A \geqslant k + 1$. Пусть данный нам ненулевой минор стоит на пересечении строк с номерами $i_1, \ldots, i_k$ и столбцов с номерами $j_1, \ldots, j_k$. Тогда существует строка с номером $i_{k + 1} \notin \{i_1, \ldots, i_k\}$, такая что система строк с номерами $i_1, \ldots, i_k, i_{k + 1}$ линейно независима. Рассмотрим матрицу $P$ размера $(k + 1) \times n$, состоящую из этих строк. Ранг этой матрицы $k + 1$. Тогда столбцы $P^{(j_1)}, \ldots, P^{(j_k)}$ линейно независимы, т.\,к. даже есть убрать строчку $i_{k + 1}$, то они будут таковыми. Дополним столбцы $P^{(j_1)}, \ldots, P^{(j_k)}$ до базиса системы столбцов $P$ некоторым столбцом $P^{(j_{k + 1})}$. Матрицы, состоящая из столбцов $P^{(j_1)}, \ldots, P^{(j_k)}, P^{(j_{k + 1})}$ имеет ненулевой определитель. Но это подматрицы $(k + 1) \times (k + 1)$ в $A$, что противоречит условию.
\end{proof}

