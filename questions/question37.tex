\section{Модуль и аргумент комплексного числа. Сопряжение и его свойства. Вещественная и мнимая части комплексного числа. Алгебраическая и тригонометрическая форма записи комплексного числа, переход между ними. Деление чисел в алгебраической форме}

Каждому комплексному числу $z = a + bi$ можно биективно сопоставить точку плоскости $(a, b)$.

\begin{definition}
    \textbf{Модулем} комплексного числа $z = a + bi$ называется неотрицательное вещественное число $|z| = \sqrt{a^2 + b^2}$ (корень арифметический).
\end{definition}

\begin{definition}
    \textbf{Аргументом} ненулевого комплексного числа $z = a + bi \ne 0$ --- это величина угла между положительным лучом оси абсцисс и радиус-вектором точки $(a, b)$, причём этот угол откладывается против часовой стрелки. Обозначается аргумент $\arg z$.
\end{definition}

\begin{definition}
    Пусть $z = a + bi$. Тогда \textbf{комплексно сопряжённое} к $z$ число --- это $\overline{z} = a - bi$.
\end{definition}

\begin{theorem}
    Сопряжение является автоморфизмом поля $\C$.
\end{theorem}

\begin{proof}
    Сначала докажем, что сопряжение является гомоморфизмом $\C \rightarrow \C$. Для этого нужно доказать, что оно сохраняет операции:
    $$
        \overline{(a + bi) + (c + di)} = \overline{(a + c) + (b + d)i} = (a + c) - (b + d)i = \overline{a + bi} + \overline{c + di},
    $$
    $$
        \overline{(a + bi)\cdot (c + di)} = \overline{(ac - bd) + (ad + bc)i} = (ac - bd) - (ad + bc)i = \overline{a - bi} \cdot \overline{c - di}.
    $$

    Очевидно также, что сопряжение отображает $\C$ на $\C$ биективно. То есть, является автоморфизмом.
\end{proof}

\begin{definition}
    \textbf{Вещественная часть} числа $z = a + bi$ --- это число $\Re z = a$. \textbf{Мнимая часть} числа $z$ --- это число $\Im z = b$.
\end{definition}

\begin{definition}[Алгебраическая запись комплексного числа]
    $z = \Re z + i \cdot \Im z$.
\end{definition}

\begin{definition}[Тригонометрическая запись комплексного числа]
    $z = |z|(\cos\arg z + i \cdot \sin\arg z)$.
\end{definition}

При этом выполнено
$$
\Re z = |z|\cos\arg z,\quad \Im z = |z|\sin\arg z.
$$

\begin{lemma}
    Тригонометрическая форма числа единственная.
\end{lemma}

\begin{proof}
    Пусть
    $$
    z = r(\cos\varphi + i\sin\varphi) = s(\cos\psi + i\sin\psi).
    $$
    Тогда $|z| = \sqrt{r^2(\cos^2\varphi + \sin^2\varphi)} = r = \ldots = s$. Значит, модули равны. Приравнивая вещественные и мнимые части, получаем $\cos\varphi = \cos\psi$, $\sin\varphi = \sin\psi$, отсюда $\varphi = \psi$ с точностью до $2\pi k$.
\end{proof}

\textbf{Деление чисел в алгебраической форме}. Домножаем на сопряжённое:
$$
\frac{a + bi}{c + di} = \frac{(a + bi)(c - di)}{(c + di)(c - di)} = \frac{(a + bi)(c - di)}{c^2 + d^2}.
$$


