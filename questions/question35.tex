\section{Характеристика поля. Какие значения может принимать характеристика? Возведение суммы в степень, равную характеристике. Малая теорема Ферма}

\begin{definition}
    Характеристика $\mathrm{char}\,\mathbb{F}$ поля $\mathbb{F}$ равна наимньшему натуральному $k$ такому, что сумма $k$ единиц равна нулю, если такое натуральное $k$ существует. А иначе говорят, что $\mathrm{char}\,\mathbb{F} = 0$.
\end{definition}

\begin{theorem}
    Характеристика поля либо равна нулю, либо является простым числом.
\end{theorem}

\begin{proof}
    Допустим, что $\mathrm{char}\, \mathbb{F} = mn$. Тогда
    $$
    \underbrace{1 + 1 + \ldots + 1}_{\text{$mn$ раз}} = \underbrace{1 + 1 + \ldots + 1}_{\text{$m$ раз}} + \underbrace{1 + 1 + \ldots + 1}_{\text{$n$ раз}} 
    $$

    Так как в поле нет делителей нуля, один из множителей равен $0$.
\end{proof}

\begin{lemma}
    Пусть $\mathbb{F}$ --- поле характеристики $p$. Если сложить элемент $a \in \mathbb{F}$ с собой $pk$ раз, то получится $0$.
\end{lemma}

\begin{proof}
    $$
    \underbrace{a + a + \ldots + a}_{\text{$pk$ раз}} = \underbrace{1 + 1 + \ldots + 1}_{\text{$p$ раз}}ka = 0\cdot ka = 0.
    $$
\end{proof}

\begin{theorem}
    Пусть $\mathbb{F}$ --- поле характеристики $p$. Тогда для $a, b \in \mathbb{F}$ выполнено 
    $$(a_1 + a_2 + \ldots + a_k)^p = a_1^p + a_2^p + \ldots + a_k^p.$$
\end{theorem}

\begin{proof}
    Докажем индукцией по количеству слагаемых $k$.

    \textbf{База индукции} ($k = 2$). По формуле бинома Ньютона
    $$
    (a + b)^p = \sum_{k = 0}^p\binom{p}{k}a^kb^{p - k}.
    $$

    При этом $p \mid \binom{p}{k} = \frac{p!}{k!(p - k!)}$ при $i = 1, 2, \ldots, p - 1$. Таким образом, в поле $\mathbb{F}$ все слагаемые, кроме крайних, равны нулю.

    \textbf{Шаг индукции}. Пусть утверждение верно для всех $k < K$. Тогда
    $$
    (a_1 + \ldots + a_{K - 1} + a_K)^p = ((a_1 + \ldots + a_{K - 1}) + a_K)^p = (a_1 + \ldots + a_{K - 1})^p + a_K^p = a_1^p + \ldots + a_{K - 1}^p + a_K^p.
    $$
\end{proof}

\begin{theorem}[Малая теорема Ферма]
    Пусть $p$ --- простое число. Для любого целого числа $n$ выполнено
    $$
    n^p \equiv n\ (\mod p).
    $$
\end{theorem}

\begin{proof}
    Утверждение теоремы равносильно тому, что в кольце $\Z_p$ выполнено $[n]^p = [n]$. Это следует из цепочки равенств
    $$
    [n]^p = ([1] + \ldots + [1])^p = [1]^p + \ldots + [1]^p = [n].
    $$
\end{proof}


