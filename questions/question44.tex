\section{Деление многочленов от одной переменной над полем с остатком. Наибольший общий делитель. Алгоритм Евклида. Линейное выражение НОД. Доказательство того, что НОД делится на все общие делители}

\begin{theorem}
    Пусть $f, g \in \mathbb{F}[x]$, причём $g \ne 0$. Тогда существуют такие многочлены $p$ и $r$, что $f = qg + r$ и $\deg r < \deg g$. Многочлены $q$ и $r$ определены этими условиями однозначно.
\end{theorem}

\begin{proof}
    Докажем сначала существование, затем единственность указанного разложения. Если $\deg f < \deg g$, то можно взять $q = 0$, $r = f$. Если $\deg f \geqslant \deg g$, то $q$ и $r$ находятся процедурой <<деления уголком>>. А именно, пусть
    $$
    f = a_0x^n + a_1x^{n - 1} + \ldots + a_{n - 1}x + a_n,
    $$
    $$
    g = b_0x^m + b_1x^{m - 1} + \ldots + b_{m - 1}x + b_n,
    $$
    где $a_0, b_0 \ne 0$. Рассмотрим многочлен
    $$
    f_1 = f - \frac{a_0}{b_0}x^{n - m}g.
    $$

    Его степень меньше, чем степень многочлена $f$. Если $\deg f_1 < \deg g$, то мы можем взять
    $$
    q = \frac{a_0}{b_0}x^{n - m},\quad r = f_1.
    $$

    В противном случае поступаем с многочленом $f_1$ так же, как с $f$.

    Теперь единственность. Пусть
    $$
    f = q_1g + r_1 = q_2g + r_2,
    $$

    где $\deg r_1 < \deg g$ и $\deg r_2 < \deg g$. Тогда
    $$
    r_1 - r_2 = (q_2 - q_1)g
    $$
    и, если $q_1 \ne q_2$, то
    $$
    \deg(r_1 - r_2) = \deg(q_2 - q_1) + \deg g \geqslant \deg g,
    $$
    что, очевидно, неверно. Следовательно, $q_1 = q_2$ и $r_1 = r_2$.
\end{proof}

\begin{definition}
    \textbf{Наибольшим общим делителем} элементов $a$ и $b$ целостного кольца называется их общий делитель, делящийся на все их общие делители. Оно обозначается через $\gcd(a, b)$.
\end{definition}

\begin{definition}
    Целостное кольцо $A$, не являющееся полем, называется \textbf{евклидовым}, если существует функция
    $$
    N: A \setminus \{0\} \rightarrow \Z_+
    $$
    (называется \textbf{нормой}), удовлетворяющая следующим аксиомам:
    \begin{enumerate}[nolistsep]
        \item $N(ab) \geqslant N(a)$, причём равенство достигается тогда и только тогда, когда элемент $b$ обратим;
        \item для любых $a, b \in A$, где $b \ne 0$, существуют такие $q, r \in A$, что $a = qb + r$ и либо $r = 0$, либо $N(r) < N(b)$;
    \end{enumerate}
\end{definition}

\begin{remark}
    Условие 2 означает возможнось деления с остатком. Его единственности не требуется.

    Вторая часть условия 1 на самом деле может быть выведена из остальных условий. В самом деле, пусть элемент $b$ необратим. Тогда $ab \nmid a$. Разделим $a$ на $ab$ с остатком:
    $$
    a = q(ab) + r.
    $$
    Так как $r = a(1 - ab)$, то
    $$
    N(a) \leqslant N(r) < N(ab).
    $$
\end{remark}

\textbf{Примеры евклидовых колец}:
\begin{enumerate}[nolistsep]
    \item Целые числа: $\Z$ (в качестве нормы можно взять модуль).
    \item Гауссовы целые числа: $\Z[i]$ (в качестве нормы можно опять же взять модуль).
    \item Кольцо многочленов над полем: $\mathbb{F}[x]$ (в качестве нормы можно взять степень многочлена).
\end{enumerate}

\begin{theorem}
    В евклидовом кольце для любых элементов $a$ и $b$ существует наибольший общий делитель $d$, и он может быть представлен в виде $d = au + bv$, где $u, v$ --- какие-то элементы кольца.
\end{theorem}

\begin{proof}
    Если $b = 0$, то $d = a = a \cdot 1 + b \cdot 0$. Если $b \mid a$, то $d = b = a \cdot 0 + b \cdot 1$. В противном случае разделим с остатком $a$ на $b$, затем $b$ на полученный остаток, затем первый остаток на второй остаток и т.\,д. Так как нормы остатков убывают, то в конце деление произойдёт без остатка. Получим цепочку равенств:
    $$
    \begin{array}{l}
        a = q_1b + r_1,\\
        b = q_2r_1 + r_2,\\
        r_1 = q_3r_2 + r_2,\\
        \vdots\\
        r_{n - 2} = q_nr_{n - 1} + r_n,\\
        r_{n - 1} = q_{n + 1}r_n.
    \end{array}
    $$
    Докажем, что последний ненулевой остаток $r_n$ и есть $\gcd(a, b)$. Двигаясь по выписанной цепочке равенств снизу вверх, получам последовательно
    $$
    r_n \mid r_{n - 1},\quad r_n \mid r_{n - 2},\quad \ldots,\quad r_n \mid r_1,\quad r_n \mid b,\quad r_n \mid a.
    $$
    Таким образом, $r_n$ --- общий делитель элементов $a$ и $b$.

    Двигаясь по той же цепочке равенств снизу вверх, получаем последовательно
    $$
    \begin{array}{l}
        r_1 = au_1 + bv_1,\\
        r_2 = au_2 +bv_2,\\
        \vdots\\
        r_n = au_n = bv_n,
    \end{array}
    $$
    где $u_i, v_i$ ($i = 1, 2, \ldots, n$) --- какие-то элементы кольца. Таким образом, $r_n$ можно представить в виде $au + bv$. Отсюда, в свою очередь, следует, что $r_n$ делится на любой общий делитель $a$ и $b$.
\end{proof}

\begin{definition}
    Необратимый нулевой элемент $p$ целостного кольца называется \textbf{простым}, если он не может быть представлен в виде $p = ab$, где $a$ и $b$ --- необратимые элементы. Простые элементы кольца $\mathbb{F}[x]$, где $\mathbb{F}$ --- поле, называются \textbf{неприводимыми многочленами}.
\end{definition}

Очевидно, что всякий многочлен первой степени неприводим. Из основной теоремы алгебры вытекает, что неприводимые многочлены над $\C$ --- это только многочлены первой степени, а над $\R$ --- это многочлены первой степени и многочлены второй степени с отрицательным дискриминантом.

\begin{lemma}
    Если простой элемент $p$ евклидова кольца $A$ делит произведение $a_1a_2\ldots a_n$, то он делит хотя бы один из сомножителей $a_1, a_2, \ldots, a_n$.
\end{lemma}

\begin{proof}
    Докажем это утверждение индукцией по $n$. 

    \textbf{База индукции} ($n = 2$). Предположим, что $p \nmid a_1$. Тогда $\gcd(p, a_1) = 1$ и, значит, существует такие $u, v \in A$, что $pu + a_1v = 1$. Умножая это равенство на $a_2$, получаем
    $$
    pua_2 + a_1a_2v = a_2,
    $$
    откуда следует, что $p \mid a_2$.

    \textbf{Шаг индукции}. Представим 
    $$
    a_1a_2\ldots a_n = a_1 \cdot (a_2\ldots a_n).
    $$
    Теперь, применяя базу, получаем $p \mid a_1$ или $p \mid a_2\ldots a_n$. Если первое не выполняется, то применяем предположение индукции.
\end{proof}

\begin{definition}
    Два элемента $a$ и $b$ кольца $\mathcal{R}$ называются \textbf{ассоциированными}, если $a = bc$, где $c \in \mathcal{R}$ обратим.
\end{definition}

\begin{statement}
    Отношение ассоциированности является отношением эквивалентности.
\end{statement}

\begin{proof}
    Если $a = bc$, то $b = ac^{-1}$, поэтому отношение ассоциированности симметрично. Т.\,к. $1$ --- обратимый элемент $\mathcal{R}$, то отношение ассоциированности рефлексивно. И т.\,к. произведение двух обратимых элементов обратимо, данное отношение транзитивно.
\end{proof}

\begin{remark}
    Получаем, что все элементы $\mathcal{R}$ распадаются на классы ассоциированности. В доказательстве следующей теоремы мы будем писать $p \sim q$, если элементы $p$ и $q$ ассоциированны.
\end{remark}

\begin{theorem}
    В евклидовом кольце всякий необратимый ненулевой элемент может быть разложен на простые множители, причём это разложение единственно с точностью до перестановки множителей и их ассоциированности.
\end{theorem}

\begin{proof}
    \textbf{Существование}. Назовём необратимый ненулевой элемент $a \in A$ пацанским, если он может быть разложен на простые множители, остальные элементы будем называть чушпанскими. Предположим, что существуют чушпанские элементы. Выберем из них элемент с наименьшей нормой. Пусть это будет элемент $a$. Он не может быть простым. Следовательно, $a = bc$, гед $b$ и $c$ --- необратимые элементы. Имеем $N(b) < N(a)$ и $N(c) < N(a)$ и, значит, $b$ и $c$ --- пацанские элементы; но тогда, очевидно, и $a$ --- пацанский элемент, что противоречит нашему предположению. Таким образом, всякий необратимый ненулевой элемент кольца $A$ может быть разложен на простые множители. 

    \textbf{Единственость}. Докажем индукцией по $n$, что если
    $$
    a = p_1p_2\ldots p_n = q_1q_2\ldots q_m,
    $$
    где $p_i$, $q_j$ --- простые элементы, то $m = n$ и, после подходящей перенумерации множителей, $p_i \sim q_i$ при $i = 1, 2, \ldots, n$.

    \textbf{База индукции} ($n = 1$). $a = p_1$ --- простой, поэтому и $m = 1$, причём $q_1 \sim p_1$ (из определения простого элемента).

    \textbf{Шаг индукции}. При $n > 1$ имеем $p_1 \mid q_1q_2\ldots q_m$, а из предыдущей леммы это значит, что $p_1 \mid q_i$. Перенумеруем так, что $i = 1$. Тогда $p_1 \sim q_1$. Сокращая на $p_1$ и применяя предположение индукции, получаем требуемое.
\end{proof}

\begin{remark}
    Это утверждение над евклидовым кольцом $\Z$ называется основной теоремой арифметики.
\end{remark}


