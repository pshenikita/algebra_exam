\section{Решения однородной системы образуют подпространство. Линейная оболочка: опредление и доказательство того, что это подпространство}

\begin{theorem}
    Совокупность всех решений системы однородных линейных уравнений с $n$ неизвестными является подпространством пространства $\mathcal{K}^n$.
\end{theorem}

\begin{proof}
    Рассмотрим произвольную систему однородных линейных уравнений:
    $$
    \begin{cases}
        a_{11}x_1 + a_{12}x_2 + \ldots + a_{1n}x_n = 0,\\
        a_{21}x_1 + a_{22}x_2 + \ldots + a_{2n}x_n = 0,\\
        \ldots\\
        a_{m1}x_1 + a_{m2}x_2 + \ldots + a_{mn}x_n = 0,\\
    \end{cases}
    $$

    Очевидно, что нулевая строка является её решением и что произведение любого решения на число также является решением. Докажем, что сумма решений $(u_1, \ldots, u_n)$ и $(v_1, \ldots, v_n)$ является решением. Подставляя её компоненты в $i$-ое уравнение системы, получаем:
    $$
    a_{i1}(u_1 + v_1) + \ldots + a_{in}(u_n + v_n) = \underbrace{a_{i1}u_1 + \ldots + a_{in}u_n}_{{} = 0} + \underbrace{a_{i1}v_1 + \ldots + a_{in}v_n}_{{} = 0} = 0.
    $$
\end{proof}

\begin{theorem}
    Совокупность всех решений произвольной совместной системы линейных уравнений есть сумма какого-либо одного её решения и подпространства решений системы однородных линейных уравнений с той же матрицей коэффициентов.
\end{theorem}

\begin{proof}
    Пусть $u \in \mathcal{K}^n$ --- какое-либо фиксированное решение системы
    $$
    \begin{cases}
        a_{11}x_1 + a_{12}x_2 + \ldots + a_{1n}x_n = b_1,\\
        a_{21}x_1 + a_{22}x_2 + \ldots + a_{2n}x_n = b_2,\\
        \ldots\\
        a_{m1}x_1 + a_{m2}x_2 + \ldots + a_{mn}x_n = b_m,\\
    \end{cases}\eqno(\ast)
    $$

    Аналогично предыдущему доказывается, что сумма решения $u$ и произвольного решения однородной системы является решением системы $(\ast)$. Обратно, если $u^\prime$ --- любое решение системы $(\ast)$, то $v = u^\prime - u$ --- решение однородной системы.
\end{proof}

\begin{remark}[из Винберга]
    Неопределённые системы линейных уравнений могут иметь различную <<степень неопределённости>>, каковой естественно считать число свободных неизвестных в общем решении системы. Однако одна и та же система линейных уравнений может допускать различные общие решения, в которых разные неизвестные играют роль свободных, и закономерен вопрос, будет ли число свободных неизвестных всегда одним и тем же. Положительный ответ на этот вопрос даётся с помощью понятия размерности векторного пространства.
\end{remark}

\begin{definition}
    Пусть $S \subset V$ ($V$ --- векторное пространство). Совокупность всевозможных линейных комбинаций векторов из $S$ называется \textbf{линейной оболочкой} множества $S$ и обозначается через $\langle S\rangle$. Говорят, что пространство $V$ \textbf{порождается} множеством $S$, если $\langle S\rangle = V$.
\end{definition}

\begin{definition}
    Пространство называется  \textbf{конечномерным}, если оно порождается конечным количеством векторов и \textbf{бесконечномерным}, если оно порождается бесконечным количеством векторов.
\end{definition}

\begin{remark}
    Я не заметил важной ремаки в конспектах Сергей Александровича --- в курсе мы считаем вообще все пространства конечномерными. С бесконечномерными возникают большие проблемы --- у нас даже нет определения линейной зависимости для бесконечного количества векторов. В дальнейшем мы будем доказывать утверждения для конечных систем, подразумевая, что для бесконечных они тоже верны (чаще всего достаточно убрать верхний предел суммирования). Цитата из Винберга по этому поводу: <<Понятия базиса и размерности могут быть перенесены на бесконечномерные векторные пространства. Чтобы это сделать, надо определить, что такое линейная комбинация бесконечной системы векторов. В чисто алгебраической ситуации нет иного выхода, кроме как ограничиться рассмотрением линейных комбинаций, в которых лишь конечное число коэффициентов отлично от нуля>>. И далее следует определение линейной комбинации для бесконечной системы векторов как выражение вида $\displaystyle \sum_{i \in I}\lambda_ia_i$, в котором лишь конечное число коэффициентов $\lambda_i$ отлично от нуля, так что сумма фактически является конечной и, таким образом, имеет смысл. Потом можно определить и линейную зависимости, и базис, и т.\,д. Но понятно, что так решается проблема только для счётных множеств. Что делать с несчётными, неясно совсем.
\end{remark}

\begin{theorem}
    Пусть $S \subset V$ ($V$ --- векторное пространство). Тогда линейная оболочка $\langle S\rangle$ является подпространством в $V$.
\end{theorem}

\begin{proof}
    По теореме 4.2 достаточно проверить, что линейная оболочка замкнута относительно операций сложения векторов и умножения вектора на число. Это сразу следует из определения.
\end{proof}

\begin{statement}[упражнение из Винберга]
    $\langle S\rangle$ --- это наименьшее подпространство в $V$, содержащее $S$.
\end{statement}

\begin{proof}
    Предположим, что из $\langle S\rangle$ можно убрать элемент $v^\prime = \lambda_1v_1 + \ldots + \lambda_kv_k$. Но мы можем сконструировать $v^\prime$ из $v_1, \ldots, v_k$ с помощью операций умножения на число и сложения векторов, поэтому (из теоремы 4.2) он должен лежать в подпространстве. Противоречие. Значит, никакой элемент $\langle S\rangle$ нельзя убрать, чтобы оно осталось подпространством.
\end{proof}

