\section{Основная теорема алгебры. Комплексные корни вещественных многочленов. Неприводимые многочлены над $\C$ и $\R$. Разложение комплексных и вещественных многочленов на неприводимые множители (существование)}

\begin{theorem}
    Всякий многочлен положительной степени над полем комплексных чисел имеет корень.
\end{theorem}

\begin{proof}
    По лемме о возрастании модуля существует $D \in \R$ такое, что при $|z| > D$ выполнено $|f(z)| > |f(0)|$. Рассмотрим круг $K = \{|z| \leqslant 2D\}$. Это замкнутое и ограниченное множество, т.\,е. компакт. Значит, существует точка $w \in K$, в которой достигается минимум функции $|f(z)|$. Заметим, что $w$ не лежит на границе $K$, т.\,к. на границе данная функция больше, чем в точке $0 \in K$. Значит, существует $\varepsilon \in \R$, $\varepsilon > 0$ такое, что $U_\varepsilon(w) \subset K$. Если $f(w) = 0$, то корень найден. Допустим, что $f(w) \ne 0$. Тогда по лемме Даламбера существует $w^\prime \in U_\varepsilon(w)$ такое, что $|f(w^\prime)| < |f(w)|$. Но $w^\prime \in K$. Получаем противоречие с выбором $w$.
\end{proof}

\begin{lemma}
    Если $z \in \C$ --- корень многочлена $f \in \R[x]$, то и $\overline{z} \in \C$ --- тоже корень.
\end{lemma}

\begin{proof}
    Сразу следует из того, что сопряжение является автоморфизмом на $\C$.
\end{proof}

\begin{statement}
    Любой многочлен с вещественными коэффициентами раскладывается в произведение линейных и квадратичных с отрицательным дискриминантом множителей.
\end{statement}

\begin{proof}
    Индукция по степени многочлена.

    \textbf{База индукции}. $\deg f = 0$ и $\deg f = 1$ очевидна.

    \textbf{Шаг индукции}. Пусть $f \in \R[x]$. Если $f$ имеет вещественный корень $c$, то $f(x) = (x - c)g(x)$, при этом $\deg g < \deg f$ и к $g$ можно применить предположение индукции.

    Пусть теперь у $f(x)$ есть комплексный корень $\lambda$, тогда $\overline{\lambda}$ --- тоже корень. Значит, 
    $$
    f(x) = (x - \lambda)(x + \lambda)h(x) = (x^2 - 2\Re\lambda + |\lambda|^2)h(x).
    $$
    При этом дискриминант квадратного полученного трёхчлена отрицательный, и к $h$ можно применить предположение индукции.
\end{proof}


