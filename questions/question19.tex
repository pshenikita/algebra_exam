\section{Алгоритм поиска обратной матрицы. Разложение невырожденной матрицы в произведение элементарных}

Чтобы найти обратную матрицу, нам нужно решить матричное уравнение
$$
AX = E.
$$

Для этого нам нужно решить $n$ систем уравнений с одной и той же матрицей коэффициентов $A$, столбцы свободных членов которых составляют матрицу $E$. Эти системы можно решать одновременно методом Гаусса. После приведения матрицы коэффициентов к единичной матрице (что возможно в силу её невырожденности) преобразованные столбцы свободных членов составят искомую матрицу $A^{-1}$.
$$
(A \mid E) \leadsto \ldots \leadsto (E \mid A^{-1}).
$$

А, как уже обсуждалось, метод Гаусса заключается в домножении матрицу на элементарные слева. А т.\,к. улучшенный ступенчатый вид невырожденной матрицы единичный, то получаем
$$
A = U_1A_1 = U_2U_1A_2 = \ldots = U_N\ldots U_1 E = U_N\ldots U_1,
$$
где $U_i$ --- элементарные матрицы.

Александр Александрович Гайфуллин рассказал полезный трюк. Часто пригождается решать матричные уравнения типа
$$
AX = B.
$$

Решением является $A^{-1}$. Однако искать обратную к $A$, а затем умножать её на $B$ может быть затруднительно. Можно поступить тем же способом, которым мы пользовались ранее, заменив единичную матрицу на $B$ (ведь по сути ничего не меняется, нам всё ещё нужно решить $n$ СЛУ с $n$ неизвестными):
$$
(A \mid B) \leadsto \ldots \leadsto (E \mid A^{-1}B).
$$

Ещё это можно объяснить следующим образом (так это делал Сергей Александрович): элементарные преобразования строк --- это умножения слева на элементарные матрицы. Мы делаем над обеими частями матрицы $(A\mid B)$ одни и те же элементарные преобразования, в результате слева получаем $U_N\ldots U_1 \cdot A = E$, значит, $U_N\ldots U_1 = A^{-1}$, а справа $U_N\ldots U_1 B = A^{-1}B$.

