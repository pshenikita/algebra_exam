\section{Определение кольца и поля. Примеры. Простейшие следствия из аксиом. Обратимые элементы, делители нуля, нильпотенты. Взаимное расположение множеств обратимых элементов, делителей нуля и нильпотентов. Критерий того, что кольцо $\Z_n$ является полем}

\begin{definition}
    \textbf{Кольцом} называется множество $\mathcal{R}$ с операциями сложения и умножения, для которого выполнены следующие аксиомы:
    \begin{enumerate}[nolistsep]
        \item $\mathcal{R}$ есть абелева группа по сложению;
        \item $\forall\!\:a, b, c \in \mathcal{R}$ выполнено $a(b + c) = ab + ac$ и $(a + b)c = ac + bc$.
    \end{enumerate}
\end{definition}

\begin{theorem}[Простешие следствия из аксиом кольца]
    \begin{enumerate}[nolistsep]
        \item $0 \cdot a = a \cdot 0 = 0$;
        \item $a(-b) = (-a)b = -ab$;
        \item $a(b - c) = ab - ac$, $(a - b)c = ac - bc$;
        \item В кольце не может быть двух различных единиц (но может не быть ни одной);
        \item Если кольцо содержит более одного элемента, то $1 \ne 0$;
        \item[$+$] Свойства аддитивной абелевой группы.
    \end{enumerate}
\end{theorem}

\begin{proof}
    \begin{enumerate}[nolistsep]
        \item В самом деле, пусть $a \cdot 0 = b$. Тогда
            $$
            b + b = a \cdot 0 + a \cdot 0 = a \cdot (0 + 0) = a \cdot 0 = b \Rightarrow b = 0.
            $$
            Аналогично доказывается, что $0\cdot a = 0$.
        \item В самом деле, 
            $$
            ab + a(-b) = a(b + (-b)) = a \cdot 0 = 0.
            $$
        \item В самом деле,
            $$
            a(b - c) + ac = a(b - c + c) = ab
            $$
            и, аналогично, $(a - b)c = ac - bc$.
        \item Аналогично доказательству того же факта для группы
        \item Если $1 = 0$, то для любого элемента $a$ имеем
            $$
            a = a 1 = a 0 = 0,
            $$
            т.\,е. кольцо состоит из одного нуля.
    \end{enumerate}
\end{proof}

\begin{definition}
    Кольцо $\mathcal{R}$ называется \textbf{ассоциативным}, если $\forall\!\:a, b, c \in \mathcal{R}$ выполнено $a(bc) = (ab)c$, \textbf{коммутативным}, если $\forall\!\:a, b \in \mathcal{R}$ выполнено $ab = ba$, \textbf{кольцом с единицей}, если $\exists\!\:e \in \mathcal{R}: \forall\!\:r \in \mathcal{R}$ выполнено $re = er = r$ и \textbf{телом}, если оно ассоциативно и $\forall\!\: r \in \mathcal{R}\;\exists\!\:r^{-1}: rr^{-1} = r^{-1}r = e$.
\end{definition}

\textbf{Примеры колец}:
\begin{enumerate}
    \item Числовые множества $\Z$, $\Q$, $\R$ являются коммутативными ассоциативными кольцами с единицей относительно обычных операций сложения и умножения.
    \item Вычеты $\mod n$: $\Z_n$ является коммутативным ассоциативным кольцом с единицей относительно обычных операций сложения и умножения.
    \item Множество $2\Z$ чётных чисел является коммутативным ассоциативным кольцом без единицы.
    \item Множество всей функций, определённых на заданном подмножестве числовой прямой, является коммутативным ассоциативным кольцом с единицей относительно обычных операций сложения и умножения функций
    \item Множество векторов пространства с операцией сложения и векторного умножения является некоммутативным и неассоциативным кольцом. Однако в нём выполняются следующие тождества, которые в некотором смысле заменяют коммутативность и ассоциативность:
        $$
        \begin{array}{cl}
            {[a, b]} + [b, a] = 0 & \text{\itshape антикоммутативность}\\
            {[[a, b], c]} + [[c, a], b] + [[b, c], a] = 0 & \text{\itshape тождество Якоби}
        \end{array}
        $$
\end{enumerate}

\begin{lemma}
    Если $A$ --- любое ассоциативное кольцо с единицей, то множество его обратимых элементов $A^\ast$ является группой по умножению.
\end{lemma}

\begin{proof}
    Множество $A^\ast$ замкнуто относительно взятия обратного (по условию). Тогда если $a, b \in A^\ast$, то и $a^{-1}, b^{-1} \in A^\ast$, а отсюда следует замкнутность $A^\ast$ относительно умножения --- пусто не замкнуто, тогда $ab$ необратим, а это неправда --- $(ab)^{-1} = b^{-1}a^{-1}$. Ассоцитативность выполняется, потому что кольцо $A$ ассоциативно. А единица есть, потому что она есть в $A$ и из замкнутости: $A^\ast \ni aa^{-1} = 1$.
\end{proof}

\begin{remark}
    Как следствие, $\Z_n^\ast$ является мультипликативной группой.
\end{remark}

\begin{definition}
    \textbf{Полем} называется коммутативное тело.
\end{definition}

\begin{remark}
    Кольцо, состоящее из одного нуля, не считается полем.
\end{remark}

\textbf{Примеры полей}:
\begin{enumerate}[nolistsep]
    \item $\Q$, $\R$.
    \item $\Z$ не является полем, в нём обратимы только $\pm 1$.
    \item $\{0, 1\}$.
\end{enumerate}

\begin{definition}
    Подмножество $\mathcal{L}$ кольца $\mathcal{R}$ называется \textbf{подкольцом}, если
    \begin{enumerate}[nolistsep]
        \item $\mathcal{L}$ является подгруппой аддитивной группы кольца $\mathcal{R}$
        \item $\mathcal{L}$ замкнуто относительно умножения
    \end{enumerate}
\end{definition}

Очевидно, что всякое подкольцо само является кольцом относительно тех же операций. При этом оно наследует такие свойства, как коммутативность и ассоциативность.

\begin{definition}
    Подмножество $\mathcal{L}$ поля $\mathbb{F}$ называется \textbf{подполем}, если
    \begin{enumerate}[nolistsep]
        \item $\mathcal{L}$ является подкольцом кольца $\mathbb{F}$.
        \item $a \in L, a \ne 0 \Rightarrow a^{-1} \in L$
        \item $1 \in L$
    \end{enumerate}
\end{definition}

Очевидно, что всякое подполе является полем относительно тех же операций.

\begin{statement}[Задача из листочка кружка в Хамовниках]
    Любое подполе поля $\R$ содержит $\Q$.
\end{statement}

\begin{proof}
    Сначала заметим, что в любом подполе $\R$ есть $0$ и $1$ (потому что это поле), а значит, там есть и все целые числа. Действительно,
    $$
    n \in \Z \Rightarrow n = \underbrace{1 + 1 + \ldots + 1}_{\text{$n$ раз}}.
    $$

    А значит, есть и все рациональные:
    $$
    q \in \Q \Rightarrow q = n m^{-1},\quad n \in \Z, m \in \Z \setminus \{0\}.
    $$
\end{proof}

\begin{statement}[Задача из Винберга]
    Поле $\Q$ не имеет нетривиальных отличных от него самого подполей.
\end{statement}

\begin{proof}
    Решение полностью аналогично решению предыдущей задачи --- в любом подполе $\Q$ лежат все целые числа, а значит, и все рациональные, а значит, оно совпадает с $\Q$.
\end{proof}

\begin{definition}
    Элемент $r \in \mathcal{K}$ ($\mathcal{K}$ содержит единицу $e$) называется \textbf{обратимым}, если существует $r^{-1} \in \mathcal{K}$ такой, что $rr^{-1} = r^{-1}r = e$.
\end{definition}

\begin{definition}
    Если $a$ и $b$ --- ненулевые элементы $\mathcal{K}$ и $ab = 0$, то $a$ называется \textbf{левым делителем нуля}, а $b$ --- \textbf{правым делителем нуля}.
\end{definition}

\begin{remark}
    Могут попросить привести пример коммутативного ассоциативного кольца с делителями нуля. Подойдёт кольцо функций на подмножестве $X$ числовой прямой, содержащем больше одной точки. В самом деле, разобьём $X$ на два непустых подмножества $X_1$ и $X_2$ и положим при $i = 1, 2$
    $$
    f_i(x) = 
    \begin{cases}
        1\quad&\text{при $x \in X_i$},\\
        0\quad&\text{при $x \notin X_i$}.
    \end{cases}
    $$
    Тогда $f_1, f_2 \ne 0$, но $f_1f_2 = 0$.
\end{remark}

\begin{lemma}
    В кольце делитель нуля не может быть обратим.
\end{lemma}

\begin{proof}
    Пусть $ab = 0$ и $a$ --- обратимый элемент. Тогда
    $$
    0 = a^{-1}0 = a^{-1}ab = b.
    $$
\end{proof}

\begin{remark}
    А в поле все обратимы, поэтому в поле нет делителей нуля.
\end{remark}

\begin{definition}
    Элемент $x \in \mathcal{K}$, $x \ne 0$ называется \textbf{нильпотентом}, если $\exists\!\:n \in \N$ такое, что $x^n = 0$.
\end{definition}

\begin{statement}[Задача Антона Александровича]
    Если матрица $A \in \underset{n \times n}{\mathrm{Mat}}$ нильпотента, то $A^n = 0$.
\end{statement}

\begin{proof}
    Пусть $\varphi: \R^n \rightarrow \R^n$ --- линейное преобразование, соответствующее матрице $A$. Тогда $\varphi(\R^n)$ --- подпространство в $\R^n$ в силу линейности $\varphi$. Заметим, что образ базиса $\{e_1, \ldots, e_n\}$ является полной системой в $\R^n$ опять же в силу линейности $\varphi$. При этом,
    $$
    (\varphi(e_1), \ldots, \varphi(e_n)) = (e_1, \ldots, e_n) \cdot A,
    $$
    т.\,к. $A$ --- матрица линейного преобразования $\varphi$. А из верхней оценки на ранг произведения следует, что $\rk(\varphi(e_1), \ldots, \varphi(e_n)) \leqslant \rk A$ (а если вспомнить нижнюю оценку, то легко понять, что на самом деле всегда достигается равенство, но нам для решения задачи это не нужно). Поэтому образ базиса --- полная линейно зависимая система. А значит, из неё можно выделить базис, в нём будет точно меньше $n$ векторов, а значит, $\dim\varphi(\R^n) < \dim \R^n = n$. То есть, каждый раз размерность уменьшается хотя бы на один. Отсюда,
    $$
    \varphi^n(\R^n) = \{\boldsymbol{0}\}.
    $$

    А значит, $\forall\!\:v \in \R^n$ выполнено $A^n\cdot v = 0$. Подставив вместо $v$ векторы из стандартного базиса в $\R^n$, легко убедиться, что все $a_{ij}$ нулевые, т.\,е. $A = 0$.
\end{proof}

\begin{lemma}
    Нильпотент является делителем нуля.
\end{lemma}

\begin{proof}
    Пусть $x$ нильпотент. Тогда возьмём $n$ наименьшим натуральным числом со свойством $x^n = 0$. Тогда
    $$
    x \cdot x^{n - 1} = 0, x^{n - 1} \ne 0, x \ne 0.
    $$
\end{proof}

\begin{theorem}
    Кольцо $\Z_n$ является полем тогда и только тогда, когда $n$ --- простое число.
\end{theorem}

\begin{theorem}
    $\Rightarrow$. Пусть $n$ составное, т.\,е. $n = k \cdot \ell$, где $1 < k, \ell < n$. Тогда $[k]_n, [\ell]_n \ne 0$, но
    $$
    [k]_n[\ell]_n = [k\ell]_n = [n]_n = 0.
    $$

    Таким образом, в кольце $\Z_n$ имеются делители нуля и, значит, оно не является полем.

    $\Leftarrow$. Уже доказывали в лемме 33.1.
\end{theorem}

\begin{remark}
    То, что написано в этом вопросе далее, не входит в программу экзамена и взято мной из лекций Е.\,Ю. Смирнова по алгебре в ВШЭ и А.\,Д. Елагина по алгебраической геометрии в НМУ.
\end{remark}

\begin{definition}[Прямое произведение колец]
    Пусть $A$ и $B$ --- кольца. На прямом произведении множеств $A \times B$ можно ввести операции сложения и умножения:
    $$
    (a_1, b_1) + (a_2, b_2) \vcentcolon = (a_1 + a_2, b_1 + b_2),\quad (a_1, b_1) \cdot (a_2, b_2) = (a_1a_2, b_1b_2).
    $$
    Очевидно, что множество $A \times B$ с введёнными таким образом операциями является кольцом.
\end{definition}

\begin{remark}
    Нулём такого кольца является пара $(0, 0)$, единицей --- пара $(1, 1)$. Прямое произвдеение колец всегда имеет делители нуля: $(a, 0) \cdot (0, b) = (0, 0)$.
\end{remark}

\begin{theorem}[Китайская теорема об остатках в слабой форме]
    Если числа $m_1, \ldots, m_k$ попарно взаимно просты, то отображение 
    $$
    [a] \in \Z_{m_1\ldots m_k} \mapsto ([a], \ldots, [a]) \in \Z_{m_1} \times \ldots \times \Z_{m_k}
    $$
    является изоморфизмом.
\end{theorem}

\begin{proof}
    Докажем индукцией по количеству взаимно простых сомножителей $k$.

    \textbf{База индукции} ($k = 2$). Рассмотрим прямое произведение колец $\Z_m \times \Z_n$ при взаимно простых $n$ и $m$. Единицей является пара $([1]_m, [1]_n)$. Обозначим её для удобства за $\boldsymbol{1}$, а нулём --- элемент $([0]_m, [0]_n)$, его обозначим за $\boldsymbol{0}$. Легко видеть, что в последовательности сумм
    $$
    \begin{array}{l}
        \boldsymbol{0}\\
        \boldsymbol{1}\\
        \boldsymbol{1} + \boldsymbol{1}\\
        \boldsymbol{1} + \boldsymbol{1} + \boldsymbol{1}\\
        \ldots\\
        \boldsymbol{1} + \boldsymbol{1} + \ldots + \boldsymbol{1}\\
        \ldots
    \end{array}\eqno(\ast)
    $$
    остатки в первой компоненте будут повторяться с периодом $m$, а во второй --- с периодом $n$, так что первое повторение получится на $mn$-ом шаге (в силу взаимной простоты $m$ и $n$). Следовательно, в $(\ast)$ мы перечислили ровно $mn$ различных элементов, т.\,е. все элементы нашего прямого произведения. Но сложение умножение элементов вида $(\ast)$ определено однозначно, и, конечно, это в точности соответствует определению операций сложения и умножения в $\Z_{mn}$. Это значит, что отображение $Z_{mn} \rightarrow \Z_m \times \Z_n$, сопоставляющее остатку $[a] \in \Z_{mn}$ сумму 
    $$
    \underbrace{([1]_m, [1]_n) + \ldots + ([1]_m, [1]_n)}_{\text{$a$ раз}},
    $$
    является изоморфизмом.

    \textbf{Шаг индукции}. Пусть утверждение верно для всех $k < K + 1$. Тогда отображение
    $$
    [a] \in \Z_{m_1\ldots m_K m_{K + 1}} \mapsto ([a], \ldots, [a], [a]) \in \Z_{m_1} \times \ldots \times \Z_{m_K} \times \Z_{m_{K + 1}}
    $$
    является изоморфизмом. Действительно, это утверждение для взаимно простых множителей $m = m_1\ldots m_K$ и $n = m_{K + 1}$ является базой индукции и было доказано.
\end{proof}


