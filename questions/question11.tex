\section{Фундаментальная система решений и алгоритм её поиска. Размерность пространства решений однородной СЛУ. Структура решений неоднородной СЛУ}

\begin{definition}
    \textbf{Фундаментальная система решений} --- это базиса подпространства решений однородной СЛУ.
\end{definition}

\begin{remark}
    Доказательство следующей теоремы даёт практический способ построения ФСР.
\end{remark}

\begin{theorem}
    Размерность пространства решений системы однородных линейных уравнений с $n$ неизвестными и матрицей коэффициентов $A$ равна $n - \rk A$.
\end{theorem}

\begin{proof}
    Рассмотрим систему уравнений
    $$
    \begin{cases}
        a_{11}x_1 + a_{12}x_2 + \ldots + a_{1n}x_n = 0,\\
        a_{21}x_1 + a_{22}x_2 + \ldots + a_{2n}x_n = 0,\\
        \ldots\\
        a_{m1}x_1 + a_{m2}x_2 + \ldots + a_{mn}x_n = 0,\\
    \end{cases}
    $$

    С помощью элементарных преобразований приведём её к ступечантому виду. Число ненулевых уравнений в этом ступенчатом виде равно $r = \rk A$. Поэтому общее решение будет содержать $r$ главных неизвестных и с точностью до перенумерации неизвестных будет иметь вид
    $$
    \begin{cases}
        x_1 = c_{11}x_{r + 1} + c_{12}x_{r + 2} + \ldots + c_{1, n - r}x_n,\\
        x_2 = c_{21}x_{r + 1} + c_{22}x_{r + 2} + \ldots + c_{2, n - r}x_n,\\
        \ldots\\
        x_r = c_{r1}x_{r + 1} + c_{r2}x_{r + 2} + \ldots + c_{r, n - r}x_n.\\
    \end{cases}
    $$

    Придавая поочерёно одному из свободных неизвестных $x_{r + 1}, x_{r + 2}, \ldots, x_n$ значение $1$, а остальным --- $0$, получим следующие решения системы:
    $$
    \begin{array}{l}
        u_1 = (c_{11}, c_{21}, \ldots, c_{r1}, 1, 0, \ldots, 0),\\
        u_2 = (c_{12}, c_{22}, \ldots, c_{r2}, 0, 1, \ldots, 0),\\
        \ldots\\
        u_{n - r} = (c_{1, n - r}, c_{2, n - r}, \ldots, c_{r, n - r}, 0, 0, \ldots, 0, 1).
    \end{array}
    $$

    Ранг системы векторов $\{u_1, u_2, \ldots, u_{n - r}\}$ равен рангу матрицы
    $$
    \left(
    \begin{array}{cccc | cccc}
        c_{11} & c_{21} & \ldots & c_{r1} & 1 & 0 & \ldots & 0\\
        c_{12} & c_{22} & \ldots & c_{r2} & 0 & 1 & \ldots & 0\\
        \vdots & \vdots & \ddots & \vdots & \vdots & \vdots & \ddots & \vdots\\
        c_{1, n - r} & c_{2, n - r} & \ldots & c_{r, n - r} & 0 & 0 & \ldots & 1
    \end{array}
    \right)
    $$

    Элементарные преобразования столбцов (как и элементарные преобразования строк) сохраняют ранг матрицы. Заметим, что поменяв местами <<блоки>>, отделённые друг от друга чертой, мы приведём матрицу к улучшенному ступенчатому виду. Как можно видеть, все её строки в улучшенном ступенчатом виде ненулевые, а потому, во-первых, ранг этой матриц равен количеству строк, т.\,е. $n - r$, а во-вторых, система $\{u_1, \ldots, u_{n - r}\}$ линейно независима. Также, эта система полна, т.\,к. любая линейная комбинация вида
    $$
    \lambda_1u_1 + \ldots + \lambda_{n - r}u_{n - r}
    $$
    является решением, в котором свободные неизвестные имеют значения $\lambda_1, \ldots, \lambda_{n - r}$.
\end{proof}

Структура решений неоднородной СЛУ --- это теорема 7.2.

