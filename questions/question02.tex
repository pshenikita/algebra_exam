\section{Экзотические уравнения. Свободные и главные переменные. Ступенчатый и улучшенный ступенчатый вид матрицы. Метод Гаусса решения СЛУ}

\begin{definition}
    \textbf{Лидером} (\textbf{ведущим элементом}) ненулевой строки $(a_1, a_2, \ldots, a_n) \in \mathcal{K}^n$ её первый ненулевой элемент.
\end{definition}

\begin{definition}
    Матрица называется \textbf{ступенчатой}, если номера ведущих элементов её ненулевых строк образуют строго возрастающую последовательность, а нулевые строки расположены в конце:
    $$
    \left(
    \begin{array}{ccccccc}
        {} & \bord & \ast  & \ast     & \ast  & \ast  & \ast \\\cline{3-3}
        {} & {}    & \bord & \ast     & \ast  & \ast  & \ast \\\cline{4-5}
        {} & {}    & {}    & {}       & \bord & \ast  & \ast \\\cline{6-6}
        {} & {}    & {}    & \bigzero & {}    & \bord & \ast \\\cline{7-7}
        {} & {}    & {}    & {}       & {}    & {}    & {}   
    \end{array}
    \right)
    $$

    Матрица называется улучшенной ступенчатой, если она ступенчатая, её ведущие элементы равны $1$, а элементы над ними равны 0:
    $$
    \left(
    \begin{array}{ccccccc}
        {} & \bord & 1     & 0        & \ast  & 0     & 0    \\\cline{3-3}
        {} & {}    & \bord & 1        & \ast  & 0     & 0    \\\cline{4-5}
        {} & {}    & {}    & {}       & \bord & 1     & 0    \\\cline{6-6}
        {} & {}    & {}    & \bigzero & {}    & \bord & 1    \\\cline{7-7}
        {} & {}    & {}    & {}       & {}    & {}    & {}   
    \end{array}
    \right)
    $$
\end{definition}

\begin{theorem}
    Всякую матрицу путём элементарных преобразований строк можно привести к ступенчатому виду.
\end{theorem}

\begin{proof}
    Если данная матрица нулевая, то она уже ступенчатая. Если она ненулевая, то пусть $j_1$ --- номер её первого ненулевого столбца. Переставив, если нужно, строки, добьёмся того, чтобы $a_{1j_1} \ne 0$. После этого прибавим к каждой строке, начиная со второй, первую строку, умноженную на подходящее число, с таким расчётом, чтобы все элементы $j_1$-го столбца, кроме первого, стали равными нулю. Теперь рассмотрим матрицу без первой строки и первый $j_1$ столбцов. Приведём её таким же методом к ступенчатому виду. Продолжая процесс таким же образом, мы получим ступенчатую матрицу.
\end{proof}

Процесс, проводимый нами в доказательстве последней теоремы, называется \textit{прямым ходом метода Гаусса}.

\begin{remark}
    Для приведения матрицы к ступенчатому виду достаточно преобразований первого типа. Действительно, преобразования второго типа нужны были нам лишь для того, чтобы поднять на первую строчку ненулевой элемент. Однако, эту задачу можно выполнить преобразованием первого типа. Пусть $a_{i_1j_1} \ne 0$. Тогда прибавим $i_1$ строку расширенной матрицы к первой. Теперь получаем $a_{0j_1} = a_{i_1j_1} \ne 0$. А далее действуем так же.
\end{remark}

Теперь приведём расширенную матрицу коэффициентов СЛУ к улучшенному ступенчатому виду. По теореме 1.1 множество её решений не изменилось, а решать её в таком виде куда проще.

\begin{definition}
    Пусть $j_1, j_2, \ldots, j_r$ --- номера ведущих коэффициентов ненулевых уравнений системы. Неизвестные $x_{j_1}, x_{j_2}, \ldots, x_{j_r}$ назовём \textbf{главными}, а остальные --- \textbf{свободными}.
\end{definition}

\begin{definition}
    Уравнения вида
    $$
    0\cdot x_1 + 0\cdot x_2 + \ldots + 0\cdot x_n = b,
    $$
    где $b \ne 0$ назовём \textbf{экзотическими}.
\end{definition}

\begin{theorem}
    Для каждого набора значений свободных переменных существует единственный набор значений главных переменных, который дополняет данный набор до решения совместной системы.
\end{theorem}

\begin{proof}
    Идём по строкам ступечатой расширенной матрицы коэффициентов СЛУ снизу вверх. Рассмотрим первую встреченную нами ненулевую строку, её номер $i_r$. В соответствующем уравнении ровно одна главная переменная --- $x_{j_r}$. Её можно выразить через свободные переменные и свободный член уравнения единственным образом. Теперь возьмём следующую строку. Её номер $i_{r - 1}$, и в ней ровно две главные неизвестные --- $x_{j_r}$ и $x_{j_{r - 1}}$. Выразим $x_{j_{r - 1}}$ через $x_{j_r}$, свободные переменные и свободный член. А теперь подставим выражения $x_{j_r}$ через свободные переменные, полученное на предыдущем шаге. Получается, $x_{j_{r - 1}}$ тоже выражается через свободный член и свободные переменные единственным образом. Продолжив процесс, получим выражения для всех главных переменных через свободные.
\end{proof}

Процесс, проводимый нами в доказательстве последней теоремы, называется \textit{обратным ходом метода Гаусса}.

\begin{remark}
    Как следствие, любая СЛУ над бесконечным полем имеет либо ни одного, либо одно, либо бесконечно много решений. А над конечным --- либо ни одного, либо одно, либо $|\mathcal{K}|^{n - r}$, где $r$ --- количество ступенек в ступенчатой матрице (иными словами, $n - r$ --- количество свободных переменных). Доказательство того, что это число $r$ определенно корректно, будет позднее.
\end{remark}


