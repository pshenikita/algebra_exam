\section{Многочлены от нескольких переменных. Порядки на мономах. Лексикографический порядок и его свойства. Старший член и моном. Лемма о старшем члене}

\begin{definition}
    \textbf{Кольцо многочленов от нескольких переменных} $x_1, \ldots, x_n$ с коэффициентами из области целостности $\mathcal{R}$ определим рекурсивно:
    $$
    \mathcal{R}[x1, \ldots, x_n] = \mathcal{R}[x_1, \ldots, x_{n - 1}][x_n].
    $$
\end{definition}

Исходя из теоремы 39.2, если $\mathcal{R}$ --- область целостности, то и $\mathcal{R}[x_1, \ldots, x_n]$ --- тоже область целостности.

\begin{definition}
    Определим \textbf{лексикографический порядок на мономах}. Пусть
    $$
    m_\alpha = x_1^{\alpha_1}\ldots x_n^{\alpha_n},\quad m_\beta = x_1^{\beta_1}\ldots x_n^{\beta_n}
    $$
    и $\alpha_1 = \beta_1, \ldots, \alpha_{k - 1} = \beta_{k - 1}б \alpha_k \ne \beta_k$. Тогда если $\alpha_k > \beta_k$, то $m_\alpha \succ m_\beta$, иначе $m_\alpha \prec m_\beta$.
\end{definition}

\begin{theorem}
    Отношение лексикографического порядка обладает следующими свойствами:
    \begin{enumerate}[nolistsep]
        \item для любых двух несовпадающих мономов либо $u \succ v$, либо $u \prec v$;
        \item не может быть $u \succ v$ и $u \prec v$;
        \item из того, что $u \succ v$ и $v \succ w$ следует $u \succ w$;
        \item из того, что $u \succ v$ следует $uw \succ vw$ для любого монома $w$;
        \item не существует бесконечных убывающих цепочек мономов $u_1 \succ u_2 \succ u_3 \succ \ldots$
    \end{enumerate}
\end{theorem}

\begin{proof}
    Свойства 1 и 2 очевидно следуют из определения. Докажем 3. Пусть первая переменная, которая не входит во все одночлены $u$, $v$, $w$ с одним и тем же показателем, входит в них с показателями $k$, $\ell$, $m$ соответственно. Тогда
    $$
    k \geqslant \ell \geqslant m,
    $$
    причём хотя бы в одном из двух случаев имеет место строго неравенство. Следовательно, $k > m$ и $u \succ w$.

    Докажем 4. При умножении на $w$ к показателям, с которыми каждая из переменных входит в $u$ и $v$, добавляется одно и то же число, и знак неравенства между этими показателями не меняется, а только эти неравенства и имеют значение при сравнении одночленов.

    Докажем 5 индукцией по $n$.

    \textbf{База индукции} ($n = 1$). Тогда $m_1 = x^k$ для некоторого $k$ и в убывающей последовательности $m_1 \succ m_2 \succ m_3 \succ \ldots$ встретятся только попарно различные мономы $m_t = x^t$, причём $t \leqslant k$, поэтому такая последовательность не может быть бесконечной.

    \textbf{Шаг индукции}. Пусть свойство 5 доказано для всех $n < m$. Докажем для $n = m$. Допустим, что существует бесконечная убывающая последовательность $m_1 \succ m_2 \succ \ldots$. При этом $m_1 = x_1^{\alpha_1}\ldots x_m^{\alpha_m}$. При переходе от $m_i$ к $m_{i + 1}$ показатель степени $x_1$ либо не меняется, либо убывает. Следовательно, убывать он может лишь конечное число раз. Значит, найдётся такое натуральное $N$, что начиная с $m_N$ показатель степени $x_1$ не изменяется. Рассмотрим последовательность без $x_1$, начиная с $N$-го члена. Её длина $m - 1 < m$ и она бесконечно убывает. Противоречие.
\end{proof}

\begin{definition}
    Среди ненулевых членов любого ненулевого многочлена $f \in \mathcal{R}[x_1, \ldots, x_n]$ найдётся единственный, который лексикографически старше всех остальных. Он называется \textbf{старшим членом} многочлена $f$. \textbf{Старшим мономом} для удобства будем называть старший член с коэффициентом.
\end{definition}

\begin{lemma}[О старшем члене]
    Старший член произведения ненулевых многочленов равен произведению их старших членов.
\end{lemma}

\begin{proof}
    Достаточно доказать это утверждение для двух многочленов (дальше индукцией). Пусть $f_1$, $f_2$ --- ненулевые многочлены, $u_1$, $u_2$ --- их старшие члены и $v_1$, $v_2$ --- какие-то их члены. Если $v_1 \ne u_1$ и $v_2 \ne u_2$, то в силу теоремы 47.1
    $$
    u_1u_2 \succ v_1v_2.
    $$
\end{proof}


