\section{Формулы Крамера}

\begin{theorem}[Крамер]
    Пусть дана система уравнений
    $$
    \begin{cases}
        a_{11}x_1 + \ldots + a_{1n}x_n = b_1,\\
        \vdots\\
        a_{n1}x_1 + \ldots + a_{nn}x_n = b_n
    \end{cases}
    $$
    и $A = (a_{ij})$ --- матрица её коэффициентов. Тогда эта система определена тогда и только тогда, когда \mbox{$\det A \ne 0$}. В этом случае решение находится по формулам
    $$
    x_i = \frac{\det A_i}{\det A},
    $$
    где $A_i$ --- матрица, полученная из $A$ заменой её $i$-го столбца столбцом свободных членов.
\end{theorem}

\begin{proof}
    Докажем первую часть теоремы. Условие $\det A \ne 0$ равносильно $\rk A = n$, что, в свою очередь, равносильно определённости системы (теорема 10.3).

    Теперь перейдём ко второй части. При любом элементарном преобразовании системы в матрицах $A$ и $A_i$ одновременно происходит соответствующее элементарное преобразование строк и, следовательно, отношения, стоящие в правых частях формул Крамера, не изменяются. С помощью элементарных преобразований строк матрицу $A$ можно привести к единичной матрице. Поэтому достаточно доказать теорему в том случае, когда $A = E$. Тогда система имеет вид
    $$
    \left\{
        \begin{array}{ccccc}
            x_1 & {} & {} & {} & =b_1,\\
            {} & x_2 & {} & {} & =b_2,\\
            {} & {} & \ddots & {} & \vdots\\
            {} & {} & {} & x_n & =b_n.\\
        \end{array}
    \right.
    $$
    
    Она, очевидно, имеет единственное решение $x_i = b_i$ ($i = 1, 2, \ldots, n$). С другой стороны,
    $$
    \det A = \det E = 1,\quad \det A_i =
    \det
    \begin{pmatrix}
        1 & 0 & \cdots & b_1 & \cdots & 0 & 0\\
        0 & 1 & \cdots & b_2 & \cdots & 0 & 0\\
        \vdots & \vdots & \ddots & \vdots & \ddots & \vdots & \vdots\\
        0 & 0 & \cdots & b_i & \cdots & 0 & 0\\
        \vdots & \vdots & \ddots & \vdots & \ddots & \vdots & \vdots\\
        0 & 0 & \cdots & b_{n - 1} & \cdots & 1 & 0\\
        0 & 0 & \cdots & b_n & \cdots & 0 & 1\\
    \end{pmatrix} = b_i,
    $$
    так что формулы Крамера в этом случае действительно верны.
\end{proof}


