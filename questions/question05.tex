\section{Понятие линейной зависимости системы векторов. Три свойства линейной зависимости}

\begin{definition}
    Пусть $v_1, \ldots, v_k$ --- векторы из векторного пространства $V$, а $\lambda_1, \ldots, \lambda_k$ --- элементы поля $\mathcal{K}$. Тогда \textbf{линейной комбинацией} векторов $v_1, \ldots, v_k$ с коэффициентами $\lambda_1, \ldots, \lambda_k$ --- это выражение
    $$
    \lambda_1v_1 + \ldots + \lambda_kv_k.
    $$

    При $\lambda_1 = \ldots = \lambda_k = 0$ линейная комбинация называется \textbf{тривиальной}.
\end{definition}

\begin{definition}
    Конечная система векторов называется \textbf{линейно зависимой}, если существует нетривиальная нулевая линейная комбинация. Бесконечная система векторов называется \textbf{линейно зависимой}, если в ней можно выделить конечную линейно зависимую подсистему. Система векторов называется \textbf{линейно независимой}, если она не линейно зависима.
\end{definition}

\begin{theorem}[Три свойства линейной зависимости]
    \begin{enumerate}[nolistsep]
        \item Пусть $\mathcal{S}$ и $\mathcal{S}^\prime$ --- две системы векторов, причём $\mathcal{S} \subseteq \mathcal{S}^\prime$. Тогда если $\mathcal{S}$ линейно зависима, то и $\mathcal{S}^\prime$ линейно зависима.
        \item Система $V$ линейно зависима тогда и только тогда, когда в ней есть вектор $v$, который линейно выражается через остальные.
        \item Пусть система $V$ линейно независима, а система $V \cup \{u\}$ линейно зависима. Тогда существует единственный набор элементов поля $\mathcal{K}$: $\mu_1, \ldots, \mu_k$ такой, что $u = \mu_1v_1 + \ldots + \mu_kv_k$, где $\{v_1, \ldots, v_k\} \subseteq V$ (равенство достигается для конечной системы).
    \end{enumerate}
\end{theorem}

\begin{remark}
    Свойство 2, как видно, из формулировки, является критерием линейной зависимости.
\end{remark}

\begin{proof}
    Докажем сначала для конечных систем:
    \begin{enumerate}[nolistsep]
        \item $\mathcal{S} = \{v_1, \ldots, v_k\}$ линейно зависима, поэтому в ней есть вектор $u$, выражающийся линейно через некоторые векторы $v_1, \ldots, v_k$. Однако заметим, что $\{u, v_1, \ldots, v_k\} \subseteq \mathcal{S} \subseteq \mathcal{S}^\prime$. Поэтому $u$ выражается через векторы системы $\mathcal{S}^\prime$: коэффициенты при $v_1, \ldots, v_k$ не меняем, а коэффициенты при остальных векторах выбираем нулевыми.
        \item $V = \{v_1, \ldots, v_k\}$ линейно зависима, значит, существует нетривиальная линейная комбинация, такая что
            $$
            \lambda_1v_1 + \ldots + \lambda_kv_k = 0,\quad (\lambda_1, \ldots, \lambda_k) \ne (0, \ldots, 0).
            $$

            Существует $j$ такой, что $\lambda_j \ne 0$, поэтому можно выразить $v_j$:
            $$
            v_j = \sum_{i \ne j}\left(-\frac{\lambda_i}{\lambda_j}\right)v_i.
            $$

            Обратно, пусть существует вектор $\displaystyle v_j = \sum_{i \ne j}\lambda_iv_i$. Тогда
            $$
            \lambda_1v_1 + \ldots + (-1)v_j + \ldots + \lambda_kv_k = 0.
            $$

            Эта линейная комбинация нетривиальная, т.\,к. $-1 \ne 0$.
        \item Существует нетривиальная линейная комбинация
            $$
            \lambda_1v_1 + \ldots + \lambda_kv_k + \mu u = 0\eqno(\ast)
            $$

            При этом $\mu \ne 0$, т.\,к. иначе останется линейная комбинация $\lambda_1v_1 + \ldots + \lambda_kv_k = 0$, а она обязательно тривиальная (т.\,к. система $\{v_1, \ldots, v_k\}$ линейно независима), а поэтому и комбинация $(\ast)$ тривиальная. Итак, $\mu \ne 0$, поэтому можно так же, как в пункте 2, выразить его через остальные:
            $$
            u = \sum_i\left(-\frac{\lambda_i}{\mu}\right)v_i.
            $$
    \end{enumerate}

    А теперь --- для бесконечных:
    \begin{enumerate}[nolistsep]
        \item $\mathcal{S}$ линейно зависима, а значит, в ней можно выделить конечную линейно зависимую подсистему. А из того, что $\mathcal{S}^\prime \supseteq \mathcal{S}$, в $\mathcal{S}^\prime$ можно выделить эту же линейно зависимую подсистему
        \item Выделим в $V$ конечную линейно зависимую систему и применим утверждение теоремы для неё.
        \item Выделим в $V \cup \{u\}$ линейно зависимую подсистему и применим утверждение теоремы для неё.
    \end{enumerate}
\end{proof}

