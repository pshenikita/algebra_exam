\section{Примитивные многочлены над факториальным кольцом. Любой многочлен пропорционален примитивному. Лемма Гаусса}

\begin{definition}
    Многочлен $f(x) = a_nx^n + \ldots + a_1x + a_0$ ($f \in A[x]$) называется примитивным, если
    $$
    \gcd(a_1, a_2, \ldots, a_n) = 1.
    $$
\end{definition}

Напомним, что для целостного кольца $A$ можно определить поле частных $\mathrm{Quot}\,A$. При этом существует вложение $A$ в $\mathrm{Quot}\,A$, $a \mapsto a / 1$. Тогда кольцо многочленов $A[x]$ вкладывается в кольцо многочленов $(\mathrm{Quot}\,A)[x]$.

\begin{lemma}
    Любой многочлен $f \in (\mathrm{Quot}\,A)[x]$ можно представить в виде $r \cdot g$, где $r \in \mathrm{Quot}\,A$, $g \in A[x]$ --- примитивный многочлен.
\end{lemma}

\begin{proof}
    Представим каждый ненулевой коэффициент $r_i$ в виде несократимой дроби $r_i = \frac{a_i}{b_i}$. Тогда $a = \gcd(a_i)$, $b = \lcm(b_i)$. Положим $r = \frac{a}{b}$. Тогда $s_i = \frac{r_i}{r} = \frac{a_ib}{ab_i} \in A$, т.\,к. $a \mid a_i$ и $b_i \mid b$. С другой стороны, допустим, что простой множитель $p$ делит все $s_i$. Теперь рассмотрим случаи:

    \textbf{Случай 1}: $p \mid b$. Тогда существует $i$ такое, что $p$ входит в $b_i$ в той же степени, что и в $b$. Т.\,к. $p \mid b_i$, получаем $p \nmid a_i$ и, следовательно, $p \nmid a$. В итоге $p$ не делит $s_i = \frac{a_ib}{ab_i}$. Противоречие.

    \textbf{Случай 2}: $p \nmid b$. Т.\,к. для каждого $i$ выполнено $p \mid s_i$, то $p \mid \gcd(a_i, b)$. Поскольку $p \nmid b$, получаем, чо для каждого $i$ верно $p \mid a_i$. Но существует такое $i$, такое что степень вхождения $p$ в $a_i$ такая же, как и в $a$. При этом $i$ степень вхождения $p$ в $s_i = \frac{a_ib}{ab_i}$ не может быть положительной. Противоречие.
\end{proof}

\begin{lemma}[Гаусс]
    Произведение двух примитивных многочленов есть примитивный многочлен.
\end{lemma}

\begin{proof}
    Пусть $f = a_0 + a_1x + \ldots + a_nx^n$, $g = b_0 + b_1x + \ldots + b_mx^m$ --- примитивные многочлены из $A[x]$. Пусть $fg = c_0 + c_1x + \ldots + c_{m + n}x^{m + n}$. При этом
    $$
    c_k = \sum_{j = 0}^ka_jb_{k - j}.
    $$
    Допустим, что $p \mid c_i$ для всех $i$. Пусть $j$ --- минимальное число, такое что $p \nmid a_j$, а $\ell$ --- минимальное число, такое что $p \nmid b_\ell$. Тогда
    $$
    c_{j + \ell} = a_0b_{j + \ell} + \ldots + a_{j - 1}b_{\ell + 1} + a_jb_\ell + a_{j + 1}b_{\ell - 1} + \ldots + a_{j + \ell}b_0.
    $$
    Все слагаемые делятся на $p$, кроме $a_jb_\ell$. Значит, и вся сумма не делится. Противоречие.
\end{proof}

\begin{remark}
    Если $f, g \in A[x]$ примитивны и $f = rg$, где $r \in \mathrm{Quot}\,A$, то $r$ обратим. В самом деле, если неприводимый элемент $p$ входит в числитель несократимого вида $r$, то все коэффициенты $f$ делятся на $p$. А если в знаменатель, то все коэффициенты $g$ делятся на $p$.
\end{remark}


