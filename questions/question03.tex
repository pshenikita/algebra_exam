\section{Единственность улучшенного ступенчатого вида матрицы. Понятие ранга матрицы (через ступенчатый вид) и его корректность}

\begin{theorem}
    Каждая матрица имеет единственный улучшенный ступенчатый вид.
\end{theorem}

\begin{proof}
    Рассмотрим матрицу $A$. Допустим, она может быть приведена элементарными преобразованиями к двум улучшенным ступенчатым видам $B$ и $C$. Наша цель --- доказать, что $B = C$. Рассмотрим однородную систему уравнений с матрицей коэффициентов $A$. Расширенная матрица коэффициентов $\widetilde{A} = (A \mid 0)$ приводится теми же элементарными преобразованиями к виду ступенчатым видам $\widetilde{B} = (B \mid 0)$ и $\widetilde{C} = (C \mid 0)$. Это означает, что однородные системы с матрицами коэффициентов $B$ и $C$ эквивалентны. Пусть $B \ne C$. Отбросим нулевые строки в матрицах $\widetilde{B}$ и $\widetilde{C}$, при этом системы останутся эквивалентными. Будем идти по строкам матриц $\widetilde{B}$ и $\widetilde{C}$, пока не дойдём до места, где будет различие. Попадаем в один из трёх случаев:
    \begin{enumerate}
        \item \textit{Строки с первой по $(k - 1)$-ую снизу в матрицах $\widetilde{B}$ и $\widetilde{C}$ совпадают, а лидеры $k$-ой строки снизу в матрицах $\widetilde{B}$ и $\widetilde{C}$ имеют различные позиции}. Пусть лидеры стоят в столбцах $p$ и $q$ соответственно. Не ограничивая общности, можем считать, что $p < q$. Из того, что строки ниже, чем $k$-ые снизу у матриц $\widetilde{B}$ и $\widetilde{C}$ совпадают, следует, что разделение переменный $x_i$ при $i > q$ на главные и свободные в системах $\widetilde{B}$ и $\widetilde{C}$ одинаково. Положим все свободные переменные с номерами $> q$ равными нулю. В системе $(C \mid 0)$ переменная $x_q$ главная и, следвоательно (т.\,к. система однородная) при указанному задании переменных она также обязательно равна нулю. В системе $(B \mid 0)$ переменная $x_q$ свободная, а потому при указанному задании переменных она может принимать значение $1$. Таким образом, нашли решение одной системы, которое не является решением другой. Противоречие с эквивалентностью систем $(B \mid 0)$ и $(C \mid 0)$.
        \item \textit{Строки с первой по $(k - 1)$-ую снизу в матрицах $\widetilde{B}$ и $\widetilde{C}$ совпадают, лидеры $k$-ой строки снизу у этих матриц имеют одинаковые позиции $p$, но есть номер $s > p$ такой, что в $s$-ом столбце в рассматриваемой строке у матриц $\widetilde{B}$ и $\widetilde{C}$ стоят различные числа}. Пусть эти числа $b$ и $c$ соответственно. Не ограничивая общности, $b \ne 0$. Тогда $x_s$ --- свободная переменая для системы $(B \mid 0)$, т.\,к. в столбцах, соответствующих главным переменным стоят нули за счёт улучшенного ступенчатого вида. А значит, $x_s$ --- свободная переменнная и для $(C \mid 0)$. Положим все свободные переменные, кроме $x_s$, равными нулю, а $x_s = 1$. Тогда из системы $(B \mid 0)$ получаем, что $x_s = -b$, а из системы $(C \mid 0)$ --- что $x_s = -c$, но $b \ne c$. Противоречие.
        \item \textit{Проходя снизу вверхн по ненулевым строкам матриц $\widetilde{B}$ и $\widetilde{C}$, мы всё время видели, что очередные строки совпадают, но строки в одной из матриц закончились, а в другой --- нет}. Пусть строки закончились в матрице $\widetilde{B}$, тогда в матрице $\widetilde{C}$ есть ещё одна нулевая строка. Из того, что строки ниже, чем данная, у матриц $\widetilde{B}$ и $\widetilde{C}$ совпадают, следует, что разделение переменных $x_i$ при $i > q$ на главные и свободные в соответствующих системах одинаково. Но в системе $(B \mid 0)$ переменная $x_q$ свободная (нет строки, где лидер имеет позицию $q$), а в системе $(C \mid 0)$ --- свободная. Далее можем поступить так же, как и в первом случае.
    \end{enumerate}
\end{proof}

\begin{remark}
    Отсюда следует, что количество ненулевых строк в любом ступенчатом виде данной матрицы одинаково. Действительно, ведь для любого ступенчатого вида можно построить улучшенный ступенчатый с таким же количеством ненулевых строк --- достаточно каждую ненулевую строку разделить на её ведущий элемент, а затем вычесть её из всех строк выше с подходящим коэффициентом. А ступенчатый вид единственный.
\end{remark}

\begin{definition}[Ступенчатый ранг матрицы]
    Количество ненулевых строк в ступенчатом виде данной матрицы $A$ называется \textbf{рангом матрицы $A$} и обозначается $\rk A$.
\end{definition}

