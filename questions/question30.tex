\section{Определитель Вандермонда. Задача интерполяции}

\begin{theorem}[Определитель Вандермонда]
    $$
    \det
    \begin{pmatrix}
        1 & x_1 & \cdots & x_1^{n - 1}\\
        1 & x_2 & \cdots & x_2^{n - 1}\\
        \vdots & \vdots & \ddots & \vdots\\
        1 & x_n & \cdots & x_n^{n - 1}
    \end{pmatrix} =
    \prod_{1 \leqslant j < i \leqslant n}(x_i - x_j).
    $$
\end{theorem}

\begin{proof}
    Обозначим данный определитель через $V(x_2, \ldots, x_n)$ и докажем утверждение индукцией по $n$:

    \textbf{База индукции} ($n = 2$). 
    $$
    V(x_1, x_2) = 
    \det
    \begin{pmatrix}
        1 & x_1\\
        1 & x_2
    \end{pmatrix} = x_2 - x_1.
    $$

    \textbf{Шаг индукции}. Вычтем каждый столбец, умноженый на $x_1$ из следующего, при этом двигаясь справа налево по столбцам матрицы. Это элементарное преобразование, а потому определитель не изменится.
    $$
    V(x_1, \ldots, x_n) = 
    \det
    \begin{pmatrix}
        1 & x_1 & \cdots & x_1^{n - 1}\\
        1 & x_2 & \cdots & x_2^{n - 1}\\
        \vdots & \vdots & \ddots & \vdots\\
        1 & x_n & \cdots & x_n^{n - 1}
    \end{pmatrix} = 
    \det
    \begin{pmatrix}
        1 & 0 & 0 & \cdots & 0\\
        1 & x_2 - x_1 & x_2(x_2 - x_1) & \cdots & x_2^{n - 2}(x_2 - x_1)\\
        \vdots & \vdots & \ddots & \vdots & \vdots\\
        1 & x_n - x_1 & x_n(x_n - x_1) & \cdots & x_n^{n - 2}(x_n - x_1)\\
    \end{pmatrix}.
    $$

    По теореме об определителе матрицы с углом нулей, последний определитель равен
    $$
    \det
    \begin{pmatrix}
        x_2 - x_1 & x_2(x_2 - x_1) & \cdots & x_2^{n - 2}(x_2 - x_1)\\
        \vdots & \ddots & \vdots & \vdots\\
        x_n - x_1 & x_n(x_n - x_1) & \cdots & x_n^{n - 2}(x_n - x_1)\\
    \end{pmatrix} = (x_2 - x_1) \ldots (x_n - x_1)V(x_2, \ldots, x_n).
    $$

    А по предположению индукции $\displaystyle V(x_2, \ldots, x_n) = \prod_{2 \leqslant j < i \leqslant n}(x_i - x_j)$, отсюда следует требуемое.
\end{proof}

\textbf{Задача интерполяции} заключается в том, чтобы найти функцию $f(x)$ (обычно из некоторого заданного класса) такую, что в $n$ попарно различных точках $x_1, \ldots, x_n$ она принимает заданные значения $y_1, \ldots, y_n$.

\begin{theorem}
    Для любых различных $x_1, \ldots, x_n$ и любых заданных $y_1, \ldots, y_n$ существует единственный многочлен степени не более $n - 1$, такой что $f(x_i) = y_i$ ($i = 1, \ldots, n$).
\end{theorem}

\begin{proof}
    Будем искать этот многочлен методом неопределённых коэффициентов. У многочлена не более чем $(n - 1)$-ой степени их $n$ (некоторые могут получиться нулями). Тогда условие $f(x_i) - y_i$ даёт линейное условие на эти коэффициенты, а совокупность таких условий является квадратной СЛУ. Матрица её коэффициентов выглядит так:
    
    \noindent
    \begin{minipage}{.3\textwidth}
    $$
    \begin{pmatrix}
        1 & x_1 & \cdots & x_1^{n - 1}\\
        1 & x_2 & \cdots & x_2^{n - 1}\\
        \vdots & \vdots & \ddots & \vdots\\
        1 & x_n & \cdots & x_n^{n - 1}
    \end{pmatrix}
    $$
    \end{minipage}
    \begin{minipage}[b]{.7\textwidth}\setlength{\parindent}{17pt}
        Её определитель равен $V(x_1, \ldots, x_n)$, а т.\,к. все $x_i$ попарно различны, то он ненулевой, и система определена. Так, коэффициенты нашего многочлена существуют и единственны.
    \end{minipage}

    Этот многочлен можно также выписать явно:
    $$
    f(x) = \sum_{i = 1}^n\left(\frac{\prod\limits_{j \ne i}(x - x_j)}{\prod\limits_{j \ne i}(x_i - x_j)}y_i\right).
    $$

    Это действительно многочлен, т.\,к. знаменатели --- константы, не равные нулю. Числители --- многочлены степени $n - 1$, поэтому их сумма --- многочлен степени не более $n - 1$. Если мы подставим в этот многочлен $x_k$, то все слагаемые, кроме $k$-го обратятся в ноль. А $k$-ое слагаемое становится равным $y_k$. Этот многочлен называется \textbf{интерполяционным многочленом Лагранжа}.
\end{proof}

