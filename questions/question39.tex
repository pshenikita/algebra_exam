\section{Целостное кольцо. Многочлены от одной переменной над целостным кольцом. Понятие степени многочлена и её свойства. Целостность кольца многочленов над целостным кольцом. Обратимые элементы в кольце многочленов над целостным кольцом. Разложение многочлена по степеням $x - x_0$. Теорема Безу}

\begin{definition}
    Коммутативное ассоциативное кольцо с единицей без делителей нуля называется \textbf{областью целостности} (\textbf{целостным кольцом}).
\end{definition}

\begin{definition}
    Пусть $\mathcal{R}$ --- коммутативное ассциативное кольцо с единицей. \textbf{Многочлен} над $\mathcal{R}$ -- это финитная (с конечным числом ненулевых элементов) последовательность $(a_0, a_1, a_2, \ldots)$, где $a_i \in \mathcal{R}$.
\end{definition}

\begin{definition}[Операции на многочленах]
    $$
    (a_0, a_1, a_2, \ldots) + (b_0, b_1, b_2, \ldots) \vcentcolon = (a_0 + b_0, a_1 + b_1, a_2 + b_2, \ldots)
    $$
    $$
    (a_0, a_1, a_2, \ldots) \cdot (b_0, b_1, b_2, \ldots) \vcentcolon = (c_0, c_1, c_2, \ldots),\quad\text{где }c_k = \sum_{j = 0}^ka_{j}b_{k - j}.
    $$
\end{definition}

\begin{definition}
    Будем обозначать множество многочленов над областью целостности $\mathcal{R}$ как $\mathcal{R}[x]$.
\end{definition}

\begin{remark}
    Поле является областью целостности.
\end{remark}

\begin{definition}
    \textbf{Алгеброй} над полем $\mathcal{F}$ называется множество $A$ с операциями сложения, умножения и умножения на элементы поля $\mathcal{F}$, для которого выполняются следующие аксиомы:
    \begin{enumerate}
        \item относительно сложения и умножения на элементы поля $A$ есть векторное пространство;
        \item относительно сложения и умножения $A$ есть кольцо;
        \item $(\lambda a)b = a(\lambda b) = \lambda(ab)$ для любых $\lambda \in \mathcal{F}$, $a, b \in A$.
    \end{enumerate}
\end{definition}

\begin{lemma}
    $(\mathcal{R}[x], +, \boldsymbol{\cdot})$ --- коммутативное ассоциативное кольцо с единицей.
\end{lemma}

\begin{proof}
    Все аксиомы очевидны, кроме ассоциативности умножения. Пусть
    $$
    ((a_0, a_1, a_2, \ldots) \cdot (b_0, b_1, b_2, \ldots)) \cdot (c_0, c_1, c_2, \ldots) = (d_0, d_1, d_2, \ldots)
    $$
    $$
    (a_0, a_1, a_2, \ldots) \cdot ((b_0, b_1, b_2, \ldots) \cdot (c_0, c_1, c_2, \ldots)) = (f_0, f_1, f_2, \ldots)
    $$

    Обозначим $(a_0, a_1, a_2, \ldots) \cdot (b_0, b_1, b_2, \ldots) = (u_0, u_1, u_2, \ldots)$, $(b_0, b_1, b_2, \ldots) \cdot (c_0, c_1, c_2, \ldots) = (v_0, v_1, v_2, \ldots)$.
    Имеем
    $$
    d_k = \sum_{j = 0}^ku_jc_{k - j} = \sum_{j = 0}^k\sum_{i = 0}^ja_ib_{j - i}c_{k - j} = \sum_{p + q + r = k}a_pb_qc_r,
    $$
    $$
    f_k = \sum_{k = 0}^ka_iv_{k - i} = \sum_{i = 0}^k\sum_{s = 0}^{k - i}a_ib_sc_{k - i - s} = \sum_{p + q + r = k}a_pb_qc_r.
    $$
\end{proof}

Заметим, что единицей кольца является элемент $(1, 0, 0, \ldots)$. При этом элементы $(r, 0, 0, \ldots)$ складываются и умножаются так же, как и элементы области целостности $\mathcal{R}$. Таким образом, отождествим $(r, 0, 0, \ldots) \mapsto r$ и получим вложение колец $\mathcal{R} \subset \mathcal{R}[x]$ (инъективный гомоморфизм). Обозначим $(0, 1, 0, \ldots) \in \mathcal{R}[x]$ через $x$.

\begin{remark}
    Кольцо многочленов над полем является алгеброй.
\end{remark}

\begin{lemma}
    $x^n$ --- это последовательность, в которой на $n$-ом месте стоит единица, а остальные элементы --- нули.
\end{lemma}

\begin{proof}
    Индукция по $n$.

    \textbf{Шаг индукции} ($n = 1$). По обозначению.

    \textbf{База индукции}.
    $$
    x^n = x^{n - 1} \cdot x = (0, 0, \ldots, 0, 1, 0, \ldots) \cdot (0, 1, 0, \ldots).
    $$

    Из формулы умножения финитных последовательностей, получаем $c_n = 1$, а остальные $0$. Действительно, ведь ненулевые только $a_1$ и $b_{n - 1}$.
\end{proof}

Таким образом, многочлен $f = (a+0, a_1, \ldots, a_n, 0, 0, \ldots)$ может быть записан как
$$
f(x) = a_0 + a_1x + \ldots + a_nx^n.
$$

\begin{definition}
    Степень $\deg f$ многочлена $f \ne (0, 0, \ldots)$ равна максимальному $n$, такому что $a_n \ne 0$.
\end{definition}

\begin{theorem}
    \begin{enumerate}[nolistsep]
        \item $\deg (f + g) \leqslant \max\{\deg f, \deg g\}$;
        \item Если $R$ --- область целостности, то $\deg fg = \deg f + \deg g$.
    \end{enumerate}
\end{theorem}

\begin{proof}
    Пусть $f = (a_0, a_1, \ldots, a_m, 0, 0, \ldots)$, $g = (b_0, b_1, \ldots, b_n, 0, 0, \ldots)$. 
    \begin{enumerate}
        \item Тогда 
            $$
            f + g = (a_0 + b_0, a_1 + b_1, \ldots),
            $$

            при $j > \max\{m, n\}$, то элемент с номером $j$ будет нулевой.
        \item Пусть $fg = (c_0, c_1, \ldots)$. Тогда $c_k = \sum\limits_{j = 0}^ka_jb_{k - j}$. Если $k > m + n$, то $c_k = 0$. При этом $c_{m + n} = a_mb_n \ne 0$, т.\,к. $\mathcal{R}$ целостное. Значит, $\deg fg = m + n$.
    \end{enumerate}
\end{proof}

\begin{remark}
    Если кольцо $\mathcal{R}$ не является целостным, то второй пункт теоремы не верен. Например, в кольце $\Z_4[x]$ выполнено $\deg(2x + 1) = 1$, но $(2x + 1)^2 = 1$ и $\deg 1 = 0$.
\end{remark}

\begin{theorem}
    Кольцо многочленов над целостным кольцом целостное.
\end{theorem}

\begin{proof}
    Если $fg = 0$, то это противоречит $\deg fg = \deg f + \deg g$.
\end{proof}

\begin{theorem}[Безу]
    Пусть $\mathcal{R}$ --- коммутативное ассоциативное кольцо с единицей, $c \in \mathcal{R}$ и $f \in \mathcal{R}[x]$. Тогда $f(x) = (x - c)q(x) + r$ ($r \in \mathcal{R}$), причём $q(x)$ --- многочлен, а $r = f(c)$.
\end{theorem}

\begin{proof}
    Пусть $\mathcal{R}$ --- область целостности, $c \in \mathcal{R}$ и $f \in \mathcal{R}[x]$. Положим $\widetilde{x} = x - c$, тогда $x = \widetilde{x} + c$. Если подставить это в $f$ и раскрыть кобки, получим многочлен $\widetilde{f}$, причём $\widetilde{f}(\widetilde{x}) = f(x - c)$. Получаем разложение $f(x)$ по степеням $x - c$, т.\,е. выражения вида
    $$
    f(x) = b_n(x - c)^n = b_{n - 1}(x - c)^{n - 1} + \ldots + b_0.
    $$
    Причём, $f(c) = r$.
\end{proof}

Деление с остатком на $x - c$ осуществляется по \textit{схеме Горнера}. А именно, пусть
$$
a_0x^n + a_1x^{n - 1} + \ldots + a_{n - 1}x + a_n = (x - c)(b_0x^{n - 1} + b_1x^{n - 2} + \ldots + b_{n - 2}x + b_{n - 1}) + r.
$$

Приравнивая коэффициенты при соответствующих степенях $x$, получаем цепочку равенств:
$$
\begin{array}{l}
    a_0 = b_0,\\
    a_1 = b_1 - cb_0,\\
    a_2 = b_2 - cb_1,\\
    \vdots\\
    a_{n - 1} = b_{n - 1} - cb_{n - 2},\\
    a_n = r - cb_{n - 1}.
\end{array}
$$

Отсюда находим следующие рекуррентные формулы для $b_0, b_1, \ldots, b_{n - 1}, r$:
$$
\begin{array}{l}
    b_0 = a_0,\\
    b_1 = a_1 + cb_0,\\
    b_2 = a_2 + cb_1,\\
    \vdots\\
    b_{n - 1} = a_{n - 1} + cb_{n - 2},\\
    r = a_n + cb_{n - 1}.
\end{array}
$$

Исходные данные и результаты вычислений удобно расположить в таблице:

\begin{center}
\begin{tabular}{c|cccccc}
    {} & $a_0$ & $a_1$ & $a_2$ & \ldots & $a_{n - 1}$ & $a_n$\\
    \hline\vspace{1mm}
    $c$ & $b_0$ & $b_1$ & $b_2$ & \ldots & $b_{n - 1}$ & $r$
\end{tabular}
\end{center}

Чтобы разложить многочлен $f$ по степеням $x - c$ можно последовательным делением многочлена $f$ на $x - c$ с остатком. А именно, при первом делении получается остаток $b_0$ и неполное частное
$$
f_1 = b_1 + b_2(x - c) + \ldots + b_n(x - c)^{n - 1};
$$
при делении $f_1$ на $x - c$ получается остаток $b_1$ и т.\,д. Записывать можно как последовательную схему Горнера, используя предыдущую строку в качестве входных данных для следующей.


