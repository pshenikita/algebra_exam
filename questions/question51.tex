\section{Несократимые правильные и простейшие рациональные дроби. Разложение правильной дроби в сумму простейших}

\begin{definition}
    Рациональная дробь называется \textbf{правильной}, если $\deg f < \deg g$.
\end{definition}

\begin{statement}
    Любую дробь можно представить в виде суммы многочлена и правильной дроби, причём такое представление единственно.
\end{statement}

\begin{proof}
    Поделим числитель на заменатель с остатком: $f = gq + r$. Тогда
    $$
    \frac{f}{g} = q + \frac{r}{g}.
    $$
    Единственость такого представления следует из единственности деления с остатком.
\end{proof}

\begin{lemma}
    Всякая правильная рациональная дробь вида
    $$
    \frac{f}{g_1g_2\ldots g_s},
    $$
    где $g_1, g_2, \ldots, g_s$ попарно взаимно просты, разлагается в сумму правильных дробей со знаменателями $g_1, g_2, \ldots, g_s$, причём единственным образом.
\end{lemma}

\begin{proof}
    Докажем это утверждение индукцией по $s$. Положим $g = g_1g_2\ldots g_s$. При $s = 2$, согласно теореме 44.2, существуют такие многочлены $u$ и $v$, что $ug_1 + vg_2 = 1$. Домножив это равенство на $f$, получаем $f = u^\ast g_1 + v^\ast g_2$. Разделив на $g$, получим
    $$
    \frac{f}{g} = \frac{v^\ast}{g_1} + \frac{u^\ast}{g_2}.
    $$

    Так как дробь $\displaystyle\frac{f}{g}$ правильная, то сумма целых частей дробей $v^\ast / g_1$ и $u^\ast / g_2$ равна нулю. Выделив их, мы получим разложение дроби $\displaystyle\frac{f}{g}$ в сумму правильных дробей со знаменателями $g_1$ и $g_2$.

    При $s > 2$ заметим, что многочлены $g_1$ и $g_2\ldots g_s$ взаимно просты, и по доказанному дробь $\displaystyle\frac{f}{g}$ разлагается в сумму правильных дроблей со знаменателями $g_1$ и $g_2\ldots g_s$. По предположению индукции правильная дробь со знаменателем $g_2\ldots g_s$ разлагается в сумму правильных со знаменателями $g_2, \ldots, g_s$.

    Докажем единственность. Пусть для данной дроби существует два разложения:
    $$
    \frac{f_1}{g_1} + \ldots + \frac{f_s}{g_s} = \frac{f_1^\ast}{g_1} + \ldots + \frac{f_s^\ast}{g_s}.
    $$
    Перенесём всё в одну сторону:
    $$
    \frac{f_1 - f_1^\ast}{g_1} + \ldots + \frac{f_s - f_s^\ast}{g_s} = 0.
    $$
    Отсюда из взаимной простоты легко понять справедливость этого утверждения для $s = 2$, а для остальных добивается индукцией.
\end{proof}

\begin{definition}
    Рациональная дробь $\displaystyle \frac{f}{g}$ называется \textbf{простейшей}, если $g = p^k$, где $p \in \mathbb{F}[x]$ --- неприводимый многочлен, и $\deg f < \deg p$.
\end{definition}

В частности, всякая дробь вида
$$
\frac{a}{(x - c)^k},\quad(a, c \in \mathbb{F})
$$
является простейшей. В случае $\mathbb{F} = \C$ дробями такого вида исчерпываюся все простейшие дроби. В случае $\mathbb{F} = \R$ имеются ещё простейшие дроби вида
$$
\frac{ax + b}{(x^2 + px + q)^k},\quad(a, b, p, q \in \R),\text{ где }p^2 - 4q < 0.
$$

\begin{theorem}
    Всякая правильная рациональная дробь $\displaystyle{f}{g}$ разлагается в сумму простейших дробей.
\end{theorem}

\begin{proof}
    Пусть $g = p_1^{k_1}p_2^{k_2}\ldots p_s^{k_s}$. Ввиду леммы 51.1 дробь $\displaystyle\frac{f}{g}$ разлагается в сумму правильных дробей со знаменателями $p_1^{k_1}, p_2^{k_2}, \ldots, p_s^{k_s}$. Поэтому нам достаточно доказать теорему в случае, когда $g = p^k$, где $p$ --- неприводимый многочлен. В этом случае, разделив $f$ на $p$ с остатком, получим
    $$
    \frac{f}{p^k} = \frac{f_1}{p^{k - 1}} + \frac{r}{p^k},\quad\deg r < \deg p.
    $$
    Второе слагаемое является простейшей дробью, а первое является правильной дробью как разность правильных дробей. Продолжая процесс, получим требуемое разложение.

    Единственность следует из единственности разложения в лемме 51.1.
\end{proof}


