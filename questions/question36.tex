\section{Гомоморфизм и изоморфизм алгебраических структур. Комплексные числа. Доказательство того, что комплексные числа образуют поле}

\begin{definition}
    \textbf{Алгебраическая структура} --- это множество $X$ с несколькими операциями $X^n \rightarrow X$ (возможно, для различных $n$), удовлетворяющих некоторым аксиомам.
\end{definition}

\begin{definition}
    Пусть есть две алгебраичекие структуры с одинаковым количеством операций от одинакового количества переменных. \textbf{Гомоморфизмом} из одной в другую называется отображение множеств, переводящее операции в операции.

    То есть, если имеем две алгебраические структуры $A$ и $B$ с операциями $\alpha_i: A^{n_i} \rightarrow A$ и $\beta_i: A^{n_i} \rightarrow B$ и $\varphi: A \rightarrow B$ --- гомоморфизм, то
    $$
    \varphi(\alpha_i(x_1, \ldots, x_{n_i})) = \beta_i(\varphi(x_1), \ldots, \varphi(x_{n_i})).
    $$

    Изоморфные структуры одинаковы с алгебраической точки зрения, если какая-то алгебраическая аксиома выполнена для одной структуры, то она выполнена и для изоморфной ей. Изоморфные структуры обозначаются $A \cong B$.

    Изоморфизм из множества на себя называется \textbf{автоморфизмом}.
\end{definition}

\textbf{Примеры гомоморфизмов}:
\begin{enumerate}
        \begin{minipage}{.5\textwidth}
        \item $\varphi: (\Z, +) \rightarrow (\Z_n, +)$, $\varphi(k) = [k]_n$;
        \item $(\mathrm{GL}_n(\R), \boldsymbol{\cdot}) \rightarrow (\R^\ast, \boldsymbol{\cdot})$, $\varphi(M) = \det M$;
        \end{minipage}
        \begin{minipage}{.5\textwidth}
        \item $\varphi: (\Z, +, \boldsymbol{\cdot}) \rightarrow (\Z_n, +, \boldsymbol{\cdot})$, $\varphi(k) = [k]_n$;
        \item $\varphi: (\Z, +, \boldsymbol{\cdot}) \rightarrow (\Q, +, \boldsymbol{\cdot})$, $\varphi(k) = k$.
        \end{minipage}
\end{enumerate}

\begin{definition}
    \textbf{Комплексные числа} --- это множество $\R^2$ с операциями
    $$
    (a, b) + (c, d) \vcentcolon = (a + c, b + d),\quad (a, b) \cdot (c, d) \vcentcolon= (ac - bd, ad + bc).
    $$
    Множество комплексных чисел обозначается $\C$.
\end{definition}

\begin{theorem}
    Комплексные числа образуют поле.
\end{theorem}

Чтобы доказать теорему, нам понадобится следующая лемма.

\begin{lemma}
    Алгебраичекая структура комплексных чисел изоморфна алгебраической структуре $\mathbb{M}$ матриц вида
    $
    \begin{pmatrix}
        a & -b\\
        b & a
    \end{pmatrix}
    $, $a, b \in \R$ с операциями сложения и умножения.
\end{lemma}

\begin{proof}
    Определим отображение
    $$
    \varphi: (a, b) \in \C \mapsto
    \begin{pmatrix}
        a & -b\\
        b & a
    \end{pmatrix} \in \mathbb{M}.
    $$
    Очевидно, что $\varphi$ --- биекция. Кроме того,
    $$
        \varphi((a, b) + (c, d)) = \varphi(a + c, b + d) =
        \begin{pmatrix}
            a + c & -(b + d)\\
            b + d & a + c
        \end{pmatrix} = 
        \begin{pmatrix}
            a & -b\\
            b & a
        \end{pmatrix} +
        \begin{pmatrix}
            c & -d\\
            d & c
        \end{pmatrix} = \varphi(a, b) + \varphi(c, d).
    $$
    $$
        \varphi((a, b) \cdot (c, d)) = \varphi(ac - bd, ad + bc) =
        \begin{pmatrix}
            ac - bd & -(ad + bc)\\
            ad + bc & ac - bd
        \end{pmatrix} = 
        \begin{pmatrix}
            a & -b\\
            b & a
        \end{pmatrix} \cdot
        \begin{pmatrix}
            c & -d\\
            d & c
        \end{pmatrix} = \varphi(a, b) \cdot \varphi(c, d).
    $$
    Таким образом, $\varphi$ --- гомоморфизм. Однако, очевидно, что $\varphi$ --- биекция. То есть, $\varphi$ --- гомоморфизм.
\end{proof}

Теперь докажем теорему:

\begin{proof}
    Проверим непосредственно выполнение всех аксиом:
    \begin{enumerate}
        \item $\C$ коммутативно относительно умножения, это следует из формулы $(a, b) \cdot (c, d) = (ac - bd, ad + bc)$.
        \item $\C$ является кольцом, т.\,к., очевидно, является абелевой группой относительно сложения и для него выполнена дистрибутивность, т.\,к. $\C \cong \mathbb{M}$, а для $\mathbb{M}$ она выполнена. Ассоциативность выполнена по той же причине. Далее, в этом кольце есть единица --- $(1, 0)$, и ко всему прочему, каждый ненулевой элемент обратим, т.\,к. $\C \cong \mathbb{M}$ и
            $$
            (a, b) \mapsto
            \begin{pmatrix}
                a & -b\\
                b & a
            \end{pmatrix},\text{{} причём {}}
            \det\begin{pmatrix}
                a & -b\\
                b & a
            \end{pmatrix}^{-1} = a^2 + b^2.
            $$
            Таким образом, эта матрица вырожденна (или, что равносильно, элемент $(a, b)$ обратим) тогда и только тогда, когда $(a, b) = (0, 0)$.
    \end{enumerate}
\end{proof}

\textbf{Вложение вещественных чисел}. Рассмотрим отображение
$$
\psi: \R \rightarrow \big\{(x, 0) \;\vcentcolon\;x \in \R\big\} \subset \C,\quad \psi(x) = (x, 0).
$$

Легко видеть, что $\psi$ --- изоморфизм. Таким образом, поле $(\R, +, \boldsymbol{\cdot})$ изоморфно подполю $\C$. В дальнейшем не будем различать эти поля.

Заметим, что $(a, 0) \cdot (c, d) = (ac, ad)$. Получаем, что на $\C$ есть операции сложения и умножения на $\R$. С этими операциями $\C$ --- векторное пространство над $\R$, изоморфное $\R^2$. Его базис --- это $(1, 0) =\vcentcolon 1$ и $(0, 1) =\vcentcolon i$. Заметим, что $i^2 = -1$. Значит, комплексные числа можно записывать в виде $a + bi$.

Причём, операции записываются естественным образом:
$$
(a + bi) + (c + di) = (a + b) + (c + d)i,\quad (a + bi) \cdot (c + di) = (ac - bd) + (ad + bc)i.
$$

\begin{definition}
    Будем называть $i$ \textbf{мнимой единицей}.
\end{definition}


