\section{Левые смежные классы по подгруппе. Индекс подгруппы. Теорема Лагранжа}

\begin{definition}
    Пусть $G$ --- группа и $H$ --- её подгруппа. Будем говорить, что элементы $g_1, g_2 \in G$ \textbf{сравнимы по модулю $H$}, и писать $g_1 \equiv g_2\ (\mod H)$, если
    $$
    g_1^{-1}g_2 \in H,\eqno(\ast)
    $$
    т.\,е. $g_2 = g_1h$, где $h \in H$.
\end{definition}

\begin{remark}
    Это определение обобщает определение сравнимости целых чисел по модулю $n$, которое получается в случае $G = \Z$, $H = n\Z$.
\end{remark}

\begin{statement}
    Отношение сравнимости по модулю $H$ является отношением эквивалентности.
\end{statement}

\begin{proof}
    \begin{enumerate}[nolistsep]
        \item $g \equiv g\ (\mod H)$, т.\,к. $g^{-1}g = e \in H$;
        \item если $g_1 \equiv g_2\ (\mod H)$, т.\,е. $g_1^{-1}g_2 \in H$, то $g_2 \equiv g_1\ (\mod H)$, т.\,к.
            $$
            g_2^{-1}g_1 = (g_1^{-1}g_2)^{-1} \in H;
            $$
        \item если $g_1 \equiv g_2\ (\mod H)$ и $g_2 \equiv g_1\ (\mod H)$, т.\,е. $g_1^{-1}g_2, g_2^{-1}g_3 \in H$, то $g_1 \equiv g_3\ (\mod H)$, т.\,к.
            $$
            g_1^{-1}g_3 = (g_1^{-1}g_2)(g_2^{-1}g_3) \in H.
            $$
    \end{enumerate}
\end{proof}

\begin{definition}
    Классы этой эквивалентности называются \textbf{левыми смежными классами} группы $G$ по подгруппе $H$. Ясно, что смежный класс, содержащий элемент $g$, имеет вид
    $$
    gH \vcentcolon = \{gh\;\vcentcolon\;h \in H\}.
    $$
\end{definition}

\begin{remark}
    Умножение не обязано быть коммутативным, поэтому мы получим, вообще говоря, другое отношение эквивалентности (и другие классы), взяв вместо условия $(\ast)$ аналогичное ему $g_2g_1^{-1} \in H$. Классы этой эквивалентности называются \textbf{правыми смежными классами} группы $G$ по подгруппе $H$. Они имеют вид
    $$
    Hg \vcentcolon = \{hg\;\vcentcolon\;h \in H\}.
    $$
    Заметим, что инверсия $g \mapsto g^{-1}$ устанавливает биекцию между множествами левых и правых смежных классов. А именно,
    $$
    (gH)^{-1} = Hg^{-1}.
    $$
\end{remark}

\begin{definition}
    Множество левых смежных классов группы $G$ по подгруппе $H$ обозначается через $G / H$. Число смежных классов (левых или правых), если оно конечно, называется \textbf{индексом} подгруппы $H$ и обозначается через $|G \vcentcolon H|$.
\end{definition}

\begin{theorem}[Лагранж]
    Если $G$ --- конечная група и $H$ --- любая её подгруппа, то
    $$
    |G| = |G \vcentcolon H|\cdot |H|.
    $$
\end{theorem}

\begin{proof}
    Все смежные классы $gH$ содержат одно и то же число элементов, равное $|H|$. Поскольку они образуют разбиение группы $G$ (как классы эквивалентности), порядок группы $G$ равен произведению их числа на $|H|$.
\end{proof}

\subsection*{Отношения эквивалентности и кольца вычетов\footnotemark}

\footnotetext{Я не знал, куда ещё вставить\ldots}

\begin{definition}
    ПУсть $M$ --- какое-либо множество. Всякое подмножество $\R \subseteq M \times M$ называется \textbf{отношением} на множестве $M$. Если $(a, b) \in R$, то говорят, что элементы $a$ и $b$ находятся в отношении $R$ и пишут $aRb$.
\end{definition}

\begin{definition}
    Отношение называется \textbf{отношение эквивалентности}, если оно обладает следующими свойствами:
    \begin{enumerate}[nolistsep]
        \item $aRa$ (рефлексивность)
        \item $aRb \Rightarrow bRa$ (симметричность)
        \item $(aRb \wedge bRc) \Rightarrow aRc$ (транзитивность)
    \end{enumerate}
    Отношение эквивалентности обычно записывается как $a \underset{R}{\sim} b$ или просто $a \sim b$.
\end{definition}

\begin{definition}
    Пусть $R$ --- отношение эквивалентности на множестве $M$. Для каждого $M$ положим
    $$
    R(a) \vcentcolon = \{b \in M\;\vcentcolon\;a \underset{R}{\sim} b\}.
    $$
\end{definition}

\begin{lemma}
    $R(a) \cap R(b) \ne \varnothing \Rightarrow R(a) = R(b)$.
\end{lemma}

\begin{proof}
    Пусть $c \in R(a) \cap R(b)$. Тогда $a \underset{R}{\sim} c$ и $c \underset{R}{\sim} b$, отсюда $a \underset{R}{\sim} b$, а значит, они лежат в одном классе, поэтому $R(a) = R(b)$.
\end{proof}

\begin{definition}
    Таким образом, подмножества $R(a)$ образуют разбиение множества $M$ (т.\,е. покрывают его и попарно не пересекаются). Они называются \textbf{классами эквивалентности отношения $R$}.
\end{definition}

\begin{definition}
    Множество, элементами которого являются классы эквивалентности отношения $R$, называется \textbf{фактормножеством} множества $M$ и обозначается через $M / R$ (если $R$ --- отношение эквивалентности, иногда используется обозначение $M / \sim$). Отображение
    $$
    a \in M \mapsto R(a) \in M / R
    $$
    называется \textbf{отображением факторизации}.
\end{definition}

\begin{definition}
    Отношение эквивалентности $R$ на множестве $M$ называется \textbf{согласованным} с операцией $\ast$, если
    $$
    a \underset{R}{\sim} a^\prime, b \underset{R}{\sim} b^\prime \Rightarrow a \ast b \underset{R}{\sim} b \ast b^\prime.
    $$
    В этом случае на фактормножестве $M / R$ также можно определить операцию $\ast$
\end{definition}

\begin{definition}
    Класс эквивалентности сравнимых по модулю $n$ целых чисел, содержащий $a$, будем называть \textbf{вычетом числа $a$ по модулю $n$} и обозначать через $[a]_n$ (или просто $[a]$, если понятно, какое $n$ имеется в виду). Фактормножество множества $\Z$ по отношению сравнимости $\mod n$ обозначается через $\Z_n$. Мы можем писать, что
    $$
    \Z_n = \{[0]_n, [1]_n, \ldots, [n - 1]_n\}.
    $$
\end{definition}

\begin{lemma}
    Отношение сравнимости по модулю $n$ согласовано с операциями сложения и умножения в $\Z$.
\end{lemma}

\begin{proof}
    Пусть
    $$
    a \equiv q^\prime\ (\mod n),\quad b \equiv b^\prime\ (\mod n).
    $$
    Тогда
    $$
    a + b \equiv a^\prime + b^\prime\ (\mod n)
    $$
    и, аналогично,
    $$
    ab = \equiv a^\prime b \equiv a^\prime b^\prime\ (\mod n).
    $$
\end{proof}

\begin{definition}
    Таким образом, можем определить в множестве $\Z_n$ операции сложения и умножения по формулам
    $$
    [a]_n + [b]_n \vcentcolon= [a + b]_n,\quad [a]_n[b]_n \vcentcolon= [ab]
    $$
\end{definition}


