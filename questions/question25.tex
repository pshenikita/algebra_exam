\section{Единственность с точностью до пропорциональности линейной кососимметрической функции строк/столбцов. Определитель произведения матриц}

\begin{theorem}
    Всякая функция $\Phi$ на множестве квадратных матриц порядка $n$, являющаяся кососимметрической полилинейной функцией строк матрицы, имеет вид
    $$
    \Phi(A) = \Phi(E) \det A.
    $$
\end{theorem}

\begin{remark}
    Во всех доказательствах этой теоремы, которые мне известны, без дополнительных объяснений используется утверждение <<если у кососимметрической функции два аргумента равны, то она зануляется>>. Но я не понимаю, как его доказать, не упираясь при этом в проблему, описанную в примечании после доказательства теоремы 24.1.
\end{remark}

\begin{proof}
    Пусть $\Phi$ --- полилинейная кососимметрическая функция строк матрицы. Пусть $e_1, e_2, \ldots, e_n$ --- единичные строки. Тогда из линейности $\Phi$
    $$
    \begin{array}{c}\displaystyle
        \Phi(A) = \Phi(A_{(1)}, A_{(2)}, \ldots, A_{(n)}) = \Phi\left(\sum_{i_1}a_{1 i_1}e_{i_1}, \sum_{i_2}a_{2 i_2}e_{i_2}, \ldots, \sum_{i_n}a_{n i_n}e_{i_n}\right) = {}\\\displaystyle{} = \sum_{i_1, \ldots, i_n \in S_n}a_{1 i_1} a_{2 i_2} \ldots a_{n i_n} \Phi(e_{i_1}, e_{i_2}, \ldots, e_{i_n}).
    \end{array}
    $$

    При этом, если какие-то из $i_1, \ldots, i_n$ равны, то $\Phi(e_{i_1}, \ldots, e_{i_n}) = 0$ в силу кососимметричности функции. Поэтому можно считать только слагаемые, у которых $i_1, \ldots, i_n$ попарно различны. Тогда они однозначно задают перестановку $(i_1, \ldots, i_n)$, а эта перестановка в свою очередь сопоставляется подстановке $\sigma: j \in \Omega_n \mapsto i_j \in \Omega_n$. Получаем
    $$
    \Phi(A) = \sum_{\sigma \in S_n} \Phi(e_{\sigma(1)}, \ldots, e_{\sigma(n)}) a_{1 \sigma(1)} \ldots a_{n \sigma(n)}.
    $$

    Из кососимметричности $\Phi$, количество операций, требуемых, чтобы переставить строки $e_{\sigma(i)}, \ldots, e_{\sigma(n)}$ в том же порядке, в котором они шли в единичной матрице, равно количеству инверсий в перестановке $\sigma$, а значит,
    $$
    \Phi(e_{\sigma(1)}, e_{\sigma(2)}, \ldots, e_{\sigma(n)}) = \sgn\sigma\cdot\Phi(E).
    $$

    Подставляя этот результат вы полученное ранее выражение, получим требуемое.
\end{proof}

\begin{theorem}
    $\det AB = \det A \cdot \det B$.
\end{theorem}

\begin{proof}
    $\det AB$ является полилинейной кососимметрической функцией от строк матрицы $AB$. А эти строки, в свою очередь, являются линейными комбинациями строк $A$ (по лемме 16.1). Поэтому $\det AB$ --- полилинейная кососиммерическая функция от строк матрицы $A$. Отсюда
    $$
    \det AB = \det EB \cdot \det A = \det B \cdot \det A.
    $$
\end{proof}


