\section{Векторное пространство. Простейшие свойства из аксиом. Подпространство. Критерий того, что подмножество является подпространством}

\begin{definition}
    \textbf{Векторным пространством} над полем $\mathcal{K}$ называется такое множество $V$ с операциями слоежния и умножения на элементы поля $\mathcal{K}$, обладающими следующими свойствами:
    \begin{enumerate}
            \item $\forall\!\:a, b, c \in V\;(a + b) + c = a + (b + c)$
            \item $\exists\!\:\boldsymbol{0} \in V : \forall\!\:v \in V\;v + \boldsymbol{0} = v$
            \item $\forall\!\:v \in V\,\exists\!\:(-v) \in V: (-v) + v = \boldsymbol{0}$.\quad\smash{\raisebox{.5\dimexpr\baselineskip+\itemsep+\parskip}{$\left.\rule{0pt}{.5\dimexpr4\baselineskip+3\itemsep+3\parskip}\right\}\text{$V$ --- абелева группа по $+$}$}}
            \item $\forall\!\: a, b \in V\;a + b = b + a$
            \item $\forall\!\: \lambda \in \R, u, v \in V\; \lambda \cdot (u + v) = \lambda \cdot u + \lambda \cdot v$
            \item $\forall\!\: \lambda, \mu \in \R, v \in V\; (\lambda + \mu) \cdot v = \lambda \cdot v + \mu \cdot v$
            \item $\forall\!\: \lambda, \mu \in \R, v \in V\; (\lambda\mu) \cdot v = \lambda \cdot (\mu \cdot v)$
            \item $\forall v \in V\;1 \cdot v = v$
    \end{enumerate}
\end{definition}

\begin{remark}
    Докажем, что аксиома 4 выводится через остальные. Сделаем это в несколько шагов (над знаками равенства указан номер аксиомы или пункта доказательства, благодаря которой сделан переход):
    \begin{enumerate}
        \item[$1^\ast$.] $0 \cdot v = (0 + 0) \cdot v \overset{6}{=} 0 \cdot v + 0 \cdot v \Rightarrow $ \fbox{$0 \cdot v = \boldsymbol{0}$}
        \item[$2^\ast$.] $(-1) \cdot v + v \overset{8}{=} (-1 \cdot v) + 1 \cdot v \overset{6}{=} (-1 + 1) \cdot v = 0 \cdot v \overset{1^\ast}{=} \boldsymbol{0} \Rightarrow$ \fbox{$(-1) \cdot v = -v$}
        \item[$3^\ast$.] $v - v \overset{8, 2^\ast}{=\joinrel=} 1 \cdot v + (-1) \cdot v \overset{6}{=} (1 - 1) \cdot v = 0 \cdot v = \boldsymbol{0} \Rightarrow$ \fbox{$v - v = \boldsymbol{0}$}
    \end{enumerate}

    Теперь выведем 4 аксиому из остальных:
    $$
    \begin{array}{c}\displaystyle
        u + v \overset{2}{=} (u + v) + \boldsymbol{0} \overset{3}{=} (u + v) + (-(v + u) + (v + u)) \overset{1}{=} ((u + v) - (v + u)) + (v + u) = {}\\\displaystyle {} = (u + \underbrace{v - v}_{{} = 0} - u) + (v + u) \overset{3^\ast}{=} (\underbrace{u - u}_{{} = 0}) + (v + u) = v + u.
    \end{array}
    $$

    Антон Александрович сказал, что остальные не выводятся. Но мы с Костей Зюбиным не смогли доказать это для 5-ой аксиомы. Доказательство независимости от остальных для каждой аксиомы проводится так: нужно привести пример такой структуры, в которой выполняются все аксиомы, кроме выбранной.
    \begin{itemize}
        \item \textbf{1 аксиома}. Рассмотрим множество $M = \{e, a, b\}$ с операцией $\ast$, заданной следующей таблицей:
            $$
            \begin{array}{c || c | c | c}
                \ast & e & a & b\\
                \hline\hline
                e & e & a & b\\
                \hline
                a & a & a & e\\
                \hline
                b & b & e & b
            \end{array}
            $$

            Заметим, что операция $\ast$ коммутативна, в $M$ существует нейтральный элемент $e$ по $\ast$ и каждый элемент имеет обратный по $\ast$. Однако эта операция не ассоциативна:
            $$
            (b \ast a) \ast a = e \ast a = a,\quad b \ast (a \ast a) = b \ast a = e.
            $$

            Теперь возьмём алгебраическую систему $V = (\R \times M, +, \star)$ с операциями, определёнными по следующим правилам:
            $$
            u + v = (a, x) + (b, y) \vcentcolon = (a + b, x \ast y),\quad \lambda \star v = \lambda \star (a, x) \vcentcolon = (\lambda a, x).
            $$

            Аксиома 1 не выполнена, т.\,к. $\ast$ неассоциативна. Выполнение аксиом 2-4 следует из того, что они выполняются для $+$ над $\R$ и $\ast$ над $M$. Проверим выполнение остальных аксиом:
            $$\footnotesize
            \begin{array}{rl}
                5:\; & \lambda \star (a + b, x \ast y) = (\lambda(a + b), x \ast y) = (\lambda a + \lambda b, x \ast y) = (\lambda a, x) + (\lambda b, y) = \lambda\star((a, x) + (b, y))\\
                6:\; & (\lambda + \mu) \cdot (a, x) = ((\lambda + \mu)a, x) = (\lambda a + \mu a, x \ast x) = (\lambda a, x) + (\mu a, x) = \lambda \star (a, x) + \mu \star (a, x)\\
                7:\; & (\lambda\mu) \star (a, x) = (\lambda\mu a, x) = \lambda \star (\mu a, x)\\
                8:\; & 1 \star (a, x) = (1 \cdot a, x) = (a, x)
            \end{array}
            $$
        \item \textbf{2 аксиома}. Без второй аксиомы нельзя ввести третью, поэтому её удаление не имеет смысла.
        \item \textbf{3 аксиома}. Рассмотрим алгебраическу систему $V = (\R \cup \{\infty\}, +, \boldsymbol{\cdot})$. Доопределим сложение и умножение для $\infty$ следующим образом:
            $$
            \infty + a = a + \infty \vcentcolon = \infty,\quad \lambda \cdot \infty \vcentcolon = \infty.
            $$

            Выполнение аксиом 1, 2 и 4 сразу вытекает из определения. Аксиома 3 не выполнена, т.\,к. у $\infty$ нет обратного по $+$. Выполнение аксиом 5-8 проверяется перебором нескольких случаев.
        \item \textbf{6 аксиома}. Рассмотрим алгебраическую систему $V = (\R, +, \star)$, в которой сложение определено так же, как в действительных числах, а умножение так:
            $$
            \lambda \star v \vcentcolon = v.
            $$

            Аксиомы 1-4 выполнены, т.\,к. они выполнены для $\R$ и $+$. Выполнение аксиом 5, 7 и 8 сразу вытекает из определения. Аксиома 6 не выполнена:
            $$
            u + u = 1 \star u + 1 \star u,\quad (1 + 1) \star u = u.
            $$
        \item \textbf{7 аксиома}. Рассмотрим $\R$ как векторное пространство над полем $\Q$ с базисом $M \supset \{1, \sqrt{2}\}$. Пусть отображение $f: \R \rightarrow \R$ задаётся своими значениями на числах из $M$, а для других чисел определяется соотношением
            $$
            f(q_1v_1 + q_2v_2 + \ldots + q_nv_n) = q_1f(v_1) + q_2f(v_2) + \ldots + q_nf(v_n),
            $$
            где $v_i$ --- некоторые векторы из базиса $M$. Такое отображение является линейным, сохраняющим все рациональыне числа.

            Теперь возьмём алгебраическую систему $V = (\R, +, \star)$, в которой сложение $+$ определено естественным образом, а умножение $\star$ определяется через $f$ и естественное умножение $\boldsymbol{\cdot}$:
            $$
            \lambda \star u \vcentcolon = f(\lambda) \cdot u.
            $$

            Аксиомы 1-4 выполнены, т.\,к. они выполнены для $\R$ и $+$. Выполнение аксиомы 5 проверяется непосредственно. Аксиома 6 выполнена, т.\,к. отображение $f$ линейно. Проверим, что аксиома 7 не выполнена:
            $$
            \sqrt{2} \star (\sqrt{2} \star u) = \sqrt{2} \star u = u,\quad (\sqrt{2} \star \sqrt{2}) \star u = 2u.
            $$

            Выполнение аксиомы 8 вытекает из определения.
        \item \textbf{8 аксиома}. Рассмотрим алгебраическую систему $V = (\R, +, \star)$, в которой сложение определено естественным образом, а умножение так:
            $$
            \lambda \star u \vcentcolon = \boldsymbol{0}.
            $$

            Аксиомы 1-4 выполнены, т.\,к. они выполнены для $\R$ и $+$. Выполнение аксиом 5, 6 и 7 проверяется непосредственно. Аксиома 8 не выполнена.
    \end{itemize}
\end{remark}

\begin{theorem}[Свойства векторного пространства]
    \begin{enumerate}
            \item В любом векторном пространстве нулевой вектор единственный
            \item Для любого вектора $v$ противоположный вектор $-v$ единственный
            \item Для любого вектора выполнено $0\cdot v = \boldsymbol{0}$
            \item Для любого числа $\lambda$ выполнено $\lambda \cdot \boldsymbol{0} = \boldsymbol{0}$
            \item Для любого вектора $v$ выполнено $(-1) \cdot v = -v$
            \item Для любого вектора $v$ выполнено $-(-v) = v$
            \item Для любых векторов $u$ и $v$ $-(u + v) = (-u) + (-v)$
            \item $\lambda v = \boldsymbol{0} \Rightarrow (\lambda = 0) \wedge (v = \boldsymbol{0})$
    \end{enumerate}
\end{theorem}

\begin{proof}
    \begin{enumerate}
        \item Пусть есть два нулевых вектора --- $\boldsymbol{0}_1$ и $\boldsymbol{0}_2$. Тогда $\boldsymbol{0}_1 + \boldsymbol{0}_2 = \boldsymbol{0}_1$ с одной стороны (т.\,к. $\boldsymbol{0}_2$ нулевой) и $\boldsymbol{0}_1 + \boldsymbol{0}_2 = \boldsymbol{0}_2$ (т.\,к. $\boldsymbol{0}_1$ нулевой). Отсюда $\boldsymbol{0}_1 = \boldsymbol{0}_2$.
        \item Пусть таких векторов два: $(-v)_1$ и $(-v)_2$. Тогда имеем
            $$
            (-v)_1 = \underbrace{(-v)_1 + v}_{{} = \boldsymbol{0}} + (-v)_2 = (-v)_2.
            $$
        \item Доказывалось ранее.
        \item $\lambda\cdot\boldsymbol{0} = \lambda\cdot(\boldsymbol{0} + \boldsymbol{0}) = \lambda\cdot\boldsymbol{0} + \lambda\cdot\boldsymbol{0}$. Вычитая $\lambda\cdot\boldsymbol{0}$ из обеих частей равенства, получаем требуемое.
        \item Доказывалось ранее.
        \item $-(-v) + (-v) = (-1) \cdot (-v) + (-1) \cdot v = (-1) \cdot \underbrace{(-v + v)}_{{} = \boldsymbol{0}} = \boldsymbol{0}$
        \item $-(u + v) = (-1) \cdot (u + v) = (-1)u + (-1)v = (-u) + (-v)$.
        \item Допустим, $\lambda \ne 0$. Тогда существует число $1 / \lambda \ne 0$. Имеем
            $$
            \boldsymbol{0} = \frac{1}{\lambda}\boldsymbol{0} = \frac{1}{\lambda}(\lambda v) = \left(\frac{1}{\lambda} \cdot \lambda\right) v = v.
            $$
    \end{enumerate}
\end{proof}

\begin{definition}
    Подмножество $U$ векторного пространства $V$ над полем $\mathcal{K}$ называется \textbf{подпространством} в $V$, если оно является векторным пространством над полем $\mathcal{K}$.
\end{definition}

\begin{theorem}[Критерий подпространства]
    Подмножество $U$ векторного пространства $V$ над полем $\mathcal{K}$ является подпространством тогда и только тогда, когда:
    \begin{enumerate}
        \item $\forall\!\:u_1, u_2 \in U\; (u_1 + u_2) \in U$
        \item $\forall\!\: u \in U, \lambda \in \mathcal{K}\; \lambda u \in U$
    \end{enumerate}
\end{theorem}

\begin{proof}
    $\Rightarrow$. Пусть $U$ --- подпространство в $V$. Тогда сумма двух его элементов лежит снова в $U$ и умножение любого элемента на число лежит в $U$.

    $\Leftarrow$. Наоборот, пусть выполнены условия теоремы. Тогда выполнение аксиом 1, 4-8 сразу следует из того, что они выполнены над $V$. Чтобы доказать справедливость аксиомы 2, нужно, чтобы $\boldsymbol{0} \in V$. Это так в силу того, что $U \ni 0 \cdot u = \boldsymbol{0}$. А для выполнения аксиомы 3 требуется, что $\forall\!\:u \in U\;(-u) \in U$. Это выполнется, т.\,к. $U \ni (-1) \cdot u = -u$.
\end{proof}


