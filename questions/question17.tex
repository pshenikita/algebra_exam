\section{Правая и левая обратные матрицы. Критерий существования. Обратная матрица, её единственность и критерий существования}

\begin{definition}
    Пусть $A$ --- матрица $m \times n$. Матрица $B$ размера $n \times m$ называется \textbf{левой обратной} к матрице $A$, если $BA = E$ и \textbf{правой обратной} к матрице $A$, если $AB = E$.
\end{definition}

\begin{definition}
    Пусть $A$ --- матрица $n \times n$. Матрица $B$ размера $n \times n$ называется \textbf{обратной} к матрице $A$, если $AB = BA = E$. Обозначается $A^{-1}$.
\end{definition}

\begin{theorem}[Критерий существования левой/правой обратной]
    Левая обратная к матрице $\underset{n \times m}{A}$ существует тогда и только тогда, когда $\rk A = n$. Если $m = n$ и левая обратная к матрице $A$ существует, то она единственна. Те же утверждения для правой обратной.
\end{theorem}

\begin{proof}
    Рассмотрим матрицу $BA$. Если $\rk A < n$, то $\rk AB \leqslant \rk A < n$, но $\rk E = n$, а значит, $AB \ne E$. Противоречие.

    Пусть теперь $\rk A = n$. Тогда строки $A$ --- полная система в $\mathcal{K}^n$, т.\,к. базис системы строк $A$ является базисом пространства $\mathcal{K}^n$ (в этом базисе все векторы линейно независимы и их количество правильное). Строки матрицы $BA$ --- это линейные комбинации строк $A$ с коэффициентами из строк $B$. Причём, т.\,к. система строк $A$ полна, то существуют коэффициенты (строки матрицы $B$), линейная комбинация строк $A$ с которыми даст строки единичной матрицы. Иными словами, существует такая матрица $B$, что $BA = E$.

    А если $m = n$ и $\rk A = n$, то вся система строк является базисом $\mathcal{K}^n$, а разложение по базису каждого вектора единственно. Поэтому существует единственный набор коэффициентов, линейная комбинация строк $A$ с которыми даст строки единичной матрицы. Иными словами, существует единственная матрица $B$, такая что $BA = E$. Для правой обратной аналогично.
\end{proof}

\begin{remark}
    Отсюда следует, что к матрице $A$ существует и левая, и правая обратные (причём, единственные) тогда и только тогда, когда $\rk A = m = n$, т.\,е. матрица $A$ квадратная и её ранг равен размеру. Теперь легко видеть, что если $B$ --- левая обратная к $A$, а $C$ --- правая, то $B = C$:
    $$
    B = B(AC) = (BA)C = C.
    $$

    Отсюда получаем слеудющую теорему:
\end{remark}

\begin{theorem}
    К матрице $A$ существует (и притом, только одна) обратная матрица тогда и только тогда, когда она квадратная и её ранг равен размеру. 
\end{theorem}

\begin{theorem}
    Одно из равенств $AB = E$ или $BA = E$ влечёт другое.
\end{theorem}

\begin{proof}
    Пусть выполнено $AB = E$. Тогда $B$ --- правая обратная к $A$. Так как правая обратная существует, то $\rk A = n$. А отсюда следует, что и левая обратная существует. Выше обсуждалось, что эти матрицы обязаны быть равны.
\end{proof}

\begin{statement}[Задача Антона Александровича]
    $A^{-1}$ (если существует) является многочленом от $A$.
\end{statement}

\begin{proof}
    Рассмотрим матрицы $A^{n^2}, A^{n^2 - 1}, \ldots, A^2, A, E$. Их всего $n^2 + 1$ штук. А размерность пространства $\underset{n \times n}{\mathrm{Mat}} \ni A$ равна $n^2$. Значит, эти матрицы линейно зависимы. Иными словами, существует их нетривиальная нулевая линейная комбинация. Пусть её коэффициенты --- $\lambda_1, \lambda_2, \ldots, \lambda_{n^2 + 1}$. Тогда рассмотрим многочлен
    $$
    \lambda_1 X^{n^2} + \lambda_2 X^{n^2 - 1} + \ldots + \lambda_{n^2}X + \lambda_{n^2 + 1}E.
    $$

    Из определения коэффициентов $\lambda_i$, $A$ является корнем этого многочлена. Значит, множество многочленов, аннулирующих $A$ непусто. Выберем из него многочлен наименьшей степени, пусть это
    $$
    f(X) = \mu_1X^k + \mu_2X^{k + 1} + \ldots + \mu_k X + \mu_{k + 1}E.
    $$

    Предположим, что $\mu_{k + 1} = 0$. Тогда имеем
    $$
    \mu_1 A^k + \mu_2 A^{k + 1} + \ldots + \mu_k A = 0.
    $$

    Домножим на $A^{-1}$ справа и получим многочлен степени $k - 1$, аннулирующий матрицу $A$. Противоречие с тем, что минимальная степень многочлена с таким свойством --- $k$. 

    Итак, $\mu_{k + 1} \ne 0$. Тогда домножим равенство
    $$
    \mu_1A^k + \mu_2A^{k + 1} + \ldots + \mu_k A + \mu_{k + 1}E = 0
    $$
    справа на $A^{-1}$ и выразим её:
    $$
    A^{-1} = \sum_{i = 1}^k\left(-\frac{\mu_i}{\mu_{k + 1}}\right)A^{k - i}.
    $$
\end{proof}

\begin{definition}
    Матрица называется \textbf{вырожденной} (\textbf{обратимой}), если у неё нет обратной.
\end{definition}

\begin{remark}
    В решении теоретических задач часто полезно помнить, что улучшенный ступенчатый вид матрицы --- это $E$ тогда и только тогда, когда она невырождена.
\end{remark}


