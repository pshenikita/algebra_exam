\section{Свойства ранга матрицы. Теорема Кронекера "---Капелли. Критерий определённости системы}

\begin{definition}
    Пусть $A$ --- матрица $m \times n$. Матрица $B = A^T$ размера $n \times m$ называется \textbf{транспонированной} к матрице $A$, если
    $$
    b_{ij} = a_{ij}.
    $$
\end{definition}

\begin{remark}
    Ясно, что $(A^T)^T = A$.
\end{remark}

\begin{theorem}[Свойства ранга матрицы]
    Пусть $A$ --- матрица $m \times n$. Тогда
    \begin{enumerate}[nolistsep]
        \item $\rk A \leqslant \min\{m, n\}$
        \item $\rk A = \rk A^T$
        \item Если к матрице добавить $k$ столбцов (строк), то ранг не уменьшится и увеличится не более, чем на $k$
    \end{enumerate}
\end{theorem}

\begin{proof}
    \begin{enumerate}[nolistsep]
        \item $\rk A$ равен рангу системы $S$ строк матрицы $A$, т.\,е. $\dim\langle S\rangle$. Система $S$, очевидно, полна в $\langle S\rangle$, а значит, из неё можно выделить базис, в нём будет не больше векторов, чем в исходной системе, т.\,е. не больше $m$. Аналогично доказывается $\rk A \leqslant n$.
        \item Ранг системы строк $A$ равен рангу системы столбцов $A^T$ (это одна и та же система).
        \item Базис системы строк (столбцов) матрицы $A$ --- это линейно независимая система. Значит, в новой матрице её можно дополнить до базиса. Другие строки матрицы $A$ в него уже не попадут (они образуют линейно зависимую систему со старым базисом), поэтому в него могут добавиться только новые строки.
    \end{enumerate}
\end{proof}

\begin{theorem}[Кронекер "---Капелли]
    СЛУ совместна тогда и только тогда, когда ранг матрицы её коэффициентов равен рангу расширенной матрицы её коэффициентов.
\end{theorem}

\begin{proof}
    Приведём расширенную матрицу коэффициентов к улучшенному ступенчатому виду (ранг и совместность СЛУ при этом не меняются).

    $\Rightarrow$. Если СЛУ совместна, то в ней нет экзотических уравнений, т.\,е. нет лидеров в последнем столбце матрицы. А, как мы доказывали ранее, базис системы столбцов состоит из тех столбцов, в которых есть лидеры. Поэтому последний столбец расширенной матрицы выражается линейно через базис системы столбцов матрицы коэффициентов. Значит, их ранги равны.

    $\Leftarrow$. Если ранг матрицы коэффициентов равен рангу расширенной матрицы, то последний столбец расширенной матрицы линейно выражается через базис системы столбцов матрицы коэффициентов (иначе нужно было бы добавить его в базис и ранг расширенной матрицы был бы больше на $1$). А значит, экзотических уравнений нет, и система совместна.
\end{proof}

\begin{theorem}[Критерий определённости системы]
    Совместная система линейных уравнений является определённой тогда и только тогда, когда ранг матрицы её коэффициентов равен числу неизвестных.
\end{theorem}

\begin{proof}
    Ранг матрицы коэффициентов равен количеству ненулевых строк в ступенчатом виде, т.\,е. количеству главных переменных. Условие теоремы равносильно тому , что главных переменных столько же, сколько всех переменных, т.\,е. тому, что система определена.
\end{proof}

