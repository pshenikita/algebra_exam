\section{Ранг системы векторов и размерность векторного пространства. Связь ранга системы и размерности линейной оболочки. Строчный, столбцовый и ступенчатый ранги матрицы, их совпадение}

Определение размерности векторного пространства было дано ранее (определение 8.3).

\begin{definition}
    \textbf{Рангом системы векторов} называется размерность её линейной оболочки.
\end{definition}

Ранее доказывалось (теорема 8.2), что это определение корректно.

\begin{definition}
    Системы векторов $\{a_1, \ldots, a_n\}$ и $\{b_1, \ldots, b_m\}$ называются \textbf{эквивалентными}, если каждый из векторов $b_j$ линейно выражается через $a_1, \ldots, a_n$ и, наоборот, каждый из векторов $a_i$ линейно выражается через $b_1, \ldots, b_m$.
\end{definition}

\begin{lemma}
    Ранги эквивалентных систем векторов равны.
\end{lemma}

\begin{proof}
    Каждый вектор из $\{a_1, \ldots, a_n\}$ линейно выражается через векторы из $\{b_1, \ldots, b_m\}$, значит, линейные комбинации вида $\sum_i\lambda_ia_i$ тоже линейно выражаются через векторы из $\{b_1, \ldots, b_m\}$. Отсюда $\langle a_1, \ldots, a_n\rangle \subseteq \langle b_1, \ldots, b_m\rangle$. Аналогично доказывается обратное, а отсюда
    $$
    \langle a_1, \ldots, a_n\rangle = \langle b_1, \ldots, b_m\rangle
    $$

    А из равенства линейных оболочек очевидно вытекает равенство их размерностей.
\end{proof}

\begin{definition}
    \textbf{Строчным рангом} матрицы $A$ называется ранг системы её строк, \textbf{столбцовым рангом} --- ранг системы её столбцов.
\end{definition}

\begin{theorem}
    Строчный, столбцовый и ступенчатый ранги совпадают.
\end{theorem}

Сначала докажем две вспомогательные леммы.

\begin{lemma}
    Каждый вид ранга не меняется при элементарных преобразованиях строк.
\end{lemma}

\begin{proof}
    Докажем утверждение отдельно для каждого вида ранга:
    \begin{itemize}
        \item \textbf{Строчный ранг}. Пусть мы пришли элементарным преобразованием от матрицы $A$ к матрице $A^\prime$. Очевидно, что системы строк этих матриц эквивалентны. А потому их ранги равны.
        \item \textbf{Столбцовый ранг}. Докажем, что линейные зависимости между столбцами матрицы не меняются при элементарных преобразованиях строк. Линейная зависимость между какими-то столбцами матрицы может пониматься как линейная зависимость между всеми её столбцами, в которую остальные столбца входят с нулевыми коэффициентами. Следовательно, если какие-то столбцы матрицы линейно зависимы, то они останутся линейно зависимыми после любых элементарных преобразований строк. Так как элементарные преобразований обратимы, то и наоборот: если какие-то столбцы линейно независимы, то они и останутся линейно независимыми. Значит, если какие-то столбцы матрицы составляют максимальную линейно независимую систему её столбцов, то после любых элементарных преобразований строк столбцы с теми же номерами буду составлять максимальную линейно независимую систему столбцов полученной матрицы, и поэтому ранг матрицы не изменится.
        \item \textbf{Ступенчатый ранг}. Утверждение для него сразу следует из единственности улучшенного ступенчатого вида.
    \end{itemize}
\end{proof}

\begin{lemma}
    В ступенчатом виде все виды ранга совпадают.
\end{lemma}

\begin{proof}
    Докажем, что ненулевые строки образуют базис системы строк (т.\,е. просто, что они линейной независимы). Пусть лидеры имеют номера $j_1, \ldots, j_r$. Уберём из системы ненулевых строк строку с номером $i$. Тогда из улучшенного ступенчатого вида матрицы, у оставшихся векторов $j_i$-ая координата равна $0$, а значит, никакой их линейной комбинацией нельзя получить $i$-ую строку. Таким образом, система ненулевых строк --- линейно независима и максимальна по включению (остальные строки нулевые), а значит, является базисом.
    
    Теперь докажем, что столбцы с номерами $j_1, j_2, \ldots, j_r$ образуют базис системы столбцов. Действительно, эта система полна и минимальна по включению, то есть, базис.

    Ступенчатый ранг матрицы равен количеству её ненулевых строк, что, в свою очередь, равно строчному рангу. Также, ступенчатый ранг равен количеству главных переменных, а это, как мы показали, и есть столбцовый ранг.
\end{proof}

Теперь совсем легко доказать теорему 9.1:

\begin{proof}
    Приведём матрицу к улучшенному ступенчатому виду, каждый вид ранга при этом не изменится. А в улучшенному ступенчатому виде все ранги совпадают.
\end{proof}

