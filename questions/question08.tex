\section{Базис системы векторов: эквивалентность 4 определений. Стандартный базис в $\R^n$. Дополнение линейно независимой системы до базиса. Выбор базиса из полной системы}

\begin{definition}
    Подсистема $L \subseteq S$ называется \textbf{полной}, если любой вектор из $S$ выражается как линейная комбинация векторов из $L$.
\end{definition}

\begin{statement}
    $L \subseteq S$ полна, если $\langle L\rangle = \langle S\rangle$.
\end{statement}

\begin{proof}
    Т.\,к. любой вектор $s_i \in S$ является линейной комбинацией векторов из $L$ (условие $\langle S\rangle \subseteq \langle L\rangle$), то линейная комбинация $\sum_i \lambda_is_i$ также является линейной комбинацией векторов из $L$ (условие $\langle S\rangle \supseteq \langle L\rangle$). Отсюда заключаем, что $\langle L\rangle = \langle S\rangle$.
\end{proof}

\begin{theorem}
    Пусть $B$ --- подсистема векторов в системе $S$. Тогда следующие условия эквивалентны:
    \begin{enumerate}[nolistsep]
        \item $B$ --- максимальная по включению линейно независимая подсистема векторов из $S$;
        \item $B$ --- полная линейно независимая подсистема векторов из $S$;
        \item $B$ --- минимальная по включению полная подсистема векторов из $S$
        \item Каждый вектор из $S$ линейно выражается через векторы из $B$, причём единственным образом.
    \end{enumerate}
\end{theorem}

\begin{definition}
    Подсистема подмножества векторного пространства, удовлетворяющая условиям предыдущей теоремы, называется \textbf{базисом} этого подмножества.
\end{definition}

\begin{proof}
    Докажем, что из каждого следующего пункта следует предыдущий:

    $1 \Rightarrow 2$. Пусть $B$ --- базис. Тогда $B$ линейно независима, нужно лишь доказать, что $B$ --- полная подсистема. Возьмём $v \in S \setminus B$. Тогда система $B \cup \{v\}$ линейно зависима по определению базиса. По свойству линейной зависимости, вектор $v$ является линейной комбинацией векторов из $v$. Значит, система $B$ полна по определению.

    $2 \Rightarrow 3$. По условию система $B$ полная. Допустим, что для некоторого $v \in B$ система $B \setminus \{v\}$ также является полной. То есть, найдутся такие $e_1, \ldots, e_k \in B$ и $\lambda_1, \ldots, \lambda_k \in \mathcal{K}$, что $v = \sum_i^k\lambda_ie_i$. Но тогда $B$ линейно зависима (по критерию линейной зависимости). Это же доказательство работает и для конечномерных пространств, достаточно убрать верхний предел суммирования.

    $3 \Rightarrow 4$. Пусть $B$ --- минимальная по включению полная подсистема. Тогда по определению полной подсистемы каждый вектор из $S$ линейно выражается через $B$. Допустим, что есть вектор $v \in S$, такой что он выражается двумя различными способами
    $$
    v = \sum_{i = 1}^n\lambda_ie_i = \sum_{i = 1}^n\mu_ie_i,\quad e_i \in B,\;\lambda_i,\mu_i \in \mathcal{K}.
    $$
    Получаем $\displaystyle 0 = \sum_{i = 1}^n(\lambda_i - \mu_i)e_i$ --- нетривиальная линейная комбинация. По критерию линейно зависимости, найдётся $e_j$, который выражается через остальные. Тогда система $B \setminus \{e_j\}$ полная. Получили противоречие с её минимальностью.

    $4 \rightarrow 1$. Допустим, что $B$ линейно зависима. Тогда по критерию линейной зависимости существует $v \in B$, такой что
    $$
    v = \sum_{i = 1}^n\lambda_ie_i,\quad e_i \in B \setminus \{v\}, \lambda_i \in \mathcal{K}.
    $$
    Получаем два линейных выражения $v$ через $B$ (второе имеет вид $v = 1 \cdot v$). Противоречие, значит, $B$ линейно независима. Докажем теперь, что она максимальна по включению. Рассмотрим $u \in S \setminus B$. По условию, вектор $u$ линейно выражается через $B$. Тогда по критерию линейной зависимости $B \cup \{u\}$ линейно зависима.
\end{proof}

\begin{definition}
    \textbf{Размерностью} конечномерного векторного пространства $V$ называется число векторов $\dim V$ в его базисе.
\end{definition}

\begin{theorem}
    Определение выше корректно. Иными словами, все базисы конечномерного векторного пространства $V$ содержат одно и то же число векторов.
\end{theorem}

\begin{proof}
    Если бы в пространстве $V$ существовали два базиса из разного числа векторов, то тот из них, в котором больше векторов, был бы линейно зависим по основной лемме.
\end{proof}

\begin{statement}[задача из Винберга]
    Число базисов $n$-мерного векторного пространства над полем из $q$ элементов равно
    $$
    \prod_{k = 0}^{n - 1}(q^n - q^k).
    $$
\end{statement}

\begin{proof}
    Всего у нас в таком пространстве $q^n$ векторов. Нам нужно выбрать среди них $q$ линейно независимых, они и будут базисом. Для выбора первого вектора есть $q^n - 1$ возможностей, нам подходят все, кроме нулевого. Для выбора второго --- $q^n - q$, нам подходят все, кроме коллинеарных первому, а таких $q$. И так далее посчитаем искомую велечину для $k - 1$ векторов. Теперь, чтобы посчитать количество способов выбрать $k$-ый линейно независимых вектор, посчитаем количество способов выбрать $k$-ый линейно зависимый вектор и вычтем из общего количества векторов. Итак, мы знаем, что система из $k - 1$ выбранных векторов линейно независима, а из $k$ --- линейно зависима. Отсюда (по третьему свойству линейной зависимости) $k$-ый вектор выражается через остальные единственным образом. А коэффициентов в этом выражении можно выбрать $q^{k - 1}$ штук (по $q$ для каждого из $k - 1$ векторов). Итак, количество возможностей выбрать $k$-ый вектор линейно независимым равняется $q^n - q^{k - 1}$. Теперь считаем ответ по правилу произведения (эти выборы не зависят друг от друга) и получаем требуемое.
\end{proof}

\begin{definition}
    \textbf{Стандартным базисом} в $\R^n$ называется базис, состоящий из строк единичной матрицы размера $n \times n$.
\end{definition}

\begin{remark}
    Можете посмотреть интересные задачи 5-9 в Винберге, мне лень оформлять их решения.
\end{remark}

\begin{theorem}
    В конечномерном пространстве любую линейно независимую систему можно дополнить до базиса.
\end{theorem}

\begin{proof}
    Пусть $\{e_1, \ldots, e_k\}$ --- конечная подсистема в $V$. Тогда, если эта система максимальная по включению, то она базис. Иначе существует $e_{k + 1} \in V$ такой, что система $\{e_1, \ldots, e_k, e_{k + 1}\}$ линейно независима. Продолжая процесс далее, получим базис (в силу конечномерности пространства $V$).
\end{proof}

\begin{theorem}
    Из любой конечной полной подсистемы $S$ (в которой есть ненулевой вектор) можно выбрать базис $S$.
\end{theorem}

\begin{proof}
    Воспользуемся тем, что базис --- минимальная по включению полная подсистема. Пусть нам дана полная конечная подсистема $\{v_1, \ldots, v_n\}$. Если её можно уменьшить (выкинуть один из векторов) так, чтобы система осталась полной, то сделаем это. Иначе она уже базис. А т.\,к. система конечна, мы не можем бесконечно её уменьшать. Поэтому через конечное число шагов дойдём до непустой (т.\,к. в $S$ был ненулевой вектор) минимальной по включению полной системы, т.\,е. до базиса.
\end{proof}


