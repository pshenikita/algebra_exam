\section{Системы линейный уравнений. Матрица коэффициентов и расширенная матрица коэффициентов. Элементарные преобразования. Эквивалентные системы. Элементарные преобразования переводят СЛУ в эквивалентную}

\begin{definition}
    \textbf{Матрицей} размера $m \times n$ над полем $\mathcal{K}$ называется прямоугольная таблица из элементов поля $\mathcal{K}$, имеющая $m$ \textbf{строк} и $n$ \textbf{столбцов}:
    $$
    \begin{pmatrix}
        a_{11} & a_{12} & \ldots & a_{1n}\\
        a_{21} & a_{22} & \ldots & a_{2n}\\
        \vdots & \vdots & \ddots & \vdots\\
        a_{m1} & a_{m2} & \ldots & a_{mn}\\
    \end{pmatrix}
    $$
\end{definition}

\begin{definition}
    \textbf{Линейным уравнением} с неизвестными $x_1, x_2, \ldots, x_n$ над полем $\mathcal{K}$ называется уравнение вида
    $$
    a_1x_1 + a_2x_2 + \ldots + a_nx_n = b,
    $$
    где \textbf{коэффициенты} $a_1, a_2, \ldots, a_n$ и \textbf{свободный член} $b$ являются элементами поля $\mathcal{K}$.
\end{definition}

Система $m$ линейных уравнений с $n$ неизвестными в общем виде пишется так:
$$
\begin{cases}
    a_{11}x_1 + a_{12}x_2 + \ldots + a_{1n}x_n = b_1,\\
    a_{21}x_1 + a_{22}x_2 + \ldots + a_{2n}x_n = b_2,\\
    \ldots\\
    a_{m1}x_1 + a_{m2}x_2 + \ldots + a_{mn}x_n = b_m,\\
\end{cases}\eqno(\ast)
$$

\begin{definition}
    Матрица
    $$
    \mathcal{A} \vcentcolon = 
    \begin{pmatrix}
        a_{11} & a_{12} & \ldots & a_{1n}\\
        a_{21} & a_{22} & \ldots & a_{2n}\\
        \vdots & \vdots & \ddots & \vdots\\
        a_{m1} & a_{m2} & \ldots & a_{mn}\\
    \end{pmatrix}
    $$
    называется \textbf{матрицей коэффициентов}, а матрица 
    $$
    \widetilde{\mathcal{A}} \vcentcolon =
    \left(
    \begin{array}{cccc | c}
        a_{11} & a_{12} & \ldots & a_{1n} & b_1\\
        a_{21} & a_{22} & \ldots & a_{2n} & b_2\\
        \vdots & \vdots & \ddots & \vdots & \vdots\\
        a_{m1} & a_{m2} & \ldots & a_{mn} & b_m\\
    \end{array}
    \right)
    $$
    --- \textbf{расширенной матрицей коэффициентов} системы $(\ast)$.
\end{definition}

\begin{definition}
    \textbf{Решение системы} --- это элемент поля $\mathcal{K}^n$, который при подстановке вместо $(x_1, \ldots, x_n)$ обращает каждое уравнение в верное равенство. Система уравнений называется \textbf{совместной}, если она имеет хотя бы одно решение, и \textbf{несовместной} в противном случае. Совместная система называется \textbf{определённой}, если её решение единственно.
\end{definition}

\begin{definition}
    \textbf{Элементарными преобразованиями} системы линейных уравнений называются преобразования следующих трёх типов:
    \begin{enumerate}[noitemsep]
        \item прибавление к одному уравнению другого, умноженного на число;
        \item перестановка двух уравнений;
        \item умножение одного уравнения на число, отличное от нуля.
    \end{enumerate}

    \textbf{Элементарными преобразованиями строк} матрицы называются преобразования следующих трёх типов:
    \begin{enumerate}[noitemsep]
        \item прибавления к одной строке другой, умноженной на число;
        \item перестановка двух строк;
        \item умножение одной строки на число, отличное от нуля.
    \end{enumerate}
\end{definition}

\begin{definition}
    Две СЛУ называются \textbf{эквивалентными}, если множества их решений совпадают. Обозначение для расширенных матриц: $\widetilde{\mathcal{A}} \sim \widetilde{\mathcal{B}}$.
\end{definition}

\begin{theorem}
    Если одну систему можно перевести в другую элементарными преобразованиями, то они \mbox{эквивалентны}.
\end{theorem}

Докажем сначала вспомогательное утверждение:

\begin{lemma}
    Преобразование, обратное к элементарному, тоже элементарное.
\end{lemma}

\begin{proof}
    Докажем для каждого преобразования:
    \begin{enumerate}
        \item (<<прибавить $i$-ое уравнение к $j$-му с коэффициентом $\lambda$>>)$^{-1} = {}$ (прибавить $i$-ое к $j$-му с коэффициентом $-\lambda$>>)
        \item Преобразование второго типа обратно к самому себе
        \item $(\text{<<умножить $i$-ое уравнение на $c \ne 0$>>})^{-1} = (\text{<<умножить $i$-ое уравнение на $c^{-1} \ne 0$>>})$
    \end{enumerate}
\end{proof}

Теперь докажем теорему:

\begin{proof}
    Докажем, что каждое решение первой системы является решением второй. Это очевидно для преобразований второго и третьего типа, докажем и для первого:
    $$
    (a_{j1} + \lambda a_{i1})x_1 + \ldots + (a_{jn} + \lambda a_{in})x_n = \underbrace{a_{j1}x_1 + \ldots + a_{jn}x_n}_{{} = 0} + \lambda\underbrace{(a_{i1}x_1 + \ldots + a_{in}x_n)}_{{} = 0} = 0.
    $$

    Далее, из леммы 1.1 обратное к элементарному преобразвоание тоже элементарное. Поэтому всякое решение второй системы является решением первой. Значит, множества решений этих систем совпадают, т.\,е. они эквивалентны.
\end{proof}

\begin{orangebox}
    Обратное утверждение неверно --- если СЛУ эквивалентны, то они не всегда переводятся друг в друга элементарными преобразованиями. Контрпример --- две системы с пустыми множествами решений:
    $$
    A = 
    \begin{pmatrix}
        0 & 1 & \bord & 1\\
        0 & 0 & \bord & 1
    \end{pmatrix} \equiv
    B = 
    \begin{pmatrix}
        1 & 0 & \bord & 1\\
        0 & 0 & \bord & 1
    \end{pmatrix}.
    $$

    Никаким элементарным преобразованием строк нельзя занулить первый столбец матрицы $\mathcal{B}$.
\end{orangebox}

