\section{Факториальность кольца многочленов над факториальным кольцом}

\begin{theorem}
    Пусть $A$ --- факториальное кольцо. Тогда кольцо $A[x]$ также факториально.
\end{theorem}

\begin{proof}
    Докажем, что неприводимые элементы в $A[x]$ --- это неприводимые элементы $p \in A$ и примитивные многочлены $f \in A[x]$, которые неприводимы в $(\mathrm{Quot}\,A)[x]$. В самом деле, если многочлен степени ноль, то он приводим тогда и только тогда, когда разлагается на 2 неприводимых многочлена степени ноль, то есть приводим в $A$. Многочлен положительной степени может разлагаться либо на произведение необратимой константы и многочлена, либо на произведение двух многочленов положительной степени. Первое возможно тогда и только тогда, когда многочлен не является примитивным. Второе даёт разложение $f$ в произведение над $(\mathrm{Quot}\,A)[x]$. Таким образом, примитивный многочлен, который неприводим в $(\mathrm{Quot}\,A)[x]$ является неприводимым в $A[x]$. Осталось доказать, что примитивный многочлен $f$, приводимый в $(\mathrm{Quot}\,A)[x]$ приводим и в $A[x]$. Пусть $f = gh$, где $g, h \in (\mathrm{Quot}\,A)[x]$. По лемме существуют элементы $r_g$ и $r_h$ из $(\mathrm{Quot}\,A)[x]$ такие, что $r_gg$ и $r_hh$ примитивны. Имеем $f = (r_gr_h)^{-1}(r_ggr_hh)$. По лемме Гаусса $r_ggr_hh$ примитивен. По замечанию $r_gr_h$ обратим в $A$. Значит, $f = ((r_gr_h)^{-1}r_gg)(r_hh)$ --- разложение на необратимые множители в $A[x]$.

    \textbf{Существование}. Пусть теперь $f \in A[x]$ --- произвольный многочлен. Представим его в виде $f = rg$, где $g$ --- примитивный многочлен, и $r \in A$. Тогда $r$ можно разложить на неприводимые в $A$, а $g$ можно разложить на неприводимые в $(\mathrm{Quot}\,A)[x]$. При этом разложение $g$ можно сделать разложением на примитивные неприводимые. Таким образом мы можем разложить любой элемента $f \in A[x]$ на неприводимые в $A[x]$.

    \textbf{Единственность}. Пусть $f = p_1\ldots p_m g_1\ldots g_k = q_1\ldots q_u h_1\ldots h_v$, где $p_i$, $q_j$ --- неприводимые в $A$, а $g_i$, $h_j$ --- примитивные неприводимые в $\mathrm{Quot}\,A$. Заметим, что НОД всех коэффициентов $f$ равен $p_1\ldots p_m$, т.\,к. $g_1\ldots g_k$ --- примитивный многочлен. Аналогично, этот же НОД равен $q_1\ldots q_u$. Значит, произведения $p_1\ldots p_m$ и $q_1\ldots q_u$ ассоциированы. Т.\,к. $A$ факториально, $u = m$ и $p_i$ и $q_j$ попарно ассоциированы. Кольцо $(\mathrm{Quot}\,A)[x]$ факториально, значит два разложения $f$ совпадают с точностью до перестановки и ассоциированности множителей. Отсюда $k = v$ и $g_i = r_ih_i$ для некоторого $r_i \in (\mathrm{Quot}\,A)[x]$. Как доказано ранее из того, что $g_i$ и $h_i$ примитивны следует, что $r_i$ --- обратимый элемент $A$.
\end{proof}

