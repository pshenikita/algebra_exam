\section{Фальшивое разложение определителя по строке/столбцу. Явная формула для обратной матрицы}

\begin{theorem}[Фальшивое разложение определителя по строке/столбцу]
    $$
    \sum_{j = 1}^na_{ij}A_{kj} = 0\text{{} при $k \ne i$},\hspace{2cm}\sum_{i = 1}^n a_{ij}A_{ik} = 0\text{{} при $k \ne j$}.
    $$
\end{theorem}

\begin{proof}
    Рассмотрим матрицу $A^\prime = (a^\prime_{ij})$, полученную из $A$ заменой $k$-ой строки на $i$-ую. В матрице $A^\prime$ две одинаковые строки, следовательно, $\det A^\prime = 0$. Разложим определитель матрицы $A^\prime$ по $k$-ой строке. Получим
    $$
    0 = \det A^\prime = \sum_{j = 1}^n a^\prime_{kj}A^\prime_{kj} = \sum_{j = 1}^n a_{ij}A_{kj}.
    $$

    Аналогично получается фальшивое разложение по столбцу.
\end{proof}

\begin{theorem}[Явная формула для обратной матрицы]
    Пусть $A = (a_{ij})$ --- невырожденная квадртная матрица. Тогда
    $$
    A^{-1} = \frac{1}{\det A}
    \underbrace{
    \begin{pmatrix}
        A_{11} & A_{21} & \ldots & A_{n1}\\
        A_{12} & A_{22} & \ldots & A_{n2}\\
        \vdots & \vdots & \ddots & \vdots\\
        A_{1n} & A_{2n} & \ldots & A_{nn}\\
    \end{pmatrix}}_{\widehat{A}^T}.
    $$
\end{theorem}

\begin{proof}
    Докажем равносильное требуемому равенство:
    $$
    A \cdot \widehat{A}^T = (\det A) \cdot E.
    $$

    Перемножим по определению:
    $$
    (A\widehat{A}^T)_{ii} = \sum_{t = 1}^n a_{it}\widehat{a}_{ti} = \sum_{t = 1}^n a_{it}A_{it} = \det A,\hspace{2cm}
    (A\widehat{A}^T)_{ij} = \sum_{t = 1}^n a_{ij}\widehat{a}_{tj} = \sum_{t = 1}^n a_{it}A_{jt} = 0.
    $$

    Первое равенство выполнено в силу теорему о разложении определителя по $i$-ой строке, а второе --- по теореме о фальшивом разложении определителя по $i$-ой строке.
\end{proof}

\begin{remark}
    Матрица $\widehat{A}$ в обозначениях предыдущей теоремы иногда называют \textbf{присоединённой матрицей}, но Сергей Александрович присоединённой матрицей называется $\widehat{A}^T$, а Винберг этого определения вообще не вводит.
\end{remark}

\begin{statement}[Задача Антона Александровича]
    Пусть $A$ --- невырожденная целочисленная квадратная матрица. Матрица $A^{-1}$ является целочисленной тогда и только тогда, когда $\det A = \pm 1$.
\end{statement}

\begin{proof}
    $\Rightarrow$. Заметим, что
    $$
    \det A \cdot \det A^{-1} = \det AA^{-1} = \det E = 1.
    $$
    Причём, матрица $A^{-1}$ целочисленная, а поэтому и $\det A^{-1}$ (как и $\det A$) является целым числом (видно из формулы по определению). Поэтому числа $\det A$ и $\det A^{-1}$ целые и взаимно обратные. Значит, они \mbox{равны $\pm 1$}.

    $\Leftarrow$. Видно из явной формулы для обратной матрицы.
\end{proof}

\begin{statement}[Задача Антона Александровича]
    Доказать, что в матрице $A$ любой минор ранга $\rk A$, образуемый пересечением линейной независимых строк и линейно независимых столбцов, отличен от нуля.
\end{statement}

\begin{proof}
    Во-первых, можно перестановкой строк (элементарными преобразованиями 2 типа) передвинуть наш минор в левый верхний угол матрицы. Теперь, строки, входящие в минор, образуют базис системы строк матрицы $A$, а поэтому можно, вычитая их линейные комбинации из остальных строк (элементраные преобразования 1 типа) занулить все строки, начиная с $\rk A + 1$. Ранее мы уже доказывали, что при этом линейные зависимости между столбцами не меняются (см. доказательство леммы 9.2), а потому первые $\rk A$ столбцов образуют базис системы столбцов матрицы $A$. Поступаем аналогично, вычитая эти столбцы из остальных. После этих действий, в нашей матрице останется только выбранный нами минор, а остальные элементы занулятся. При этом, мы совершали элементарные преобразования, поэтому ранг не изменился. Если предположить, что выбранный нами минор нулевой, но его ранг меньше $\rk A$, но тогда и ранг оставшейся матрицы меньше $\rk A$. Противоречие.
\end{proof}

\begin{remark}
    Эти задачи есть также в Винберге, но впервые я их узнал от Антона Александровича.
\end{remark}

