\section{Элементарные матрицы. Умножение на элементарные матрицы слева и справа. Матрица, обратная к произведению. Обратная к транспонированной матрице}

\begin{definition}
    Непосредственным вычислением проверяется, что элементарные преобразования строк какой-либо матрицы $A$ равносильные её умножению слева на \textbf{элементарные матрицы} следующих трёх типов:
    $$\footnotesize
    \begin{array}{c}
    E + cE_{ij} = 
    \begin{pmatrix}
        1 & {} & {} & {} & {} & {} & {}\\
        {} & \ddots & {} & {} & {} & {} & {}\\
        {} & {} & 1 & \cdots & c & {} & {}\\
        {} & {} & {} & \ddots & \vdots & {} & {}\\
        {} & {} & {} & {} & 1 & {} & {}\\
        {} & {} & {} & {} & {} & \ddots & {}\\
        {} & {} & {} & {} & {} & {} & 1
    \end{pmatrix},\quad
    P_{ij} = 
    \begin{pmatrix}
        1 & {} & {} & {} & {} & {} & {}\\
        {} & \ddots & {} & {} & {} & {} & {}\\
        {} & {} & 0 & \cdots & 1 & {} & {}\\
        {} & {} & \vdots & \ddots & \vdots & {} & {}\\
        {} & {} & 1 & \cdots & 0 & {} & {}\\
        {} & {} & {} & {} & {} & \ddots & {}\\
        {} & {} & {} & {} & {} & {} & 1
    \end{pmatrix},\quad
    Q_i(c) = 
    \begin{pmatrix}
        1 & {} & {} & {} & {}\\ 
        {} & \ddots & {} & {} & {}\\
        {} & {} & c & {} & {}\\ 
        {} & {} & {} & \ddots & {}\\
        {} & {} & {} & {} & 1
    \end{pmatrix},
    \end{array}
    $$

    где $i \ne j$ и $c \ne 0$, а все элементы этих матриц, не выписанные явно, такие же, как у единичной матрицы.
\end{definition}

Умножение на элементарные матрицы справа дают нам элементарные преобразования столбцов. Так, умножение матрицы $A$ слева на $E + cE_{ij}$ ($i \ne j$) приводит к тому, что к $i$-ой строке прибавляется $j$-ая строка, умноженная на $c$. А если умножить ту же матрицу на $A$ справа, то к $j$-му столбцу прибавляется $i$-ый столбец, умноженный на $c$.

\begin{remark}
    Из Винберга. Метод Гаусса в матричной интерпретации состоит в последовательном умножении уравнения
    $$
    AX = B
    $$
    слева на элементарные матрицы, имеющем целью приведения матрицы $A$ к улучшенному ступенчатому виду. Используя вместо элементарных матриц какие-либо другие матрицы, можно получить другие методы решения систем линейных уравнений, которые, быть может, не столь просты в теоретическом отношении, но, скажем, более надёжны при приближённых вычислениях (в случае $\mathcal{K} = \R$). Таков, например, метод вращений, при котором в качестве элементарных берутся матрицы вида
    $$
    \begin{pmatrix}
        1 & {} & {} & {} & {} & {} & {}\\
        {} & \ddots & {} & {} & {} & {} & {}\\
        {} & {} & \cos\alpha & \cdots & -\sin\alpha & {} & {}\\
        {} & {} & \vdots & \ddots & \vdots & {} & {}\\
        {} & {} & \sin\alpha & \cdots & \cos\alpha & {} & {}\\
        {} & {} & {} & {} & {} & \ddots & {}\\
        {} & {} & {} & {} & {} & {} & 1
    \end{pmatrix}.
    $$
\end{remark}

\begin{statement}
    \begin{enumerate}[nolistsep]
        \item $(AB)^{-1} = B^{-1}A^{-1}$;
        \item $(A^T)^{-1} = (A^{-1})^T$.
    \end{enumerate}
\end{statement}

\begin{proof}
    Нужно просто перемножить и удостовериться, что получается $E$:
    \begin{enumerate}
        \item $(B^{-1}A^{-1})(AB) = B^{-1}(A^{-1}A)B = B^{-1}B = E$;
        \item $(A^{-1})^TA^T = (AA^{-1})^T = E^T = E$.
    \end{enumerate}

    Как следствие, если $A$ и $B$ обратимы, то и $AB$ тоже.
\end{proof}


