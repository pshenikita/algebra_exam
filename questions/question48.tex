\section{Симметрические многочлены. Основная теорема о симметрических многочленах}

\begin{definition}
    Многочлен $f \in \mathbb{F}[x_1, \ldots, x_n]$ называется \textbf{симметрическим}, если он не изменяется ни при каких перестановках переменных.
\end{definition}

\begin{definition}
    Следующие симметрические многочлены называются \textbf{элементарными}:
    $$
    \begin{array}{l}
        \sigma_1 = x_1 + x_2 + \ldots + x_n,\\
        \sigma_2 = x_1x_2 + x_2x_3 + \ldots + x_{n - 1}x_n,\\
        \hspace{1.5cm}\vdots\\
        \sigma_k = \sum\limits_{i_1 < \ldots < i_k} x_{i_1}\ldots x_{i_k},\\
        \hspace{1.5cm}\vdots\\
        \sigma_n = x_1x_2\ldots x_n.
    \end{array}
    $$
\end{definition}

Нетривиальный пример симметрического многочлена --- это $V^2(x_1, \ldots, x_n)$. Действительно, при перестановках $x_1, \ldots, x_n$ определитель Вандермонда может лишь поменять знак, а значит, его квадрат при любых перестановках переменных не меняется. То есть, они совпадают как функции, а из этого следует и совпадение многочленов (см. теорему 43.1). 

Очевидно, что сумма и произведение симметрических многочленов, а также произведение симметрического многочлена на число являются симметрическими многочленами. Иными словами, симметрические многочлены образуют подалгебру в алгебре всех многочленов.

\begin{theorem}
    Всякий симметрический многочлен единственным образом представляется в виде многочлена от элементарных симметрических многочленов.
\end{theorem}

Перед тем, как её доказывать, докажем два вспомогательных утверждения.

\begin{lemma}
    Пусть $u = x_1^{k_1}x_2^{k_2}\ldots x_n^{k_n}$ --- старший член симметрического многочлена $f$.
\end{lemma}

\begin{proof}
    Предположим, что $k_i < k_{i + 1}$ для некоторого $i$. В силу симметричности, многочлен $f$ помимо члена $u$ должен содержать и член
    $$
    u^\ast = x_1^{k_1}x_2^{k_2}\ldots x_i^{k_{i + 1}}x_{i + 1}^{k_i}\ldots x_n^{k_n},\eqno(\star)
    $$
    ведь один получается из другого транспозицией $[i, i + 1]$. Однако $u^\ast \succ u$, поэтому $u$ не может быть старшим членом. Противоречие.
\end{proof}

\begin{lemma}
    Для любого одночлена $u = x_1^{k_1}x_2^{k_2}\ldots x_n^{k_n}$, показатели которого удовлетворяют неравенствам $(\star)$, существуют такие неотрицательные числа $\ell_1, \ell_2, \ldots, \ell_n$, что старший член многочлена $\sigma_1^{\ell_1}\sigma_2^{\ell_2}\ldots \sigma_n^{\ell_n}$ совпадает с $u$. Числа $\ell_1, \ell_2, \ldots, \ell_n$ определены этим условием однозначно.
\end{lemma}

\begin{proof}
    Старший член многочлена $\sigma_k$ равен $x_1x_2\ldots x_k$. В силу леммы о старшем члене старший член многочлена $\sigma_1^{\ell_1}\sigma_2^{\ell_2}\ldots \sigma_n^{\ell_n}$ равен
    $$
    x_1^{\ell_1}(x_1x_2)^{\ell_2}\ldots(x_1x_2\ldots x_n)^{\ell_n} = x_1^{\ell_1 + \ell_2 + \ldots + \ell_n}x_2^{\ell_2 + \ldots + \ell_n}\ldots x_n^{\ell_n}.
    $$
    Приравнивая его одночлену $u$, получаем систему линейных уравнений
    $$
    \left\{
        \begin{array}{r}
            \ell_1 + \ell_2 + \ldots + \ell_n = k_1,\\
            \ell_2 + \ldots + \ell_n = k_2,\\
            \ldots\\
            \ell_n = k_n,
        \end{array}
    \right.
    $$
    которая, очевидно, имеет единственное решение
    $$
    \ell_i = k_i - k_{i + 1}\ (i = 1, 2, \ldots, n - 1),\quad \ell_n = k_n.
    $$

    Из условия леммы следует, что определённые таким образом числа $\ell_1, \ell_2, \ldots, \ell_n$ неотрицательны.
\end{proof}

Теперь докажем теорему 48.1.

\begin{proof}
    Пусть $f \in \mathbb{F}[x_1, \ldots, x_n]$ --- симметрический многочлен. Нам нужно найти такой многочлен $F \in \mathbb{F}[X_1, \ldots, X_n]$, что
    $$
    F(\sigma_1, \sigma_2, \ldots, \sigma_n) = f.
    $$
    Если $f = 0$, то можно взять $F = 0$. В противном случае пусть $u_1 = ax_1^{k_1}x_2^{k_2}\ldots x_n^{k_n}$ --- старший член многочлена $f$. По лемме 48.1 выполняются неравенства $(\star)$, по лемме 48.2 существует такой одночлен $F_1 \in \mathbb{F}[X_1, \ldots, X_n]$, что старший член многочлена $F_1(\sigma_1, \sigma_2, \sigma_n)$ был равен $u_1$. Рассмотрим симметрический многочлен
    $$
    f_1 = f - F_1(\sigma_1, \sigma_2, \ldots, \sigma_n).
    $$

    Если $f_1 = 0$, то можно взять $F = F_1$. В противном случае повторяем процесс, строя многочлены $f_2, f_3, \ldots$ Процесс обязательно конечный, т.\,к. степени многочленов $f_k$ убывают. А выражение через элементарные симметрические многочлены будет выглядет как
    $$
    F = F_1 + F_2 + \ldots + F_N.
    $$

    Теперь докажем, что многочлен $F$ определён однозначно. Предположим, что $F$ и $G$ --- такие многочлены, что
    $$
    F(\sigma_1, \sigma_2, \ldots, \sigma_n) = G(\sigma_1, \sigma_2, \ldots, \sigma_n).
    $$
    Рассмотрим разность $H = F - G$. Тогда
    $$
    H(\sigma_1, \sigma_2, \ldots, \sigma_n) = 0.
    $$

    Нам нужно доказать, что $H = 0$. Предположим, что это не так, и пусть $H_1, H_2, \ldots, H_s$ --- все ненулевые члены многочлена $H$. Обозначим через $w_i$ ($i = 1, 2, \ldots, s$) старший член многочлена
    $$
    H_i(\sigma_1, \sigma_2, \ldots, \sigma_n) \in \mathbb{F}[x_1, x_2, \ldots, x_n].
    $$

    В силу леммы 2 среди одночленов $w_1, w_2, \ldots, w_s$ нет пропорциональных. Выберем из них старший. Пусть это будет $w_1$. По построению одночлен $w_1$ старше всех остальных членов многочлена $H_1(\sigma_1, \sigma_2, \ldots, \sigma_n)$ и всех членов многочленов $H_i(\sigma_1, \sigma_2, \ldots, \sigma_n)$ ($i = 2, \ldots, s$). Поэтому после приведения подобных членов в сумме
    $$
    H_1(\sigma_1, \sigma_2, \ldots, \sigma_n) + H_2(\sigma_1, \sigma_2, \ldots, \sigma_n) + \ldots + H_s(\sigma_1, \sigma_2, \ldots, \sigma_n) = H(\sigma_1, \sigma_2, \ldots, \sigma_n)
    $$
    член $w_1$ не сократится, так что эта сумма не будет нулевой, что противоречит нашем предположению.
\end{proof}


