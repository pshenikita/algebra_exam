\section{Инверсии. Чётность перестановки и подстановки. Знак подстановки, изменение чётности при умножениии на транспозицию. Разложение подстановки на транспозиции. Знак произведения подстановок}

\begin{definition}
    В подстановке $\sigma$ положение двух элементов $\sigma(i)$ и $\sigma(j)$ ($i < j$) называется \textbf{порядком}, если $\sigma(i) < \sigma(j)$ и \textbf{инверсией}, если $\sigma(i) > \sigma(j)$.
\end{definition}

\begin{definition}
    \textbf{Чётностью подстановки} называется чётность общего числа инверсий в ней. \textbf{Знак подстановки} $\sgn\sigma \vcentcolon = (-1)^{\text{чётность $\sigma$}}$.
\end{definition}

\begin{remark}
    Определения для перестановок даются так же, как для соответствующих им подстановок.
\end{remark}

\begin{lemma}
    При умножении подстановки $\sigma$ на транспозицию $[i, j]$ слева меняет местами числа $i$ и $j$, а справа --- меняет местами числа $\sigma(i)$ и $\sigma(j)$.
\end{lemma}

\begin{proof}
    Если $\sigma(x) \notin \{i, j\}$, то $([i, j]\circ\sigma)(x) = \sigma(x)$. Если же $\sigma(x) = i$, то $([i, j]\circ\sigma)(x) = j$ и наоборот. Это доказывает первое утверждение.

    Если $x \notin \{i, j\}$, то $[i, j](x) = x$. Теперь $(\sigma\circ[i, j])(i) = \sigma(j)$, а $(\sigma\circ[i, j])(j) = \sigma(i)$.
\end{proof}

\begin{theorem}
    Чётность подстановки меняется при умножении на транспозицию.
\end{theorem}

\begin{proof}
    При транспозиции соседних элементов меняется взаимное расположение только этих элементов, так что число инверсий изменяется (увеличивается или уменьшается) на $1$; следовательно, чётность меняется. Теперь заметим, что
    $$
    [i, j] = [i, i + 1] \circ [i + 1, i + 2] \circ \ldots \circ [j - 1, j] \circ [j - 2, j - 1] \circ \ldots \circ [i, i + 1].
    $$

    То есть, транспозиция несоседних элементов раскладывается в произведение $2(j - i) + 1$ соседних элементов. При каждой из них меняется чётность подстановки, а всего их нечётное количество, поэтому в итоге чётность изменится.
\end{proof}

\begin{theorem}
    Любая подстановка раскладывается в произведение транспозиций.
\end{theorem}

\begin{proof}
    Любая подстановка раскладывается в произведение независимых циклов. А каждый цикл раскладывается в произведение транспозиций:
    $$
    [i_1, i_2, \ldots, i_k] = [i_2 \circ i_1] \circ [i_3 \circ i_2] \circ \ldots \circ [i_k, i_{k - 1}].
    $$
\end{proof}

\begin{remark}
    Из двух предыдущих теорем следует, что если $\sigma = \sigma_1 \cdot \ldots\ \cdot \sigma_N$ --- разложение подстановки в произведене транспозиций, то $\sgn\sigma = (-1)^N$. А для цикла (разложение на транспозиции которого приводилось выше) $\sgn[i_1, \ldots, i_k] = (-1)^{k - 1}$.
\end{remark}

\begin{statement}[Задача из Кострикина]
    Доказать, что любая чётная подстановка представима в виде произведения циклов длины 3.
\end{statement}

\begin{proof}
    Чётная подстановка раскладывается в произведения чётного числа транспозиций. Осталось заметить, что
    $$
    [i_1, i_2] \circ [i_3, i_4] = [i_1, i_2] \circ \underbrace{[i_1, i_3] \circ [i_1, i_3]}_{\varepsilon} \circ [i_3, i_4] = [i_1, i_2, i_3] \circ [i_1, i_3, i_4].
    $$
\end{proof}

\begin{theorem}
    $\sgn(\sigma \circ \delta) = \sgn\sigma \cdot \sgn\delta$.
\end{theorem}

\begin{proof}
    Разложим подстановки в условии в произведения подстановок:
    $$
    \sigma = \sigma_1 \circ \ldots \circ \sigma_N,\quad\delta = \delta_1 \circ \ldots \circ \delta_M,
    $$
    Тогда
    $$
    \sgn(\sigma \circ \delta) = \sgn(\sigma_1 \circ \ldots \circ \sigma_N \circ \delta_1 \circ \ldots \circ \delta_M) = (-1)^{N + M} = (-1)^N \cdot (-1)^M = \sgn\sigma \cdot \sgn\delta.
    $$
\end{proof}

\begin{remark}
    Как следствие, $\sgn\sigma^{-1} = \sgn\sigma$, т.\,к.
    $$
    1 = \sgn\varepsilon = \sgn(\sigma \circ \sigma^{-1}) = \sgn\sigma \cdot \sgn\sigma^{-1}
    $$
    и $\sgn(\sigma \circ \delta) = \sgn(\delta \circ \sigma)$. Отсюда же можно извлечь ещё одно полезное следствие:
\end{remark}

\begin{theorem}
    При $n > 1$ число чётных подстановок равно числу нечётных.
\end{theorem}

\begin{proof}
    Докажем, что отображение $\sigma \overset{\varphi}{\mapsto} [1, 2]\circ\sigma$ является биекцией из множества чётных подстановок в множество нечётных. Из уже доказанного, если $\sigma$ чётная, то $[1, 2]\circ\sigma$ нечётная. Докажем же, что $\varphi$ --- биекция. Во-первых, $\varphi$ --- сюръекция, действительно, каждая нечётная подстановка $\sigma$ является образом чётной подстановки $[1, 2]\circ\sigma$. Во-вторых, $\varphi$ --- инъекция, так как, очевидно, $\sigma \ne \delta \Rightarrow [1, 2]\circ\sigma \ne [1, 2]\circ\delta$.
\end{proof}

\begin{remark}
    Множество чётных подстановок порядка $n$ обозначается $A_n$. Из уже доказанного, легко увидеть, что они образуют группу по операции композиции.
\end{remark}


