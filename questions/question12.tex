\section{Линейное отображение. Его матрица в фиксированных базисах. Образ заданного вектора. Изоморфизм. Любое конечномерное пространство изоморфно пространству строк}

\begin{definition}
    Отображение $\varphi: U \rightarrow V$ ($U$ и $V$ --- векторные пространства) называется \textbf{линейным}, если $\varphi(u_1 + u_2) = \varphi(u_1) + \varphi(u_2)$ и $\varphi(\lambda u) = \lambda\varphi(u)$.
\end{definition}

\begin{definition}
    Фиксируем базис $e = \{e_1, \ldots, e_n\}$ в $U$ и базис $f = \{f_1, \ldots, f_m\}$. Матрица $A(\varphi, e, f) = (a_{ij})$, по столбцам которой стоят координаты образов базисных векторов из $e$ в базисе $f$, называется \textbf{матрицей линейного отображения $\varphi$}.
\end{definition}

\begin{remark}
    Размеры матрицы линейного отображения --- $m \times n$.
\end{remark}

\begin{statement}
    $\varphi(v) = A \cdot v$, где $A$ --- матрица линейного отображения $\varphi$.
\end{statement}

\begin{proof}
    С одной стороны,
    $$
    \varphi(v) = \varphi\left(\sum_i^kv_ie_i\right) = \sum_i^kv_i\varphi(e_i).
    $$

    А с другой стороны,
    $$
    A \cdot v = 
    \begin{pmatrix}
        f(e_1) & f(e_2) & \ldots & f(e_k)
    \end{pmatrix} \cdot
    \begin{pmatrix}
        v_1\\
        v_2\\
        \vdots\\
        v_k
    \end{pmatrix} = \sum_i^kv_i\varphi(e_i).
    $$

    Отсюда сразу следует $\varphi(v) = A \cdot v$.
\end{proof}

\begin{definition}
    Векторные пространства $U$ и $V$ над полем $\mathcal{K}$ называются \textbf{изоморфными}, если существует линейное биективное отображение $\varphi: U \rightarrow V$. Само отображение $\varphi$ называется при этом \textbf{изоморфизмом} пространств $V$ и $U$.
\end{definition}

См. также лемму 8.1.


