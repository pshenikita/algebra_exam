\section{Результант. Свойства результанта. Связь результанта многочлена и его производной с дискриминантом многочлена. Выражение результанта через определитель (формулировка)}

Пусть даны два многочлена из $\mathbb{F}[x]$:
$$
f = a_0x^n + a_1x^{n - 1} + \ldots + a_n,\quad g = b_0x^m + b_1x^{m - 1} + \ldots + b_m.
$$
Будем считать, что у них количество корней с учётом кратности равно степени:
$$
f = a_0\prod_{i = 1}^N(x - x_i),\quad g = b_0\prod_{j = 1}^m(x - y_j).
$$

\begin{definition}
    Результантом $R(f, g)$ многочленов $f$ и $g$ называется число
    $$
    a_0^mb_0^n\prod_{1 \leqslant i \leqslant n, 1 \leqslant j \leqslant m}(x_i - y_j).
    $$
\end{definition}

\begin{theorem}[Свойства результанта]
    \begin{enumerate}[nolistsep]
        \item $R(f, g) = 0$ тогда и только тогда, когда $f$ и $g$ имеют общий корень (<<основное свойство результанта>>);
        \item $R(g, f) = (-1)^{mn}R(f, g)$;
        \item $\displaystyle R(f, g) = a_0^m\prod_{i = 1}^ng(x_i) = (-1)^{mn}b_0^n\prod_{j = 1}^mf(y_j)$.
    \end{enumerate}
\end{theorem}

\begin{proof}
    Свойства 1 и 2 сразу следуют из определения. Второе равенство свойства 3 доказывать тоже не нужно --- оно сразу следует из свойства 2. Итак, докажем первое равенство свойства 3:
    $$
    a_0^m\prod_{i = 1}^ng(x_i) = a_0^m\prod_{i = 1}^n\left(b_0\prod_{j = 1}^m(x_i - y_j)\right) = a_0^mb_0^n\prod_{1 \leqslant i \leqslant n, 1 \leqslant j \leqslant m}(x_i - y_j) = R(f, g).
    $$
\end{proof}

\begin{theorem}
    $$
    R(f, f^\prime) = (-1)^{\frac{n(n - 1)}{2}}a_0D(f).
    $$
\end{theorem}

\begin{proof}
    $f = a_0(x - x_1)\ldots(x - x_n)$. Тогда
    $$
    f^\prime(x) = a_0\sum_{i = 1}^n\frac{\prod(x - x_j)}{x - x_i}.
    $$
    Имеем
    $$
    f^\prime(x_i) = a_0\prod{i \ne j}(x_i - x_j).
    $$
    Теперь докажем равенство лобовым вычислением:
    $$
    \begin{array}{c}\displaystyle
        R(f, f^\prime) = a_0^{n - 1}\prod_{i = 1}^nf^\prime(x_i) = a_0^{2n - 1}\prod_{i = 1}^n\prod_{i \ne j}(x_i - x_j) = a_0^{2n - 1}\prod_{i < j}(x_i - x_j)\prod_{i > j}(x_i - x_j) =\\\displaystyle a_0^{2n - 1}\prod_{i < j}(x_i - x_j)(-1)^{\frac{n(n - 1)}{2}}\prod_{i < j}(x_i - x_j) = (-1)^{\frac{n(n - 1)}{2}}a_0^{2n - 1}\prod_{i < j}(x_i - x_j)^2 = (-1)^{\frac{n(n - 1)}{2}}a_0D(f).
    \end{array}
    $$
\end{proof}

\begin{theorem}
    $$
    R(f, g) = \det
    \begin{pmatrix}
        a_0 & a_1 & a_2 & \cdots & a_n & 0 & 0 & 0 & \cdots & 0\\
        0 & a_0 & a_1 & a_2 & \cdots & a_n & 0 & 0 & \cdots & 0\\
        0 & 0 & a_0 & a_1 & a_2 & \cdots & a_n & 0 & \cdots & 0\\
        \vdots & \vdots & \vdots & \ddots & \vdots & \ddots & \vdots & \vdots & \ddots & \vdots\\
        0 & 0 & 0 & \cdots & 0 & a_0 & a_1 & a_2 & \cdots & a_n\\
        b_0 & b_1 & b_2 & \cdots & b_n & 0 & 0 & 0 & \cdots & 0\\
        0 & b_0 & b_1 & b_2 & \cdots & b_n & 0 & 0 & \cdots & 0\\
        0 & 0 & b_0 & b_1 & b_2 & \cdots & b_n & 0 & \cdots & 0\\
        \vdots & \vdots & \vdots & \ddots & \vdots & \ddots & \vdots & \vdots & \ddots & \vdots\\
        0 & 0 & 0 & \cdots & 0 & b_0 & b_1 & b_2 & \cdots & b_n\\
    \end{pmatrix}
    $$
\end{theorem}

Дополнительные билеты допишу потом.


