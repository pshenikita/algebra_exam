\section{Вывод обобщённой ассоциативности для ассоциативной операции}

\begin{remark}
    Здесь для удобства часто будем называть операцию умножением.
\end{remark}

\begin{definition}
    Ассоциативная операция называется \textbf{обобщённо ассоциативной}, если произведение любого количества сомножителей не зависит от расстановки скобок в этом произведении.
\end{definition}

\begin{definition}
    Назовём \textbf{стандартной} такую расстановку скобок, в которой операции выполняются слева направо.
\end{definition}

\begin{theorem}
    Ассоциативная операция всегда обобщённо ассоциативна.
\end{theorem}

\begin{proof}
    Доказательство проведём индукцией по количеству множителей $k$:
    
    \textbf{База индукции} ($k = 3$). Верно в силу ассоциативности.

    \textbf{Шаг индукции}. Пусть для $k < n$ утверждение верно. Докажем его для $k = n$. Докажем, что для произвольной расстановки скобок результат будет таким же, как и для стандартной. Рассмотрим некоторую расстановку скобок в произведении $x_1 \ast \ldots \ast x_n$. Пусть последнее умножение перемножает скобки $A$ и $B$. Рассмотрим два случая:
    \begin{enumerate}
        \item $B = x_n$. Тогда для количество множителей в скобке $A$ равно $n - 1 < n$, а потому по предположению индукции можем стандартно расставить скобки в ней. Но тогда и во всё выражении они будут расставлены стандартно и теорема доказана.
        \item $B \ne x_n$. Внутри $B$ по предположению индукции расстановку скобок можно принять стандартной, а потому $B = C \ast x_n$. А отсюда
            $$
            A \ast B = A \ast (C \ast x_n) = (A \ast C) \ast x_n
            $$
            из ассоциативности, и мы попали в предыдущий случай.
    \end{enumerate}
\end{proof}

