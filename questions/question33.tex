\section{Следствия из теоремы Лагранжа}

\begin{theorem}
    Порядок любой подгруппы конечной группы делит порядок группы.
\end{theorem}

\begin{proof}
    По теореме Лагранжа $|G| = |G \vcentcolon H| \cdot |H|$, отсюда $|H| \mid |G|$.
\end{proof}

\begin{theorem}
    Порядок любого элемента конечной группы делит порядок группы.
\end{theorem}

\begin{proof}
    Это вытекает из предыдущей теоремы и того, что порядок элемента равен порядку порождаемой им циклической подгруппы.
\end{proof}

\begin{theorem}
    Всякая конечная группа простого порядка является циклической.
\end{theorem}

\begin{proof}
    В силу теоремы 33.1 такая группа должа совпадать с циклической подгруппой, порождённой любым элементом, отличным от 1.
\end{proof}

\begin{theorem}
    Если $|G| = n$, то $g^n = e$ для любого $g \in G$.
\end{theorem}

\begin{proof}
    Пусть $\ord g = m$. В силу следствия 2 имеем $m \mid n$. Значит, $g^n = e$.
\end{proof}

\begin{lemma}
    Если $p$ --- простое число, то мультипликативная группа $\Z_p^\ast$ поля $\Z_p$ есть (абелева) группа порядка $p - 1$.
\end{lemma}

\begin{proof}
    Сначала докажем, что $\Z_p^\ast = \{1, 2, \ldots, p - 1\}$ (т.\,е., обратимы все, кроме $0$). Пусть $m$ --- число, не делящееся на $p$. Тогда
    $$
    \gcd(m, p) = 1 \Rightarrow \exists\!\:u, v \in \Z: um + vp = 1.
    $$

    Возьмём $\mod p$ от обеих частей. Получаем $um = 1\ (\mod p)$, отсюда $[u][m] = 1$, и мы нашли обратный. А утверждение следует напрямую из леммы 34.1.
\end{proof}

\begin{theorem}[Малая теорема Ферма]
    Пусть $p$ --- простое число и пусть $a \in \Z$, $p \nmid a$. Тогда
    $$
    a^{p - 1} \equiv 1\ (\mod p).
    $$
\end{theorem}

\begin{proof}
    Если $p$ --- простое число, то мультипликативная группа $\Z_p^\ast$ поля $\Z_p$ есть (абелева) группа порядка $p - 1$ (из леммы выше). Следовательно, $g^{p - 1} = 1$ для любого элемента $g \in \Z_p^\ast$. Это означает, что
    $$
    a^{p - 1} \equiv 1\ (\mod p).
    $$
\end{proof}

\begin{definition}[Функция Эйлера]
    Для любого $n$ порядок группы $\Z_n^\ast$ обратимых элементов кольца $\Z_n$, равный количеству чисел в ряде $1, 2, \ldots, n$, взаимно простых с $n$, обозначается через $\varphi(n)$.
\end{definition}

\begin{theorem}[Эйлер]
    Пусть $a, n \in \N$ и $\gcd(a, n) = 1$. Тогда
    $$
    a^{\varphi(n)} \equiv 1\ (\mod n).
    $$
\end{theorem}

\begin{proof}
    Применяя теорему 33.4 к группе $\Z_n^\ast$, получаем требуемое.
\end{proof}


