\section{Однородные системы с количеством неизвестных большим количества уравнений. Основная лемма о линейной зависимости}

\begin{definition}
    СЛУ называется однородной, если свободные члены во всех уравнениях равны нулю.
\end{definition}

\begin{orangebox}
    Однородные СЛУ всегда совместны, в качестве решения подходит строка $(0, \ldots, 0) \in \mathcal{K}^n$.
\end{orangebox}

\begin{lemma}
    Если в однородной системе неизвестных больше, чем уравнений, то она имеет ненулевое решение.
\end{lemma}

\begin{proof}
    Количество главных переменных равно количеству лидеров, а оно, в свою очередь, не превышает количества уравнений (строк матрицы коэффициентов), а оно меньше количества неизвестных. Значит, количество главных переменных меньше количества всех переменных, значит, среди переменных есть свободные, т.\,е. решений этой системы бесконечно много. Среди них можно выбрать ненулевое.
\end{proof}

\begin{lemma}[Основная лемма о линейной зависимости]
    Пусть $\{v_1, \ldots, v_n\}$ и $\{u_1, \ldots, u_m\}$ --- две системы векторов из $V$. Допустим, что каждый вектор $v_i$ линейно выражается через систему векторов $\{u_1, \ldots, u_m\}$ и при этом $n > m$. Тогда система $\{v_1, \ldots, v_n\}$ линейно зависима.
\end{lemma}

\begin{proof}
    Из условия, существуют такие $\lambda_{ij}$, что
    $$
    v_i = \sum_{j = 1}^m\lambda_{ij}u_j.
    $$


    Составим линейную комбинацию
    $$
    \sum_{i = 1}^n\mu_iv_i = \sum_{i = 1}^n\left(\mu_i\sum_{j = 1}^m\lambda_{ij}u_j\right) = \sum_{i = 1}^n\sum_{j = 1}^m(\mu_i\lambda_{ij}u_j) = \sum_{j = 1}^m\sum_{i = 1}^n(\mu_i\lambda_{ij}u_j) = \sum_{j = 1}^m\left(\sum_{i = 1}^n\mu_i\lambda_{ij}\right)u_j.
    $$

    Эта линейная комбинация (уж точно) равна нулю, если $\displaystyle \sum_{i = 1}^n\mu_i\lambda_{ij} = 0$ $\forall\!\:j$. Докажем, что мы можем подобрать подходящие $\mu_i$. Имеем однородную СЛУ с переменными $\mu_i$ ($n$ штук) и коэффициентами $\lambda_{ij}$ ($m$ штук), причём $n > m$, значит (по предыдущей лемме), у этой системы есть ненулевое решение. Итак, существуют такие коэффициенты $\mu_i$, не все равные нулю, что $\mu_1v_1 + \ldots + \mu_nv_n = 0$.
\end{proof}

