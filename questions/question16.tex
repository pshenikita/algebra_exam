\section{Верхние оценки на ранг суммы и произведения матриц}

Антон Александрович задавал нам на дом вывести и нижние оценки, поэтому они здесь будут.

\begin{statement}
    $|\rk A - \rk B| \leqslant \rk(A + B) \leqslant \rk A + \rk B$.
\end{statement}

\begin{proof}
    Сначала докажем оценку сверху. Строки матрицы $A + B$ --- это линейные комбинации строк матриц $A$ и $B$. А значит, базис строк $A + B$ --- линейно независимая система, линейно выражающаяся через базисы строк $A$ и $B$. А значит, по основной лемме о линейной зависимости $\rk(A + B) \leqslant \rk A + \rk B$.

    Оценка снизу равносильна системе
    $$
    \begin{cases}
        \rk A - \rk B \leqslant \rk(A + B),\\
        \rk B - \rk A \leqslant \rk(A + B)
    \end{cases}
    $$

    Докажем первое из неравенств системы, второе доказывается аналогично. Итак, из оценки сверху:
    $$
    \rk(A + B) + \rk B = \rk(A + B) + \rk(-B) \leqslant \rk A.
    $$

    Перенеся слагаемые в нужные стороны, получим то, что хотели.
\end{proof}

\begin{remark}
    $\rk A = \rk (\lambda A)$ ($\lambda \ne 0$), т.\,к. по сути умножение матрицы на ненулевое число --- это умножение каждой из её строк на это число, а это преобразование является элементарным.
\end{remark}

\begin{statement}
    $\rk A + \rk B - n \leqslant \rk AB \leqslant \min\{\rk A, \rk B\}$, где $\underset{m \times n}{A}$ и $\underset{n \times k}{B}$.
\end{statement}

\begin{lemma}
    Столбцы $AB$ --- линейные комбинации столбцов $A$ с коэффициентами из столбцов $B$. Строки $BA$ --- это линейные комбинации строк $B$ с коэффициентами из строк $A$.
\end{lemma}

\begin{remark}
    Далее за $X^{(i)}$ будем обозначать $i$-ый столбец матрицы $X$, а за $X_{(i)}$ --- её $i$-ую строку.
\end{remark}

\begin{proof}
    Докажем утверждение непосредственным умножением:
    $$
    (AB)^{(j)} = 
    \begin{pmatrix}
        \displaystyle\sum_{t = 1}^n a_{1t}b_{tj}\\
        \displaystyle\sum_{t = 1}^n a_{2t}b_{tj}\\
        \vdots\\
        \displaystyle\sum_{t = 1}^n a_{mt}b_{tj}
    \end{pmatrix} =
    \sum_{t = 1}^nb_{tj} A^{(t)}.
    $$

    Второе утверждение доказывается аналогично.
\end{proof}

Теперь докажем оценки на ранг произведения:

\begin{proof}
    Сначала докажем верхнюю оценку. Система столбцов матрицы $AB$ линейно выражается через столбцы $A$, поэтому (из основной леммы о линейной зависимости) $\rk AB \leqslant \rk A$. Аналогично, $\rk AB \leqslant \rk B$. Отсюда сразу следует требуемое.
    Теперь докажем и нижнюю оценку. Для этого рассмотрим матрицу
    $$
    \begin{pmatrix}
        \underset{n \times n}{E} & \underset{n \times k}{0}\\
        \underset{m \times n}{0} & \underset{m \times k}{AB}\\
    \end{pmatrix}
    $$

    размера $(m + n) \times (n + k)$. Как нетрудно заметить, её ранг равен $\rk AB + \rk E$. Из леммы 16.1, можно производить элементарные преобразования не над отдельными её строками, а над блоками из матриц (Антон Александрович называл их <<гиперстроками>>), ведь такие преобразования можно разбить на цепочки преобразований обычных строк. Итак, элементарными преобразованиями (которые, как известно, не меняют ранг) переведём нашу матрицу в такую:
    $$
    \begin{pmatrix}
        E & 0\\
        0 & AB\\
    \end{pmatrix} \leadsto
    \begin{pmatrix}
        E & 0\\
        A & AB
    \end{pmatrix} \leadsto
    \begin{pmatrix}
        E & -B\\
        A & 0
    \end{pmatrix} \leadsto
    \begin{pmatrix}
        E & B\\
        A & 0
    \end{pmatrix}
    $$

    Теперь докажем, что
    $
    \rk
    \begin{pmatrix}
        E & B\\
        A & 0
    \end{pmatrix} \geqslant \rk A + \rk B
    $. Приведём нашу блочную матрицу к улучшенному ступенчатому виду. Для этого можно брать строки единичной матрицы и вычитать их из строк матрицы $A$, обнуляя их. Это возможно, т.\,к. строки матрицы $E$ образуют стандартный базис в $\mathcal{K}^n$, а потому строки $A$ точно выражаются как линейные комбинации строк $E$. Что же будет при этом происходить со вторым <<гиперстолбцом>>? Там мы просто каждый раз будем из нулевой матрицы вычитать $B$ с каким-то коэффициеном. А потому результат будем $\lambda B$. Итак, улучшенный ступенчатый вид нашей матрицы:
    $$
    \begin{pmatrix}
        E & B\\
        0 & \lambda B
    \end{pmatrix}.
    $$

    Её ранг равен количеству ненулевых строк, т.\,е. $n + \rk B$. А из неравенства $\rk A \geqslant n$ получаем то, что хотели. Итак,
    $$
    \rk AB + n = \rk AB + \rk E = \rk
    \begin{pmatrix}
        E & 0\\
        0 & AB
    \end{pmatrix} \geqslant \rk A + \rk B.
    $$

    Перенеся слагаемые в нужные стороны, получим то, что хотели.
\end{proof}


