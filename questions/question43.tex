\section{Кратные корни многочлена. Сумма кратностей не превышает степень многочлена. Формальное и функциональное равенство многочленов от одной переменной}

\begin{definition}
    Будем говорить, что \textbf{многочлен $f$ имеет корень кратности $k$}, если он может быть представлен в виде $f(x) = (x - a)^kq(x)$ и не может быть представлен в виде $(x - a)^{k + 1}r(x)$.
\end{definition}

\begin{statement}
    Любой многочлен $f \in \C[z]$ степени $n$ раскладывается на линейные множители с коэффициентами из $\C$.
\end{statement}

\begin{proof}
    Следствие основой теоремы алгебры и теоремы Безу.
\end{proof}

\begin{statement}
    Сумма кратностей корней многочлена не превышает степень многочлена.
\end{statement}

\begin{proof}
    Из уже доказанного следует, что многочлен $f$ (с коэффициентами из $\R$ или $\C$) степени $n$ пишется как
    $$
    f(x) = a(x - c_1)^{k_1}\ldots(x - c_n)^{k_n}.
    $$
    Отсюда сразу следует требуемое.
\end{proof}

\textbf{Формальное и функциональное равенство многочленов} --- не одно и то же. Например, многочлены $f(x) = x$ и $f(x) = x^2$ над полем $\Z_2$ задают одинаковые функции, но многочлены это, очевидно, разные (не совпадают задающие их финитные последовательности). Если $\mathcal{R}$ --- конечная область целостности, то многочлен
$$
f(x) = \prod_{r \in \mathcal{R}}(x - r)
$$
задаёт тождественно нулевую функцию, хотя сам многочлен ненулевой. По малой теореме Ферма многочлены $x^p$ и $x$ задают одну и ту же функцию над полем $\Z_p$.

\begin{theorem}
    Пусть $\mathcal{R}$ --- бесконечная область целостности. Тогда из функционального равенства многочленов из $\mathcal{R}[x]$ следует их формальное равенство.
\end{theorem}

\begin{proof}
    Пусть многочлены $f$ и $g$ определяют одну и ту же функцию. Тогда их разность $h = f - g$ определяет нулевую функцию, т.\,е. $h(c) = 0$ для всех $c \in \mathcal{R}$. Предположим, что $h \ne 0$, и пусть
    $$
    h = a_0 + a_1x + a_2x^2 + \ldots + a_{n - 1}x^{n - 1}\quad(a_{n - 1} \ne 0).
    $$
    возьмём различные $x_1, x_2, \ldots, x_n \in \mathcal{R}$ (здесь используется бесконечность области целостности $\mathcal{R}$). Совокупность равенств
    $$
    \begin{cases}
        a_0 + a_1x_1 + a_2x_1^2 + \ldots + a_{n - 1}x_1^{n - 1} = 0,\\
        a_0 + a_1x_2 + a_2x_2^2 + \ldots + a_{n - 1}x_2^{n - 1} = 0,\\
        \ldots\\
        a_0 + a_1x_n + a_2x_n^2 + \ldots + a_{n - 1}x_n^{n - 1} = 0.
    \end{cases}
    $$
    будем рассматривать как (квадратную) однородную СЛУ относительно $a_0, \ldots, a_{n - 1}$. Определитель матрицы коэффициентов этой системы есть определитель Вандермонда и потому отличен от нуля. Следовательно, система имеет только нулевое решение, что противоречит нашему предположению.
\end{proof}


